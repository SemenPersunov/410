\documentclass[a4paper]{book}
\usepackage[T2A]{fontenc}
\usepackage[utf8]{inputenc}
\usepackage[english,russian]{babel}
\usepackage[unicode]{hyperref}
\usepackage{indentfirst}
%\usepackage{citehack}


\title{410}
\author{Семен Персунов}
\begin{document} 
\maketitle

\tableofcontents


\chapter{Пролог}
Автостанция встречала меня толпами народа и горами мусора. Туда-сюда сновали южане-носильщики, расталкивая люд своими тележками. 
 
-- Пастаранысь, дарогу-дарогу! Насильщики, такси нэдораго!

Посадки ждать еще полчаса. А потом почти тридцать часов тряски по родным просторам в компании других студентов, что едут из столицы домой. Хорошо хоть не среди пьяных челноков.

Багажа у меня немного: лишь небольшой рюкзачок, в котором лежат лаптоп да несколько полезных в быту вещиц. Еды почти нет, проще перекусить на остановке. А так, хватит яблок и  воды. 

Началась процедура сдачи багажа. Люди передают сумки крепенькому низкорослому водителю, который споро расставляет их по объемистому багажному отделению. А вот и открылась дверь в салон, можно располагаться. 

Мечтать о том, что удастся ехать одному на двух сиденьях, не приходится, народу много. Празднички. Надеюсь, хоть сосед будет приличный. Рюкзак закинут под кресло, теперь можно и устроиться поудобнее. Насколько это возможно, очень уж тесно в таких автобусах. Такое впечатление, что их проектируют для лилипутов.

-- Молодой человек, подвиньтесь! -- Все надежды на комфортную поездку пошли прахом. Тетенька своими статями напоминала бегемота, вставшего на задние лапы. Как бы под ней кресло-то не развалилось. Наверняка будет храпеть, пихаться и портить воздух. 

Проверка билетов. Контрольный обход салона. Никого не забыли? Ну, поехали! Можно и яблочко пожевать. Отрезаю ножиком небольшие кусочки и ем. 

-- Молодой человек, а зачем вам такой нож? -- раздался пронзительный голос со стороны прохода.

-- Чтобы резать. -- каков вопрос, таков ответ.

Поел -- можно и поспать. Что еще для счастья надо? Чтобы в бок не тыкали. И звука очередного шедевра отечественного кинематографа, коим экипаж нашего дорожного лайнера развлекает пассажиров, не слышать. Впрочем, последнее -- не проблема. Заряда плеера хватит на всю дорогу да еще и останется. Звукоизоляция маленьких, но удаленьких внутриканальных наушников практически идеальна, а музыка, хорошая музыка сну не преграда. Включив любимый эмбиент, я отключился.

\paragraph{}<++>

Это пробуждение не похоже на предыдущие. Не ощущается движения автобуса, не
ворочается похожая на тролля соседка\ldots Плеер молчит. Странно, аккумулятора должно было хватить, как минимум, еще часов на пять. 

Холодно. Затекло все тело, болит, кажется, каждая клеточка. Под боком что-то жесткое и неприятное. Да это подлокотник, а лежу я на боку. Как и весь автобус. Проклятье!

Лицо покрыто коркой запекшейся крови. Должно быть, хорошо обо что-то приложился. Да и битого стекла везде хватает. Руки и ноги, вроде, целы.

Темно. Что-то оттягивает ногу. Лямка рюкзака. Всегда перекидываю ее через ногу, когда кладу рюкзак под сиденье. Вот сейчас бы дотянуться\ldots Не очень-то удобно вытаскивать рюкзак из-под сиденья лежа на боку. Расстегнуть фастекс, потянуть.. Вот так!

Визг молнии, и в руке аккумуляторный светодиодный фонарик. Ослепительный луч света рассекает темноту, освещая пустой салон. Мертвых пассажиров не видно, их вещей тоже. Лезут нехорошие мысли: <<Авария, все ушли, эвакуировались. А почему меня оставили? Ай-ай-ай>>.

Какого черта? Где я? Это не двухэтажный <<Неоплан>>. Что-то старое. Перед глазами проносятся воспоминания из детства. Прильнув к окну, наблюдаю многометровый обрыв, здоровенные каменюки и бурную горную речку. Кажется, что стоит только пошевелиться, чуть сместить центр масс автобуса, и мы опрокинемся и полетим вниз. Дорога в Енгиабад поначалу всегда пугала. А ходили туда <<ЛАЗы>> и <<Икарусы>>. Точно -- <<Икарус>>.

Панику задушить в зародыше. Аптечка! Нужна аптечка. Она обязательно должна быть в автобусе. Теоретически. Вскарабкаться на боковину кресла, шатаясь из стороны в сторону, перебраться на следующее. Все ближе и ближе к отгороженной фанеркой кабине. Еще ближе, еще. 

По лбу что-то течет прямо в глаза. Рана начала кровить. В узком пучке света вижу красную клеенчатую сумку с белым крестом. Висит неудобно. Хотя чего уж там. Когда автобус лежит на боку, есть ли смысл говорить об удобстве? 

Есть! Дотянулся. Ассортимент медикаментов не радует: зеленка да пачка стерильных бинтов. Какие-то таблетки в белой бумажной упаковке. Активированный уголь. Ну хоть что-то. Не до хорошего, того же промедола теперь и у вояк не сыщешь. Обработал порезы, стер кровь влажным носовым платком. Как хорошо, что всю воду из фляжки я не успел выпить.

Надо выбираться. Как там, выдернуть шнур, выдавить стекло? Здесь оно, по крайней мере, на меня не упадет. А где этот шнур-то? 
\\
\paragraph{}<++>


Место, где я оказался, не понравилось сразу. Дорога старая и разбитая. Такое впечатление, что по ней не ездили лет двадцать. Обошел автобус, принюхался. Да вроде ничего. В смысле, совсем ничего: в свете фонарика не удалось разглядеть тормозного пути. 

Автобус-то -- действительно <<Икарус>>. Красный с белым. Именно такой, как я помню. Тормозного пути нет. Странно. 

А это что за дрянь? Вся поверхность дороги и земля вокруг была припорошена какой-то белесой субстанцией. Нет, пробовать на вкус это и даже трогать руками совершенно не хочется.

Не видно ни звезд, ни луны. Все затянуто какой-то мутной пеленой. 

Мобильник не ловит. Кто бы сомневался. Часы, кстати, показывают три часа ночи. И, судя по календарю, я провалялся в отключке трое суток. Парадоксально, но факт.

Холодно. Где там была моя флиска? Ох, как хорошо, что я не американский подросток из фильмов ужасов категории B. Проведем ревизию. Что там у меня в рюкзачке? Ага.
\begin{itemize}
	\item Собственно, флисовая кофта. 
	\item Лаптоп с зарядником, впрочем, толку от него сейчас немного.
	\item Запасной комплект аккумуляторов к фонарю, зарядник.
	\item Компактный мультимедийный плеер с хорошими наушниками. Вот уж, в данных условиях, вещи сомнительной полезности.
	\item Хороший американский мультитул в чехле из толстой кожи. Там же набор сменных бит. Наверняка пригодится. Повешу на пояс.
	\item Прочный складной нож с толстенным клинком и мощным замком. Сгодится в качестве оружия. Лучше, чем ничего.
	\item Сменное белье: три пары носков, трое трусов. 
	\item Где-то половина рулона туалетной бумаги. Пойдет, в том числе, и на растопку. Зажигалка тоже есть.
	\item Литровая фляжка с водой, воды меньше половины.
	\item Три яблока.
	\item Документы. 
\end{itemize}

Мда, не густо. А ведь становится все холоднее и холоднее. Костерок что ли зажечь? В автобусе-то сидеть совсем не хочется. Натаскаю набивки кресел, да подпалю. Вспороть ножом обивку, вытащить оттуда куски резиноподобного материала. Поджег для пробы зажигалкой -- ну и вонь! Нет, идея с костром явно никуда не годится. 

Надо бы в багажном отделении порыться, наверняка ведь что-нибудь полезное отыщется. Только, для начала, нужно влезть на боковушку автобуса. Матерки, кряхтение, лоб опять начал кровить. Разок соскользнул и чувствительно приложился копчиком. Но вот, Эверест взят. 

А что у нас в багаже? Багажное отделение было пусто. Такое впечатление, что все пассажиры организовано покинули автобус, получили свои чемоданы и сумки, а потом кто-то взял и зашвырнул автобус неведомо куда. Ну, и меня заодно. Прям руки опускаются. На тебе, мерзкий автобус, на. Ну вот, только ногу отбил о створку. 

Сел, свесив ноги. Яблочко-яблочко, ну хоть какая-то еда. И воды немножко. Посетили невеселые мысли о том, что же я буду жрать, когда кончатся сии скудные запасы. И сколько километров до ближайшего селения. Интересно, а приемник в плеере что-нибудь поймает? Попробую-ка, для начала, включить. Нажатие на кнопку и на экране появилась заставка загрузки. Есть заряд! Сканирование, сканирование, пусто. Проклятье! 

Остается ждать утра. Надеюсь, не замерзну. Кажется, сидя на поваленном на бок автобусе я задремал. Сквозь рваный и беспокойный сон чудилось, будто внизу что-то бродило. Стало страшно, и я перебрался подальше от края. Хотелось верить, что это вышли погулять ежики. Или мне просто показалось. 
\paragraph{}<++>


Рассвет. Или только кажется? Солнца не видно, но стало заметно светлее и не так холодно. Туман, впрочем, никуда не делся, хотя граница видимости заметно отодвинулась. Даже не туман, а дымка. Все-таки холодно. Удивительно, что я не замерз, ночуя на холодном металле. Время 7:30. 

Памятуя о бродящих неподалеку неведомых зверушках, разложил и взял наизготовку нож. Короткий, но толстый и широкий, с характерной овальной дыркой в клинке, крупной насечкой на обухе и удобной рукояткой с упором. Эдакий складной ломик. Придает уверенности, да еще как. Ух, теперь меня так просто не возьмешь!

С этой мыслью и спрыгнул с автобуса. Надо бы следы поискать, что же тут ночью бродило? Далеко ходить не пришлось, вот они, отпечатались в белесой дряни. И следы это явно собачьи. Может быть волки? Хотя, какая разница, одичавшие псы ничуть не лучше. 

Куда идти? Была не была -- двинусь в сторону предполагаемого движения автобуса. Ну, куда направлена кабина. Туман, или скорее дымка навевала неприятные чувства, ассоциации с <<туманом войны>> из компьютерных игр. Ощущение, что где-то там притаился враг. Рука судорожно стискивала рукоять ножа.

Сколько я иду? Уже почти полдень. Дорога несколько раз поворачивала, и честно говоря, я уже потерял ориентацию в пространстве. Как же надоел этот туман! Несколько раз оступился. Асфальт-то местами провалился, кое-где даже чахлая травка-муравка проросла.

Не покидает чувство опасности. Будто кто-то смотрит в спину голодным и злым взглядом. Обернешься -- никого. Постоишь, прислушаешься -- тихо, очень тихо. Ни звука ветра, ни пения птиц, ни кузнечиков, ничего. Гнетущая обстановка. Надо двигаться дальше. Привал? Главное -- не потерять направление. Рюкзак сойдет в качестве указателя на время отдыха. Долго сидеть не могу, накрывает страх и чувство опасности. Что-то идет следом, но пока не решается напасть. 

Устал. Болела нога -- подвернул, не углядев очередную яму. Воду экономлю, как могу, но на дне фляжки осталась лишь пара глотков. На сколько меня хватит? Чувство тревоги никак не отпускает, так и чувствуется на себе эдакий неприятный взгляд. Кто бы то ни был, но нападать он пока не спешит.

Впереди сквозь сгущающийся туман показались какие-то постройки. Я ускорил шаг. Да это же остановка. Старая, с отвалившимися облицовочными плитами и прогнившими деревянными скамейками. Сколько же лет прошло с тех пор, когда люди здесь ждали автобуса?

Бросил рюкзак ручкой в сторону движения. В животе урчит, ох, как же хочется есть. Осталось последнее яблоко, пора его приговорить. Однозначно, пора. Достав яблоко, я с аппетитом его съедаю, не утруждая себя нарезкой его на кусочки. Эх, ну вот и все. Впрочем, без еды можно протянуть гораздо дольше, чем без воды. 

Усталость. Усталость накатила внезапно, и такая, что я нашел в себе сил лишь сделать несколько шагов к стене остановки, и сесть на землю, привалившись к этой самой стене спиной. Глаза слипались. Я понимал, что поступаю глупо, что нельзя вот так устраивать ночлег. Особенно, памятуя о недобрых тварях, таящихся в тумане. Эти мысли проявлялись где-то на задворках сознания и уплывали вдаль. 
\\
\paragraph{}<++>


Вокруг клубящийся туман\ldots Видимость отвратная. Рядом виднеются чахлые, низкорослые деревья, лишенные листьев. Они, словно пытаются дотянуться до чего-то, схватить\ldots Под ногами противно чавкает грязь вперемешку с палой листвой, а мелкие колючие кустарники бессильно скребут по гетрам, пытаясь, видно, впиться в шнурки ботинок, намотать их на себя, обездвижить человека и впитать, растворить.. 

Небо скрыто серой пеленой. Ни звезд, ни луны, но какой-то источник света все-таки есть. Быть может, светится туман? 

Боковым зрением фиксируются какие-то движущиеся тени, но стоит присмотреться, повернув голову, и видно лишь проступающие сквозь туман деревья и кустарник. Ночные птицы, странные твари\ldots Чупакабра, снежный человек -- кто знает. 

Что я здесь делаю? Преследую. Кого? Не вижу следа, не чувствую запаха. Иду вперед по наитию. Или просто блуждаю кругами? Кто я? 

Туман сгущается, он все плотнее и плотнее, такой, что не увидишь и вытянутой руки. Гляжу на свои кисти -- левая обожжена, на ней расплавившиеся остатки перчатки. Мгновение, и руки тают в тумане\ldots
\\
\paragraph{}<++>


Пробуждение было внезапным. Чувство опасности просто оглушает, словно кричит: <<вставай, вставай или умрешь!>> Вскочил. Щелчок замка разложившегося ножа. Темно. Сколько же я спал? Тусклый луч фонаря в энергосберегающем режиме высветил лежащий на том же месте, где я его оставил, рюкзак. Дальше виднеются гнилые скамейки. 

Видимо, свет фонаря послужил сигналом. Что то прыгнуло в ноги и больно вцепилось в лодыжку. Мелкое, верткое. Фонарь от неожиданности упал на землю, а клинок ножа достал лишь воздух. Зато правая нога впечатала мелкую гадину в асфальт. Другая тварь прыгнула на грудь, но была сбита ударом рукоятки практически на излете. Собаки! Это была стая одичавших бобиков. Сколько же они за мной шли, не решаясь напасть, и караулили, когда я, наконец, усну?

С матерным воплем, я бросился наружу. На нож надеяться не приходится, попробуй достань коротким клинком собаку, что в холке тебе по колено. А вот топтать и пинать эту свору, оглашая окрестности истошными воплями мы можем. 

-- А-а-а-а, суки бля! А-а-а-а-а убью-ю-ю-ю нахуй! -- кажется, блохастые твари не ожидали такого напора и разбежались, поджав хвосты. Далеко ли?

Не долго думая, подпрыгнул, зацепившись руками за бетонную крышу остановки, и помогая себе ногами, втащил себя наверх. Ух, надо отдышаться. Без сил я повалился на бетонную панель, что служила крышей. Боевое возбуждение схлынуло и пришла боль. Терпеть можно, но черт возьми, как же больно. Драная штанина, и рваная рана -- плоды усилий вшивого отродья. Посмотрим, что там. Рана не глубокая, но кровит основательно. 

Надо бы перевязку сообразить. Отрезал кусок от штанов -- все равно рваные. Теперь будут бриджами. Приложил носовой платок к ране, кое как, матерясь сквозь зубы, примотал обрезком штанины. На этом силы меня покидают.

Кажется, сквозь сон слышал внизу какое-то возню и потяфкивание. Псины, должно быть, устроили дележку трофеев.

Утро. Как холодно\ldots Кажется, лежа на бетоне, я отморозил себе спину и бок. К тому же все тело затекло и почти не ощущалось. Ох, как же мне хреново! Да, зато нога хорошо чувствуется -- рана прямо-таки горела огнем. Интересно, здесь где-нибудь ставят уколы от бешенства? Ну да, ну да, мечтать не вредно. 

Спустился с крыши. Хотя, спустился -- это слишком громко сказано. Свалился, худо бедно приземлившись на ноги. Тут же взвыл от боли и сел на задницу. Полежав чуть-чуть на земле, поднялся сначала на четвереньки, а потом и в более подобающее человеку положение. Надо бы осмотреть поле боя. Мда, рюкзак псины изрядно потрепали, но так и не смогли разодрать прочную кордуру окончательно. Зато, судя по характерной влаге и запаху, не постеснялись его пометить. Попросту, обоссали. Фонарь валялся поодаль, так и продолжая работать в энергосберегающем режиме. Собаки, очевидно, нашли его малосъедобным. 

Тьфу мерзость! С брезгливостью осмотрел рюкзак. Ноут пережил даже обоссывание -- хорошая, защищенная техника. Недаром такие любят вояки и всевозможные путешественники. Да вроде ничего особо не пострадало, кроме моей многострадальной ноги. Как же больно! 

Видимо, боль слегка приводит мысли в порядок. У меня же в рюкзаке бинты с зеленкой! С зубовным скрежетом отдираю от раны присохшие тряпки. Надо смазать рану зеленкой, рана сама себя не смажет. Больше, еще больше, а теперь замотать хорошенько.

Закинув на спину рюкзак, я вознамерился двигаться дальше. Но куда? Походив туда-сюда по дороге, нашел свои вчерашние следы. Все-таки хорошо, что здесь повсюду эта белая субстанция -- хоть какой-то от нее прок. Собачьи следы были повсюду, похоже, порезвились бобики здесь на славу. Хоть бы найти какое-нибудь укрытие, прежде чем они опять решат попробовать меня на прочность. Второго такого боя я, пожалуй, не выдержу.

С этими невеселыми мыслями я захромал дальше.
\\
\paragraph{}<++>


Скрипя зубами и матерясь я шагал по разбитой дороге. Кровь, пропитав платок и импровизированную давящую повязку, стекала в ботинок. Рюкзак омерзительно вонял песьей ссаниной, от чего то и дело накатывали приступы тошноты. При этом дико хотелось пить, но воды во фляжке больше не осталось.

Наступив в очередную выбоину, я повалился на асфальт. Неужели конец? Если четвероногие друзья человека сейчас меня догонят, то вряд ли я смогу оказать им хоть какое-то сопротивление. Стать собачьей сытью на неведомой дорожке -- воистину достойный финал. 

Сплюнув, кое-как поднялся. В белесой дымке что-то виднелось. Силуэты каких-то столбов. На грани видимости, но ошибки быть не может -- там что-то есть. 

Строения уже можно разглядеть. Покосившиеся коровники или что-то в этом духе. Неказистые, оплывшие, похоже, что из саманных блоков. 

Подхожу к стене, слегка пинаю ее здоровой ногой. Беззубым провалом зияет вход. Может, это и не коровник? Склад какой-нибудь. Хочется войти, но что-то останавливает. Следы. Здесь кто-то был.  Мне становится дурно. Следы не человеческие. И не собачьи. Узкая стопа, длинные пальчики. Расстояние между следами небольшое, они могли бы принадлежать десятилетнему ребенку. Но это не ребенок. Может быть обезьяна? Какая-нибудь макака-переросток? Цепочка ведет вдаль от дороги, в туман. Идти изучать совершенно не хочется. Прочь, прочь отсюда. 

Бегом! Хороший рывок. Даже боль ушла на задний план. Знать не хочу, что за создания обитают в заброшенных коровниках и с утра уходят на работу в туман. А ведь я тоже оставляю след, если это нечто вернется, пойдет по нему\ldots Ведь чем-то оно питается. 

Забег кончается бесславно. Оступился, нога попала в выбоину, и я растянулся на разбитом асфальте. Тут и рана о себе напомнила. Встать! Сначала на четвереньки, потом выпрямиться\ldots 

Тревожно. Оборачиваюсь. Щелчок. Толстая титановая пластина встала враспорку, надежно зафиксировав клинок. Мне кажется, или там, позади, какое-то движение? Макака вернулась и увидела, что кто-то почтил визитом ее обитель? Гадина. Выпотрошу! Да нет, вроде никого и ничего. Вышел ежик из тумана, вынул ножик из кармана -- как раз про меня. Все еще всматриваясь вдаль, пячусь. Потом разворачиваюсь, и иду быстрым шагом, насколько позволяет больная нога.

Сколько я так иду? Нет сил даже достать телефон и посмотреть время. Воды нет. Туман, как будто, сгущается, превращаясь в, прямо-таки, плотную стену. Не потерять бы дорогу. Продираюсь вперед, периодически наклоняясь, и проверяя под ногами асфальт. Если я его потеряю, страшно представить, куда забреду. 

Тычок в спину, точнее --  в рюкзак. Несильный. Нож из руки я не выпускал. Отмашка на развороте. Достал. Кажется, лишь вскользь, не сильно.

- Уи-и-и-и-и-и-и! -- по ушам бьет омерзительный визг. Отпрыгиваю. Поджилки трясутся, зубы стучат. Отступаю. Что-то явно движется неподалеку, чувствую. Снять фонарь с кармана, несколько тычков в кнопку -- максимальная яркость. Не сказать, что луч вспарывает туман, но лучше, чем ничего.

Какая-то возня рядом. Отпрыгиваю. Тварюка меня явно видит. Или слышит. Или носом чует. Или все вместе взятое. А я ее -- нет. Боковое зрение фиксирует завихрение тумана, смещаюсь, поворачиваюсь\ldots И тут оно прыгнуло. Белесое. С тонкими ручками, что вцепились в мою флиску. А лицо, точнее морда\ldots Запомнил только пасть. Огромную пасть с кучей узких, иглоподобных зубов. Фонарь бьет ударной кромкой куда-то в эту самую отвратную морду, а правая рука работает сама: тычок, еще тычок, тычок. Быстрые, короткие движения. По пальцам течет.

Тварь с визгом отскакивает или откатывается, и на четырех костях скрывается в тумане. 

-- Сдохни, мразь! -- ору ей вслед. И правда, очень хочется верить, что она куда-нибудь уползет и сдохнет от кровотечения, а не залижет раны и нападет вновь. Сколько там ножевых? Десять, может все пятнадцать.

Колени трясутся, зубы стучат. Страшно. Наклоняться и проверять асфальт нет никакого желания. Вперед, в белую муть.
\\
\paragraph{}<++>



Не знаю, сколько я так брел. Периодически, все-таки, останавливался и ощупывал дорогу. Было несколько поворотов, и коли б пошел наобум, неизвестно куда и к кому б забрел. Тяжело. Бой с тощей зубастой тварью не прошел даром: рана под повязкой начала кровоточить.

Бинт пропитался, и мерзкие теплые капли начали затекать в ботинок. До чего же гадкое, ощущение, когда что-то теплое стекает по твоему телу. Понимаешь, что это уходит жизнь. 

Надо остановиться. Скинул рюкзак. Ба! Неспроста был тот тычок в спину. Из рюкзака торчит тоненькая пика, что-то похожее на дротик без оперения. Полметра в длину, почти невесомая. Кость? Уткнулась в лаптоп. Сколов небольшой кусок пластика и обнажив металлический каркас. Вот те на, прям как с тем воякой, коему спас жизнь айпод, в который на излете попала пуля.

Почему-то эта пика не вызывает никакого желания прихватить ее с собой. Даже прикосновение к ней вызывает чувство брезгливости. Хочется бросить эту дрянь подальше, а еще лучше закопать. 

Бинты. Оторвать пропитавшиеся кровью, залить еще той же зеленкой. Проклятье! Как же больно! 

Да что ж такое?! Переживший бой с собаками, а также с неведомой зверушкой, флакончик выскальзывает из пальцев и с приглушенным звоном разбивается о потрескавшийся асфальт.  Остается только замотать рану потуже чистым бинтом и хромать дальше. 

Мне кажется, или туман рассеивается? Как могу, прибавляю шаг. Идти больно, накатывает слабость и жажда. Но нужно двигаться. Да, уже видно не то, что вытянутую руку, но и дорогу. 

Сколько времени прошло? Трудно сказать. В очередной раз останавливаюсь, и достаю из рюкзака телефон. Странно. Время 23:36, но вокруг довольно светло. И как будто, с каждым шагом вперед становится светлее.  

Туман постепенно превращается в легкую дымку, а потом и вовсе совершенно истаивает. Невзирая на то, что телефон показывает 02:14, здесь, судя по всему, раннее утро. 

Позади туман, впереди -- обман. Как пелось в одной из песен моей юности. Если обернуться, взору предстает впечатляющее зрелище: белая стена высотой до небес. И простирается до горизонта. Завораживает и, в тоже время, внушает животный ужас. 

Дорога стала гораздо ровней. Нет жутких трещин и ям. Покрытие, конечно, неровное и далеко не идеальное, но дорога вполне себе качественная. Даже, пожалуй, слишком качественная для нашей-то страны. 

Невзирая на заметно улучшившиеся условия, идти тяжело. Нога болит, усталость валит с ног, а еще жутко хочется пить. Губы потрескались, а язык распух. 

А солнце припекает все сильнее. Снял кофту, скатал ее и затолкал в рюкзак. Эх, как же хочется его бросить прямо здесь и шагать налегке. Но нет. Негоже это, разбрасываться толковыми вещами. 

Куда же я приду? Что, если до ближайшего населенного пункта пара сотен километров? Да я же сдохну тут, по дороге. 

Местность как-то незаметно меняется. Идти все тяжелее, дорога идет в гору. Не сказать, что холмы, но уже и не голая равнина. На приличном удалении от дороги виднеются то ли перелески, то ли посадки. Приходит мысль свернуть с дороги и посидеть в тенечке. Но сколько до них идти?Нет, пройду еще немного по дороге. 

Пот льется градом. Спина под рюкзаком сопрела, дыхание напоминает хрип. Голова кружится. Кажется, лишь на мгновение глаза закрылись, ноги подгибаются и я, как куль, валюсь на обочину. 

Подняться заставляет живое воображение. Оно рисует картину засохшей мумии на обочине дороги, которую клюют мерзкие черно-белые птицы и пробуют на зуб шакалы. С кряхтением и невнятными матерками кое-как принимаю вертикальное положение. Левой! Правой! Левой\ldots

Что это там впереди? Да это же ЛЭП. А еще дальше виднеются какие-то строения. Поднажмем!

И вот, я у цели. Не сказать, что дошел -- скорее дополз. На последнем издыхании. Что я вижу? Побитые ржавчиной ворота со звездой, на которых написано де <<пионерский лагерь>>. <<Совенок>>. По бокам от ворот статуи, видимо, этих самых пионеров. Девочка воздевшая над головой руку и мальчик c трубой. А рядом указатель автобусной остановки. Номер 410. Толкаю ворота. Заперто.

-- Эй! Люди! Кто-нибудь, впустите меня! Впустите, помираю! Эй! -- кричу во все горло, молотя в ворота рукоятью ножа. 

Что это? Кажется, я слышу чьи-то шаги. Кто-то явно бежит по направлению к воротам. Лязг засова, скрип не смазанных, ржавых петель. 

Ворота приоткрываются, и я вижу человеческую фигуру. Девушка. 

-- Эй, ты кто? -- кажется, она говорит что-то еще, но все звуки сливаются в монотонное бормотание.

-- Семен, -- это единственное, что я успел сказать, прежде чем повалился на землю. 

\chapter{День первый}

Я бегу по городу-призраку. Под ногами хрустит белая дрянь, местами ее тут намело огромные сугробы. Белые сугробы, но не из снега. 

Дома, как будто, целые, никаких разрушений нет, но понятно, что город это неживой, нет здесь людей. Нет шума машин, улицы пусты. Автотранспорт есть, но стоит в полном беспорядке. Где-то автомобили явно припаркованы, где-то столкнулись\ldots Но людей нет. Ни живых, ни мертвых.

Гнетущая тишина, только стучащая в ушах кровь, шумное дыхание и топот. Мой топот. Быстрее, еще быстрее. Откуда-то я знаю, что если я замешкаюсь, и не приду вовремя, случится что-то ужасное.  Рывок! Перепрыгнуть через поваленное дерево, взбежать по лестнице, прыгая сразу через три ступеньки. 

Тяжело. Срезать через арку и дальше через дворы. Покинутая детская площадка, пустые скамейки на которых больше никогда не будут сидеть ни старушки, ни загулявшая молодежь. Дома, как будто, с укоризной, глядят темными провалами окон. Пот, пропитав бандану, стекает по лбу. Жарко. 

Дворы кончились. Мимо школы, прямиком по стадиону. Открытое место. Опасно, но надо торопиться\ldots 

Мелькающие улицы, поднимающиеся облака белой пыли из-под ног. Туман то сгущается в непроглядное марево, то превращается в легкую дымку. 

Еще немного\ldots Хлопок! Что-то бьет в левую ногу. Падаю, удар о землю выбивает из легких воздух. Откатываюсь, пытаюсь подгрести ногами. Откуда прилетело? Укрыться, быстрей укрыться! Слышны еще хлопки. Накатывает боль и сознание гаснет.
\\
\paragraph{}<++>
 

-- Не дергайся! Паршиво повязку наложил, теперь еще с кушетки удрать пытается\ldots -- слышу приятный женский голос. Открываю глаза и вижу склонившуюся надо мной фигуру в белом халатике. Колдует над ногой. 

-- А-э-э-э, ступня на месте? -- мда, голосу моему твердости явно недостает.

-- Да на месте, на месте. Кто это тебя так?

-- Стрельнули.  

-- Что?! Вот фантазер! Да лежи ты спокойно! Обгорел на солнце, тепловой удар заработал. А рана рваная, кто подрал?

-- А\ldots Ну да, кажется, собаки дикие. Я вообще где? -- действительно, ведь я вроде куда-то брел через туман, разбившийся автобус, заброшенные коровники.. 

-- Да в медпункте ты, в медпункте. В пионер-лагере <<Совенок>>. То подстрелили, то дикие собаки. Сроду тут ничего такого не было. Семен, значит тебя зовут?

-- Ну да. А тебя?

-- Не тебя, а вас. Виола я. Вот, вот так. -- судя по ощущениям она затягивает повязку. -- Полежи пока, а я за Ольгой Дмитриевной схожу. 

-- Попить есть что-нибудь? 

Медсестра встает, берет со стола кружку, споласкивает в раковине и наливает воды из литровой банки. А в банке-то, похоже, кипятильник.

-- Держи.

Да, а халатик сидит ладно, в обтяжку. Да и сама дама ладная, ух. Мде, что-то я не о том думаю. Или о том?

Хлопает дверь и медсестра удаляется. 

Интересный сон. Что это был за город? Вроде, я в нем никогда и не бывал, но почему-то он казался знакомым. Какого черта? Чьи это воспоминания? Ладно. Это сейчас не главное.

Надо бы осмотреться. Кабинет, ну, вроде бы, обычный. Шкаф со склянками, шторка над кушеткой. Штуковина для измерения роста, казенный стул с дерматиновой сидушкой. Письменный стол. Ух ты, прямо у меня над головой стоит выключенный компьютер. Какая древность! Неужто ДВК? У нас такие были на <<станции юных техников>> поначалу. И ведь эта штука не выглядит старой. Все чудесатее и чудесатее. 

Весы, вторая койка. Ну да, медпункт. Был бы обычным, если не считать этот музейный экспонат на столе. Комп притягивает взгляд. Сам не знаю почему, но начинаю напевать старую песню, слышанную когда-то на сисопке.
\begin{verse}
Мы были молоды и не страшились преград,\\
Где не спасет перезапуск, поможет format,\\
А если не было копий, мы тактику брали иную -\\
По дискетам мы ползали, и по частям\\
Собирали останки погибших программ\\
И шестнадцатиричные dump'ы вводили вручную.\\
\bigskip
Мы привычно плевали на любой Copyright,\\
Нам казались простором даже 100 килобайт,\\
Мы учили ассемблер, не знавший команд умноженья.\\
Распечатки не резали мы на листы,\\
И наши первые вирусы были просты,\\
Но мы все-таки были в восторге от их размноженья.\\
\bigskip
Мы не боялись тогда -- мы были много смелей -\\
Ни плохих секторов, ни магнитных полей,\\
И даже сбой по питанию не был источником страха.\\
Нам было все трын-трава, нам было просто совсем\\
Одним нажатием на кнопку повесить СМ,\\
Нам служил ДВК, и нам повиновалась Yamaha\ldots\footnote{Стихи Ю. Нестеренко}
\end{verse}

-- Ух ты! Да ты поэт! А что такое копирайт? А что такое ассембрер? -- раздается тонкий девичий голосок. Из-за приоткрытой двери виднеется озорное девчачье личико. А вот и вся девчина целиком. Лет четырнадцать, загорелая, волосы даже не рыжие, а скорей красные. Красная же футболка с надписью СССР и белые шортики.

-- Уп. Ты кто? 

-- Ульяна, Ульяна, а ты Семен? Про тебя весь лагерь говорит, что пришел весь оборванный и руки по локоть в кровище!  

Ну да, когда швейной машинкой поработал на сумеречной твари, должно быть, хорошо перемазался. Смотрю на свои руки. Да нет, чистые, если не считать того, что под ногтями. 

-- Ага. А еще у меня вот такие зубы. И зрачки вертикальные. Бу! 

-- Ой, ладно, мне пора, -- говорит девчонка и резво скрывается за дверью.

Снаружи слышатся женские голоса и приближающиеся шаги.

-- А вообще, надо бы в милицию позвонить, мало ли что\ldots 

В медпункт входит та самая ладная медсестричка, и следом за нею девушка лет двадцати пяти. Ее фигурой природа также не обделила.

-- А вот и наш раненный в ногу рысь, -- объявляет Виола. 

-- Меня Ольга Дмитриевна зовут, я вожатая. Рассказывай. -- сразу берет быка за рога вторая девушка.

Выкладывать все, как на духу, если честно, не хочется. Да и сам я мало что понимаю. Как-то все перепуталось. Мегаполис, лес и болота, заброшенный город, снайпер, путь в тумане, собаки, зубастая тварь с дротиком\ldots Смешались в кучу кони, люди.

-- Да я плохо помню. Очнулся в разбившемся автобусе. Икарус, вроде. Ехал куда-то, значит.

-- А другие ребята?! К нам как раз должны были пионеры приехать, что с ними? Где авария случилась?

-- Да не было никого. Пустой автобус. Пустое багажное отделение. Следов тоже не видел\ldots 

Рассказываю урезанную версию про путешествие через туман, про стычку с собаками. Встречу с зубастиком предпочитаю опустить. 

На лицах девушек читается удивление. 

-- Нет, точно нужно в милицию звонить, пусть ищут, -- говорит Ольга. -- Один в пустом автобусе?!

-- А мне что тут делать? -- спрашиваю.

-- Как что? Лечись, отдыхай. Ты же отдыхать сюда ехал? 

-- Да я не помню, на самом деле\ldots 

-- Да, видимо, головушку тебе крепко напекло. А может и сотрясение при аварии. Сознание не терял? Не тошнило? -- задает серию вопросов Виола.

Расспросы о моем физическом и душевном состоянии. О необходимости звонить в милицию и непременно искать пропавших. Каких, к черту пропавших, не было там никого. И не в лагерь я ехал, староват я для этого. Пионеры, новенький ДВК\ldots Однако.

-- Семен, идти можешь? -- спрашивает вожатая.

Встаю. Вроде нормально. Видимо, квалифицированная медпомощь способна творить чудеса. 

-- Да, вполне.

-- Пойдем, я тебе одежду выдам и жилье тебе подыщем. 

-- Кстати, где мой рюкзак? 

-- Да вон он, под столом же. -- отвечает медсестричка. Только сейчас обращаю внимание на ее глаза. Один карий, другой голубой.

Подхватываю рюкзак (воняет он вроде поменьше) из-под стола, накидываю на одно плечо лямку и следую за Ольгой. Идти почему-то неудобно, периодически запинаюсь на ровном месте и шаркаю. Ботинки, что ли стали велики?

Входим в один из домиков, судя по всему, женское жилище. 

-- Ну вот и пришли. Надо бы тебе форму подобрать, а то выглядишь как пират двадцатого века.


-- Скорей уж двадцать первого, хе-хе, -- слетает с языка непрошеная фраза.


-- Ладно, вот рубашка, шорты должны подойти, вроде твоего размера, где-то тут должны быть сандалии, -- бурчит вожатая, роясь в шкафу. -- Вот! 

Сгребаю весь этот гардероб в охапку. Видимо, мне одному кажется, что бородатый пивонер-переросток -- жуткая нелепица.

Ольга что-то там говорит, но особо не вслушиваюсь\ldots 


-- Эй! Заселяться идешь, или тут жить собрался?


-- А, ну да, ну да.
\paragraph{}<++>




-- Витька, опять курил, подлец? -- Ольга возмущенно машет кулаком перед носом рыжего и конопатого паренька с мелкими чертами лица. Урожай на рыжих здесь явно удался. -- А ну давай сюда сигареты! Сколько ж ты их сюда притащил?! -- продолжает злобиться вожатая. 

-- Да не, это старый запах не выветрился, не курил я! Нету ничего больше\ldots

В общем, поселили меня с этим незадачливым курильщиком. Прочтя лекцию о неподобающем поведении, вреде курения, и подобном, Ольга удалилась, дав напоследок наказ подружиться. 

Болтать что-то нет желания. Устал. Стягиваю ботинки, и по комнате распространяется неповторимый, устойчивый аромат. Да, бактерицидная пропитка. Да, дорогие треккинговые носки. Но если идти несколько дней подряд, то никуда от характерного амбре не денешься. Рюкзак занимает место под кроватью. Мелькает мысль, что надо бы электронику выложить, плеер зарядить и радио послушать. Ладно, потом. Все потом. 

Витек что-то спрашивает, но мне плевать. Поворачиваюсь на бок к стенке и отключаюсь.

Просыпаюсь от чувства голода. Есть хочется жутко. Немудрено, не жрамши минимум пару дней. 

Сосед куда-то утек, ну и ладно. Пойду на разведку. В комнате умывальника нет, посему натягиваю чистые носки на грязные ноги. Зашнуриваю ботинки, вперед, навстречу приключениям, а главное -- еде. 

Привычно хлопаю по карману. Опа, а ножа-то нет. Неужто рыжий-конопатый спер? Или я  выронил, когда у ворот отрубился? Вот те задачка. Не комфортно я себя чувствую без оружия. Пусть даже такого примитивного. Мультик, впрочем, по-прежнему висит в чехле на поясе. Ладно, сойдет на безрыбье-то. Вешаю мультитул клипсой на карман и выхожу на улицу. Запереть дверь, выданным вожатой ключом\ldots

Мысль об добротном пыряльнике, к тому же, прошедшем испытание боем, перекрывает даже чувство голода. А вон, и ворота виднеются. Прямо по улице. Вперед!

Темнеет. Подсвечивая фонарем, осматриваю окрестности в поисках утерянного. Нет, не видать. Надо бы и за ворота заглянуть, вроде же там отключился. Осмотр земли и асфальтовой площадки ничего не дал. 

Гляжу вдаль. Туманная стена никуда не делась. Точнее, сейчас она напоминает эдакий сгусток мрака, за которым не видно звезд. Что же это такое? Приятных чувств такое зрелище не вызывает, более того, гнетет. Сразу вспоминается зубастик с его пикой. А сколько таких тварей там водится? И ведь ориентировалось оно в тумане неплохо. Что помешает подобной гадине добраться до этого райского местечка? Ответов нет, одни вопросы.

Продолжаю поиски. 


-- Семен! -- обернувшись, я вижу девушку, открывшую мне ворота.

-- О, привет. Спасибо за спасение!

-- Да что ты\ldots -- кажется, она слегка смутилась. 

-- А тебя-то как зовут? -- Спрашиваю, рассматривая пионерку. То, что я вижу, мне определенно нравится: открытое лицо с большими голубыми глазами, густые русые волосы, заплетенные в длинную косу, ладная фигурка.

-- Славя. -- Отвечает девушка. -- Вообще, полное имя Славяна, но все меня Славей зовут. И ты тоже зови. 

-- Ну вот, и познакомились, -- произношу, подойдя к ней поближе. -- Кстати, ты тут ничего не находила?

-- Ой! Ну да, тот жуткий нож, которым ты в ворота колотил. Я его к себе прибрала, все-таки холодное оружие\ldots

-- Да ладно, -- протянул я, пожав плечами, --  обычный складник.

-- Ну ты его лучше Ольге Дмитриевне не показывай. Пойдем, верну находку.

Так, не спеша, мы идем в здешний <<спальный район>>. Проходим площадь с памятником. 


-- Это, наверное, Ленин? -- в потемках-то видно плохо, да и кому еще может быть памятник в пионерлагере?

-- Ты что, это же Генда!

-- Кто?! 

-- Ну как так, ты чего, это ж любой знает с колыбели.

-- А, ну да, ну да\ldots

-- Кстати, ты ведь не ужинал?

-- Угу. А еще не обедал и не завтракал.

-- А вот, столовая. Нож твой никуда не убежит. -- Девушка, погремев связкой ключей, выбрала нужный и отперла замок.

Ужин был нехитрым. Видать, голодные пионеры все умяли. Несколько булочек, стакан кефира. Ненавижу молочное, но голод не тетка. Пока ем, Славя сидит напротив и смотрит на меня.


-- Что?

-- Ты, ну, с таким аппетитом ешь\ldots

-- Ну дык, пару дней не жрамши.

Вкратце рассказываю ей свою историю, начиная с пробуждения в автобусе.


-- Странно. Этот туман тут пару дней назад появился. Должны были еще ребята приехать, но появился один ты. 

-- Меня беспокоит, что в этом тумане водится. Собаки вот эти.

-- Ну, ты уставший был, лежал. А так на людей они не должны нападать. 

-- Хе. Кстати, до ментов-то дозвонились?

-- Ольга Дмитриевна пыталась, но что-то телефон не работает.

-- Занятно. 

Ну вот, не сказать, что набил живот, но хоть что-то. Быстренько прибрали за собой посуду, закрыли столовую и двинулись к домику Слави. 


-- Подожди здесь, соседка спит уже.

Подожду, конечно подожду. Ночь, на первый взгляд, безмятежна. Яркие звезды, ни одного облачка.  Приветливо светит месяц. Воздух свеж и приятен, в траве стрекочут насекомые\ldots Но стена тумана никуда не пропала, и это пугает. 


-- Вот, держи. -- девушка тихо выскользнув из своего домика, протягивает мое оружие. 

Нож не сложен. Полосатый исцарапанный клинок покрыт бурыми пятнами. Мде. Представляю, как это все выглядело со стороны. С негромким шелестом клинок складывается в рукоять и нож занимает место под ремнем.


-- Спасибо. Спокойной ночи.

-- И тебе тоже. Не проспи завтрашнюю линейку.

Спать, однако не хочется. Где-то тут я видел то ли душевую, то ли умывальники -- надо бы ополоснуться. Улицы освещены фонарями, но пионеров не видно. Никто не слоняется туда-сюда, не  орет песни, не мажет маркером стены\ldots Как это не похоже на тот лагерь, куда я ездил в детстве. Правда, тот был не пионерский, пионеры кончились вместе со страной, в которой я родился. 

Конец девяностых, путевки от градообразующего, которое наконец-то восстало из могилы. Дети инженеров и работяг. <<А ты чего слушаешь: металл или рэп?>>. Пробежки за куревом и пивом в расположенный неподалеку ларек\ldots Купание под присмотром взрослых? Ну уж нет, постоянно убегали подальше. А километрах в десяти, выше по течению, был железнодорожный мост, и с него можно было весело попрыгать в воду. Вечером -- дискотека и посиделки на лавочках. Вырезанные на любых деревянных поверхностях названия любимых групп, подрыв сортира бомбой на дымном порохе, упертым из отцовского сейфа прямо перед выездом\ldots 

Здесь же все прямо-таки образцово-показательно. И какого черта меня принимают за пионера? 

Ну вот, дошел. Умывальные колонки, шланг, горячей воды-то, конечно же, нет. Снимаю ботинки и носки\ldots Так. Вроде никого поблизости, можно и ополоснуться по-человечески. Жаль, бельишко из рюкзака захватить не додумался. Раздеваюсь до трусов и поливаю себя из шланга. Да, тепло, трусы и на мне высохнут\ldots

Внимание привлекает шуршание в кустах. 


-- Эй, кто там? -- шаг в сторону лежащей на земле одежды. -- Лучше покажись по-хорошему!

Тихонько стягиваю со штанов свой пыряльник-ковыряльник, и тихо, без щелчка, раскладываю, заведя руку за спину.  

Не люблю ходить босиком. Или в открытой, непрочной обуви. Но натягивать и зашнуривать ботинки нет времени. В кустах никого. Следопыт-то из меня не ахти, точнее -- почти никакой. Но вижу, что трава примята. Становится тревожно. С водными процедурами определенно пора заканчивать.  

Дверь в моем новом жилище не заперта. Сосед объявился, да завалился спать. Ну и ладно, пора и мне на боковую. 

\chapter{День второй}
Темнота. Темнота и боль. В ноге, да в спине, за левой лопаткой. Боль в спине тупая, как от удара молотком. Как же больно!

Что-то грубо пихает в бок. Еще пару раз. 

-- Что тут у нас? Сдох, сдох, точно сдох, хорошо.

Еще тычок. Кто-то грубо переворачивает меня на спину и грубо обшаривает, пытаясь что-то содрать. В нос бьет запах немытого тела и гнилых зубов. 

-- Ну давай, отстегивайся, -- некто сильно дергает.

Правая рука, как будто сама, ложится на прорезиненную рукоятку ножа. Левая хватает немытого за одежду, и дергает на себя. Короткий скрип стали о пластик, и клинок  входит в тело врага. Еще удар, еще -- как швейная машинка. Свалить наземь, подмять под себя, все нанося и нанося удары. Враг хрипит -- похоже, пробиты легкие. Кромсать, разорвать на куски! По рукам течет теплое и липкое.

Кто-то бешено трясет меня за плечо. Тычок локтем в сторону угрозы, откатиться\ldots
\\
\paragraph{}<++>


-- Ай, бля!!! У-у-у, больно-то как! -- у кровати, зажимая нос, из которого течет кровавая юшка, стоит и скулит Витек. А перед моими глазами все еще мостовая, покрытая белой пакостью, в же ушах -- хрип и хлюпанье. Взгляд на руку. Нет, она не по локоть в крови, да и нож в ней отсутствует.

-- Бля. Ты что, совсем мудак, так будить? За ногу подергай, если получать не нравится. Ну, дай посмотреть.

Сосед ругается, совершенно неподобающе пионеру, но руку от пострадавшей части лица убирает. Да вроде б не сломал, просто кровит. 

-- Платком зажми, все сейчас здесь уделаешь. 

-- Козел! Я ж тебя разбудить\ldots На линейку! А ты! У-у-у! -- гнусавит рыжий.

Футболку с штанами я не снимал, прямо так с ночи завалился. Натягиваю ботинки. Что-то великоваты стали. 

-- За битого двух небитых дают. Пошли.

Утро прекрасно. Жары еще нет, на голубом небе пушистые белые облачка, кругом зелень, чирикают птички, в траве стрекочут насекомые\ldots Но стоит повернуть голову в сторону трассы, и взору открывается мрачная серая стена. Кажется, она стала еще ближе.

Не успеваем отойти от домика, как навстречу нам выбегает вожатая. Расположение духа у нее, надо сказать, не самое радужное.

-- Вот вы где? Живо на линейку! Все уже собрались. Особое приглашение нужно? Ой! Витя, что у тебя с носом?

-- Упал. 

-- Семен! Что у вас случилось? -- Ольга смотрит на меня с явным подозрением.

-- Да ничего. В медпункт идем.

-- Живо, живо! А я за Виолой!

Путь-дорога лежит через площадь. Похоже, тут собрался весь лагерь. Картина вполне обыденная, если не брать во внимание пионерскую форму. Ребята разбились на группки и общаются. Кто-то сидит на скамейках, кто-то прямо на газоне у памятника\ldots Детвора из младших отрядов с веселыми криками носится туда и сюда\ldots Такую картину можно увидеть у любой школы. Не хватает только курильщиков, да лающего смеха четких поцанчиков. Что-то их тут, в лагере, совсем не видно.

Пришли. Вот он -- медпункт. А вон и Ольга с Виолой спешат со стороны <<спального района>>. Точнее, спешит вожатая, а вот медсестра явно не прочь бы еще поспать.

-- Гм. Ну и что тут у вас, мальчики, случилось? Неужто подрались? -- мурлычет фигуристая служительница фонендоскопа и клизмы.

-- Так, Семен. А ты марш умываться, переодеваться и на линейку! Опять в рванье своем и ботинках этих! -- Ольга явно недовольна моим внешним видом. -- Да ты посмотри на себя в зеркало, разве так должен выглядеть пионер?

Краем глаза фиксирую свое отражение. Что-то не то. Поворачиваюсь и смотрю внимательно. 

-- Какого? -- однозначно, не то. Из зеркала на меня смотрит совершенно другой человек. Первое, что бросается в глаза -- бороды нет. Растительность на лице отсутствует. А вместо короткой стрижки -- довольно густая шевелюра, какой я щеголял в школьные годы.

Всматриваюсь в отражение внимательнее и понимаю, что это все-таки я. Только заметно помолодевший. Ощупываю языком ротовую полость. Удивительно, но пара потерянных, в одной из драк, зубов на месте. Ростом стал поменьше, поуже в плечах, да и полегчал минимум на треть. Как же я этого всего сразу не заметил? Потому и ботинки великоваты, и футболка. Сколько мне здесь? Лет шестнадцать-семнадцать, не больше.

-- Ты долго себя разглядывать будешь? Марш умываться! -- прерывает созерцания и размышления голос вожатой.

В немом удивлении я покидаю медпункт и оправляюсь к своему жилищу, на ходу ощупывая себя руками. Просто поразительно!

По дороге к нашему с Витьком домику, не удержавшись, продолжаю ощупывать свое лицо руками. Кажется, собравшиеся на линейку пивонеры, на меня удивленно косятся. Ну и ладно, ну их.

А вот и наше скромное жилище. Так, где-то здесь точно было зеркало. Стягиваю футболку и штаны, осматриваюсь. Странно, пропали и те шрамы, которые я получил до шестнадцатилетия. Вот, к примеру, ободранная при падении на мотоцикле, бочина всю дальнейшую жизнь немножко отличалась цветом. Шов от удаления аппендикса, как это ни странно, на месте. Ладно, чего с этим всем поделать? Надо бы, и правда, привести себя в порядок. 

Пакетик с мыльно-рыльными принадлежностями обнаружился на тумбочке. Ага. Зубная щетка, кусок мыла, и какая-то круглая баночка. Неужели зубной порошок? Хотя, чего я ожидал? Да и после внезапной смены тела, меня, пожалуй, уже ничем не удивишь. 

Взгляд зацепился за висящую на стуле пионерскую форму. Штанцы-то мои, мало того, что на ладан дышат, так еще и велики. Футболка грязная, разве что, колом не стоит, штаны под стать, да еще и драные. Ладно, почему бы и нет? Слиться с толпой и не отсвечивать. Ремень из комплекта униформы не порадовал: слишком гибкий. Нет, свой, добротный, усиленный пластиком по всей длине я на эту дрянь не променяю. 

Чехол с мультитулом на пояс, фонарь на левый карман, нож на правый. Рубашку навыпуск. А с этой красной фигней что делать? Ну да, галстук. Повяжу-ка я его на голову. Чтоб, доставшиеся с новым телом волосы, в глаза не лезли. Да и пот, если что, впитает. Как бандану не наденешь, маловат, а вот если свернуть несколько раз и повязать вокруг лба -- сойдет. 
% узнать размеры пивонерского галстука

В <<душевой>> было безлюдно. И немудрено -- все собрались на площади. Вода по-прежнему ледяная, а порошок непривычный. 

-- Привет, новенький! -- к умывальникам направлялась пионерка. Рыжая. Высокая и стройная, ну просто загляденье.

-- Ну, привет, коли не шутишь. Форму, как погляжу, тоже недолюбливаешь? -- Рубашка на девушке была повязана на манер топика под грудью, а несчастный галстук обернут вокруг запястья.

-- Дык: <<безобразно, зато единообразно>>, тьфу. Это не тебя там, кстати, вожатая ищет, бегая с воплями по всему лагерю? Поспать людям не дает, -- произнесла пионерка  и потянулась. 

-- Может и меня. Умывайся что ли, да пойдем, посмотрим, чего там на площади происходит. 

-- Хе. А с чего мне с тобой куда-то идти?

-- Не хочешь, как хочешь, колхоз -- дело добровольное.

-- Ладно, только чтоб увидеть как вожатую перекосит.

\paragraph{}<++>

-- Сычев, Двачевская, вы как одеты? Нет, вы надо мной издеваетесь, заявившись на линейку в таком виде? Быстро привели себя в порядок! -- Разорялась Ольга.

-- А чего не так-то? Хоть какое-то разнообразие, -- с довольным видом произнесла рыжая.

-- Я тоже за творческий подход, -- поддерживаю.

-- Нет, ну вы у меня оба получите на орехи. Семен, сними галстук со лба, Алиса, рубашку расправь. Когда только спеться успели?

-- Алиса, значит?

-- Угу. Тебя-то уже пол-лагеря знает.

Вожатая не унималась:

-- Сними галстук с головы, я тебе сказала, и выйди, представься перед строем.

-- Ну ладно, -- делаю несколько шагов вперед, и оборачиваюсь к выстроившимся пионерам. -- Здравствуйте. Меня зовут Семен и я алкоголик. -- Слышатся одинокие смешки.

Ольга, прямо-таки побледнела и изменилась в лице. 

-- Да что с тобой такое? Что ты творишь?!

-- Ладно, на самом деле я не алкоголик, просто ничего не помню, посему рассказать особо нечего.

Вожатая глядит, прямо-таки волком. 

-- Вот. Обходной лист. Принесешь, как подпишешь.

-- Обязательно. Без бумажки ты какашка, а с бумажкой человек.

Линейка подошла к концу и пионеры начали разбредаться кто куда. Взгляды, которые они на меня бросали, были, по большей части, неодобрительные. Странно, а в том, другом лагере, другой жизни, подобные выходки пользовались немалым успехом. Другие времена, другие нравы\ldots

Рыжая, однако, никуда не уходила, и с подозрением на меня поглядывала. 

-- Да уж, вот бы ее физию сфотографировать, когда ты про алкоголика сказал! Но все равно, как-то жестко получилось.

-- Семен! -- послышался знакомый голос. Славя приближалась с явно недобрыми намерениями. Щеки красные, кулачки сжаты. -- Ты чего устроил, с ума, что ли сошел? Ты зачем так над Ольгой Дмитриевной издеваешься? Алиса, это ты его подговорила?!

-- А я чо, а я ничо\ldots 

-- От казарменного духа воротит -- раз. И все эти красные галстуки, линейки, гимны и присяги. А как вспомнишь, чем это все закончилось -- тошнит. Это два.

Кажется, эта тирада вызвала, мягко говоря, недоумение. Славя в момент утратила рассерженный вид, приобретя обеспокоенный, Алиса тоже выглядела озадаченной.

-- Славя, Семен, я  пойду, пожалуй\ldots 

-- Что закончилось? Ты ведь говорил, что ничего не помнишь. Ведешь себя странно. И появился странно. И одежда у тебя странная. И нож твой жуткий. А еще Вите нос разбил. Ты в милиции на учете не состоишь?

-- Вот уж чего не помню, того не помню.

-- Как удобно: того не помню, этого не помню. Откуда ты такой здесь появился?

-- Хотел бы я знать\ldots 

Девушка фыркнула и удалилась.
\\
\paragraph{}<++>


Завтракал я в одиночестве. Большинство пионеров уже управились и отбыли по своим пионерскими делам, а оставшиеся, как мне казалось, косились и вели себя как-то настороженно. Ну и ладно. 

Поднос с незамысловатой пищей: тарелка  каши, стакан компота из сухофруктов, вареное яйцо. Дальний столик, место спиной к стене. 

На душе тревожно. Новое тело, туман и сумрачные твари, этот подозрительно образцовый лагерь. Провал в прошлое или параллельную реальность? А тело? Ладно, допустим -- перенос сознания. А вещи? Одежда, снаряжение. Тоже не сходится. Обрывки чужих воспоминаний тоже не идут из головы. Какая-нибудь религиозная чушь, вроде чистилища? 

Ладно. Надо бы здесь осмотреться. Кстати, а что-там по поводу той смой бумажки? Ага, <<обходной лист>>. Кружок кибернетиков, музыкальный кружок, медпункт, библиотека. А ведь, действительно, может удастся что-то понять: в той же библиотеке должна быть недавняя периодика. 

Завтрак закончен, пора бы и осмотреться. Вот туман этот, будто одного меня беспокоит. А ведь, кажется, стена придвинулась еще ближе.

-- Привет, поэт-алкоголик! А ты матерные стишки знаешь? А Алиска на гитаре умеет играть! Матерные частушки! Ольга Дмитриевна позеленеет! А Петра Иваныча вообще б кондрашка хватила.

-- Ульянка, а что за Петр Иванович? 

-- Так директор лагеря. Он сразу после твоего появления в туман пошел, автобус искать, наверно. 

-- Стоп. И что, его до сих пор никто не хватился? И про автобус откуда узнала?

-- Так земля слухами полнится, -- ухмыльнулась девчушка, -- а чего его хвататься? Он же здесь самый взрослый.

-- Зашибись, логика. -- вспомнились и собачки, и тварь, и то, что за моим вечерним умыванием наблюдало\ldots -- Дерьмище!

-- Эй, стой, ты куда? Меня подожди! Ты мне матерных частушек не спел!
\\
\paragraph{}<++>

На стук никто не откликался. Видимо, вожатая куда-то отправилась по своим очень важным делам. Может в медпункте сидит с Виолой треплется?

-- Ульянка, Дмитриевну не видела? 

-- Неа! Она скучная. И вообще, пойду в футбол играть.

-- Иди-иди, угу.

Бегом к медпункту. Дверь не заперта. Виола сидела за столом и что-то чиркала в тетради. Моего появления она явно не ждала и, прямо-таки, подпрыгнула, после чего судорожно захлопнула тетрадь, и затолкала ее в недра стола.

-- А, Семен. Опять кому-то нос разбил? -- медсестра поднялась со стула и сделала пару шагов в мою сторону.

-- Нет. Ольгу Дмитриевну не видели?

-- Да на пляже она, наверное. А что случилось? -- немного удивления в голосе, как бы невзначай поправляет на груди халатик. Что-то я тут все на женские прелести заглядываюсь, вместо того, чтобы хоть в чем-то разобраться. А хороша, да. Кхм.

-- Вы что, здесь все совсем не пуганные? Человек почти сутки назад ушел в странный туман, от него ни слуха, ни духа, а у вас тут пляж и дебильные линейки! -- получилось как-то слишком эмоционально и даже истерично. 

Взгляд медсестры, как будто, принял осмысленное выражение, а на лице появилась тревога. 

-- А что с ним может случиться-то? Да и не ушел, а уехал с дедом Кузьмичом на запорожце. Хотя, действительно, что-то они долго. 

-- Кузьмичом? -- воображение живенько нарисовало деда с клочковатой седой бородой, сидящего за столом и нарезающего пыжи из старого валенка.

-- Так сторож наш. Он же лодочник и плотник. 

-- Понятно. До милиции-то дозвонились?

-- Ну телефон не работает же. -- Виола сделала еще пару шагов в мою сторону. Уф, а ведь она почти на пол-головы меня выше. 

-- Действительно. Ладно, пойду. -- поворачиваюсь, уже собираясь сделать шаг по направлению к двери, как на плечо ложится женская рука.

-- Стой, пионер. -- женщина взяла меня под руку и потянула в сторону кушетки. -- Ты присядь, давай я тебя посмотрю, раз уж зашел. Нога, я смотрю, уже не беспокоит?

-- Да и забыл уже, -- отвечаю, плюхаясь на кушетку. -- А вы тут странного ничего не заметили, когда меня осматривали?

-- Например? -- на лице легкое недоумение, в левой руке очки. Покусывает дужку. Правая рука уперта в округлое бедро.

-- Ну откуда я знаю, я ж не медик, -- черт, кажется, от близости этой дамочки я начинаю потеть. Как школьник, тьфу.

-- Да нет, слегка покусанный худосочный пионер, ничего такого. -- изящная кисть ложится на мой вспотевший лоб, медсестра слегка наклоняется и  ее объемный бюст оказывается прямо у меня перед носом. --  Ой, да никак у тебя жар?
%Первый укол от бешенства ? 

-- Хе-хе, ну эта, -- сдвигаюсь по лежанке, и пытаюсь протиснуться мимо Виолы, -- я пойду-таки. -- Пара широких шагов, и вот спасительная дверь.

-- Ну, ты заходи, если что, -- слышится вслед.
\\
\paragraph{}<++>

Да уж, зашел в медпункт. А ведь я эту медсестричку хочу, и она это, судя по всему, прекрасно видит. Интересно, это у нее со всеми так? Страшно представить, что приключилось с Витьком. 

Ладно, надо бы пойти и все-таки отыскать вожатую. Не хватало, чтобы еще кто-нибудь отправился на встречу приключениям. Да и вооружиться посерьезнее не помешает. Надо бы в Кузьмичовье логово наведаться и что-нибудь полезное оттуда упереть.

Однако, на героических свершениях поставила крест банальная резь в животе. Видно, питание кефирчиком и кашами не идет на пользу молодому растущему организму. Или это нервное потрясение от встречи с озабоченной медичкой? Где-то я тут неподалеку удобства видел. Обошел медпункт и вуаля, вот он. Туалет типа <<сортир>>: эм и жо. Некрупное кирпичное строение с эдакими коридорчиками-предбанниками( то есть предсортирниками) с рукомойниками. Удивительно, но и здесь стены не были осквернены похабными надписями и рисунками. Чего уж, даже крылатого слова из трех букв не видно. 

Пошарившись в округе, бумаги так и не нашел. Просто замечательно. Зачем пионерам такие удобства, как горячая вода и туалетная бумага? Это все пережитки империализма. Как кстати мне Ольга обходной лист вручила. Ну и лопуха, или что это тут растет,  еще можно нарвать\ldots Уф, ну все, теперь помыть руки и вперед. Нас ждут великие дела!

По площади бродили пионеры. Парочками или небольшими группками. Вон, кажется, куда-то спешит Славя. На скамейке сидит девушка с неестественным цветом волос и читает книжку. Слышен детский смех, разговоры -- ну просто идиллия кругом. Группка ребят с полотенцами в руках, очевидно, направляются в сторону пляжа.

С одной стороны, надо бы все-таки вожатую порасспросить на счет этой горе-поисковой команды, с другой, что тут еще можно сделать? Может и вернутся: я же сюда дошел. Да и свое снаряжение надобно в порядок привести. Рюкзак постирать и высушить, лаптоп осмотреть и подзарядить, да эфир послушать не помешает. Решено, занимаюсь сейчас своими делами. 

Мое скромное жилище было не заперто. Как оказалось, сосед только пришел с купания и переодевался. 

-- Ну, как нос? -- я непринужденно осведомился о состоянии его здоровья.

-- Да ничего. -- донеслось в ответ. Натянув шорты, Витек обернулся и взглянул на меня с некоторой опаской. Я же сгреб со своей кровати грязную одежду и бросил ее на пол. 

-- Как там, тебя Виола не снасильничала? -- С ухмылкой спросил я, вытягивая из-под кровати все еще воняющий рюкзак и обернулся. 

-- Не, вату в нос вставила, да отпустила. А чего? -- Мне кажется, или сосед все-таки немного покраснел? Хотя, черт их знает таких рыжих. -- А ты чего, подписи собрал уже? 

-- Хе. Тебя ж на линейке, вроде бы, не было, -- я с остервенением ковырял заевшую молнию.

-- Да не, я потом подошел, видел твое выступление. Хорошо ты вожатку выбесил. Кстати, ты куришь? -- Рыжий покопался под матрасом и вытащил оттуда помятую белую пачку. 

-- Не, курить -- здоровью вредить. -- Молния никак не поддавалась, и пришлось подковырнуть ее лезвием мультитула. Ага, пошла!

-- Кто не курит и не пьет, тот здоровеньким помрет, -- парировал рыжий. -- О, клевый ножик. Импортный, что ли? -- Блестящая штуковина вызвала явный интерес. -- Можно?

-- На, не порежься. -- я катнул инструмент по полу.

-- <<Кожаный человек заряд>>. Фига себе, точно импортный. -- Сколько удивления-то, ну да, ну да: при Союзе и <<белочка>> для школьника была сокровищем. Хотя, делались и весьма недурственные складники. Парень, тем временем, шустро открывал и закрывал многочисленные предметы. -- Эх, а штопора-то нет. -- И, переходя на шепот, -- контрабандный поди?

-- Да ну, какой там, отец из Болгарии привез. -- Молния, наконец, окончательно расстегнулась и из рюкзака появился портативный компьютер. -- А пробки нормально основным лезвием выковыриваются.

-- А это что? -- Черный обрезиненный прямоугольник, с проглядывающим на уголках металлом, вызвал не меньше интереса. 

-- Да приемник это такой, приемник. Где его, кстати, запитать можно? А то я тут ни одной розетки не нашел. -- Я еще раз окинул комнату взглядом, но розеток так и не обнаружил.

-- А, ну у кибернетиков, например. А еще, знаешь что, давай лампочку выкрутим и напрямую подключим! -- Проявил недюжинную смекалку пионер. -- Ух и клево с музыкой будет! 

-- Да, голь на выдумки хитра. Кибернетики, это у входа что ли? Надо зайти к ним, посмотреть, чего там интересного. -- Сразу же вспомнилась <<станция юных техников>>, где много лет назад, можно сказать, что в другой жизни, я так любил пропадать, дни напролет ковыряясь с древними ДВК и Искрами, а потом и двушками\ldots Вспомнились толстые подшивки <<Техники молодежи>>, валяющиеся тут и там учебники по <<васику>> и <<паскалю>>, запах канифоли\ldots Однозначно нужно нанести местным технарям визит вежливости.

-- Да они ботаники, сидят там целыми днями, ковыряются. Вообще, по-моему ты американский шпион, -- сделал вывод соседушка. -- Но я тебя не выдам, а то слишком уж здесь скучно и вожатка дурная.

-- Не американский, а марсианский. Прилетел выведать секрет приготовления пшенной каши, ага. -- Съехидничал я, расталкивая гаджеты по карманам и, взяв лаптоп подмышку, направился к выходу. Вообще, дурная идея была тут при Витьке в своем скарбе ковыряться, теперь прятать придется от не в меру любопытных глаз. Подумав, поднял с полу футболку и понюхал. Вроде не сильно воняет. Да, вот в нее комп и заверну, нечего тут отсвечивать иновременной техникой. -- Нож давай сюда. И это, рюкзак не трогай, его собаки обоссали.

-- Фу. А чо, он так и будет тут вонять? -- Сморщил нос Витька. -- На, забирай.

-- Ну, как приду -- постираю. -- Пообещал я, запинывая устроенный бардак под кровать.

\paragraph{}<++>


Дверь, с легким скрипом, отворилась и моему взгляду предстал местный <<Клуб Кибернетиков>>. Однако, прямо в детство попал: большой прямоугольный стол, полученный сдвиганием четырех маленьких, весь усеянный радиодеталями, да и паяльник здесь явно не для виду: в воздухе отчетливо пахнет канифолью. Полки заполненные книгами и журналами, на стенах модели самолетиков. А на столе у окна компьютер. Два компа в каком-то захудалом пионерлагере? Очень интересно. Ничего, что вместо монитора -- телевизор, да и корпуса у ЭВМ еще нет\ldots 

-- Шестьдесят одна, -- пробормотал я себе под нос, пересчитав клавиши. -- Однако.

-- Эй, не трогай, от компьютера отойди! -- раздался недовольный голос из коридора, ведущего, видимо, в кладовки и подсобки. В дверном проеме стоял серьезного вида пионер в очках: просто вылитый Шурик из знаменитых советских фильмов. -- Обходной лист пришел подписывать? Давай сюда.

-- Да, в общем-то, нет. Мне бы тут запитаться, подзарядить кое-что, да вот еще компьютер заинтересовал. Сам спаял что ли? -- я с уважением провел рукой по клавиатуре. -- А вот клава-то явно фабричная.

-- Ну не трогай ты, -- скривился пионер, -- да, с Электроником спаяли по статьям из журнала. Вон розетка, подключай свой магнитофон или что там у тебя.

-- Погоди-погоди. Так это <<Радио>> что ли? -- вспомнилось, как мы с отцом, прочитав цикл статей в одноименном журнале, пытались собрать что-то подобное. Точнее читал и собирал отец, а я лишь с интересом наблюдал и всячески мешал. -- Круто! Вот молодцы! -- Восхитился я. -- Кстати, меня Семеном зовут, -- и протянул руку.

-- Шурик. -- Пионер явно не ожидал такой реакции и с опаской, но тем не менее, крепко пожал руку. -- А я думал, ты как Витька с Лиской: будешь вожатую бесить, да задирать всех вокруг. <<Радио-78РК>> это, на двадцати пяти микрухах\footnote{Имеется в виду <<Радио-86РК>>, год и количество микросхем изменены умышленно}. А вот этот магнитофон -- внешнее запоминающее устройство, хотя и музыку иногда слушаем с него. Ты паять умеешь? 

-- Угу. И <<васик>> немного знаю.

-- Ну, тогда я тебя прямо сейчас в клуб и запишу. Все равно тебе к нам дорога. -- Перешел в наступление Шурик. -- Вот смотри, тут у нас\ldots

Честно говоря, я совершенно потерял счет времени и даже, как будто, забыл зачем сюда пришел. С упоением ковырялся в радиодеталях, листал журналы и книги, опять ковырялся, пытался что-то паять, вспоминая давно забытое хобби. Подметил, что вся обнаруженная литература, в том числе и периодика, издана до восемьдесят третьего года. Да и сложилось стойкое впечатление, будто этот мир (в том, что это параллельная реальность, сомнений, в общем-то, не осталось) несколько технически опережал наш на данном временном интервале. Внезапно скрипнула входная дверь.

-- Шурик! Ты опять обед пропустил! Там Ульянка цирк устроила: жабу в кастрюлю с супом подбросила. -- На пороге застыл белобрысый пионер с широкой улыбкой на добродушном лице. -- Ой. Семен, да? А ты чего здесь делаешь?

-- Привет. А ты, наверно, Электроник. -- Отложив паяльник, я поднялся и протянул руку. -- Похоже, я теперь третьим буду, коллега. А Шурик пленку проявляет.

-- Слушай, там тебя Ольга Дмитриевна искала, хотя теперь по всему лагерю Ульянку ловит. -- поведал пионер, пожимая руку. -- Так что, наверно, можешь не торопиться. 

-- Хе. Да, в общем-то, не собирался. А вот, что обед пропустил -- плохо. Тут пожрать что-нибудь найдется?

Пионер показал небольшой бумажный пакет. 

-- Вообще, я тут Шурику булок с кефиром захватил, возьми что-нибудь. -- Решив не наглеть, я ограничился одной булочкой и прикончил ее в пару укусов. 

-- Благодарю, коллэга. -- Видимо, прием пищи слегка улучшил мозговую деятельность, и я вспомнил, что хотел послушать местный радиоэфир. -- А вон тот приемник рабочий? И почему, кстати <<Электроник>>?

-- Да понимаешь\ldots -- замялся Электроник, -- на всех диапазонах в последнее время или тишина, то есть статика одна, или какие-то обрывки, не слышно ничего. Мы с Шуриком на эту тему голову ломали, даже на крыше антенну собрали. Добились того, что на СВ что-то странное поймали, слышно плохо. -- А <<Электроник>> -- потому, что Серега Сыроежкин, как в детской книжке.

-- Интересно. А что там за странность на средних? Давай, врубай уже.

-- Сейчас, -- паренек унес приемник на верстак, вытянул откуда-то из-за шкафа кабель, очевидно, стационарной антенны, и начал подключать его заместо съемной телескопической. -- Вот, иди сюда, сейчас.

Из динамика, сквозь статику, пробивались, в лучшем случае, отдельные слова, а то и части. Удалось разобрать только <<живых>> и <<безумие>>. 

-- Вот так и бубнит. -- Произнес, появившийся из фотолаборатории, Шурик. -- Он там еще, что-то про место сбора говорил. Запись зациклена. Иногда слышно лучше, иногда хуже, но понять в целом не получается. 

-- Коллеги, -- заявил я со всей возможной серьезность, -- а вам не кажется, что все это дурной знак? Вы в курсе, что телефон не работает?

-- Что? Как не работает? -- опешил Шурик, -- Я к Ольге Дмитриевне подходил, по поводу радио рассказал. Она мне говорила, что позвонит в город, и вообще директор вернется и во всем разберется. 

-- Ну, значит не позвонила. Власти, как всегда, скрывают. -- не удержался я от колкости. -- Кстати, по поводу директора у меня нехорошие мысли. И по поводу сторожа. 

-- А что с ними? -- подал голос Электроник.

-- В туман ушли. Не доехавших пионеров искать, видимо. -- я присел на верстак, и взял в руки лежавший там молоток. -- Ну, то есть, как медсестра говорит, на неком автотранспорте уехали. -- Взгляд упал на лежавший на этом же самом верстаке приличный кусок кожи. А ведь выхватить из ножен и ударить ножом -- это быстрее, чем снять этот нож с кармана, разложить и перехватить. Думается, что наслаждаться спокойной пионерской жизнью в новом теле не получится. -- Парни, а вам этот вот материал сильно нужен? 

-- Да нет, забирай. 

-- Мне тут еще шило бы, да ниток крепких и иголку цыганскую\ldots -- раскатал я губу.

-- Семен, вообще-то это клуб кибернетики, а не рукоделия, -- заявил Шурик. -- Хотя, шило я тебе дам. А вместо нитки проволока есть разная, вон в том ящике покопайся.

Катушка тонкой стальной проволоки обнаружилась быстро. Пожалуй, даже лучше будет, удобней шить за отсутствием иголки. 

-- Так чего с директором, сторожем и пионерами? -- задал вопрос, внезапно посерьезневший, Сыроежкин. -- Что там в тумане было?

Пока я обмеривал незаметно извлеченный нож, загородив спиной обзор пионерам, и раскраивал кожу, вкратце поведал свою историю, избегая скользких моментов. 

-- В общем, ясно, что ничего не ясно. -- Резюмировал Шурик. -- Туман, который как бы и не туман, пропавшая связь, странные передачи по радио. Семен, ты ничего больше нам рассказать не хочешь?

-- А чего еще рассказать? -- произнес я, пробивая шилом отверстия в заготовке, -- Есть подозрения, что водится там что-то похуже собачек. Потому, соваться туда не рекомендую. Да и вообще, по одиночке бы не ходить, особенно в темное время суток. 

Пионеры, видимо, решили обсудить новости наедине и удалились в противоположный угол помещения, я же продолжил работу. Долго ли, коротко ли, как говорится в сказках, но вот и готово. Придирчиво оглядываю плоды своих трудов: получились простенькие ножны финского типа. Вместо классического подвеса --  шнурок  на кончике и довольно просторная шлевка для крепления с внутренней стороны ремня. Разместил, проверил не мешает ли рубашка. Нормально. Потом уединюсь где-нибудь и потренируюсь.

Пионеры, с задумчивым видом, сидели за столом и молчали. Кажется, мои манипуляции их заинтересовали, но подойти и посмотреть они почему-то не решались.

-- Это у тебя нож там? -- нарушил молчание Электроник. 

-- Ну, -- буркнул я в ответ, -- рекомендую тоже чем-нибудь таким обзавестись. Вон, и молоток сойдет.

Дверь распахнулась даже не скрипнув. Вместо этого, она с силой ударилась о стену. А на пороге стояла вожатая, олицетворяя собой воплощение ярости.

-- Ах вот ты где! Я его по всему лагерю ищу, а он сидит тут себе спокойно! Ты почему бегунок не подписал? Почему на обеде не был? -- распалялась Ольга. -- Тоже мне, пионер. Не хватало мне этой троицы, и ты туда же!

-- Эй! -- крикнул я и хлопнул в ладоши. Вожатая умолкла на полуслове. -- Вы мне сказать дадите? Хорошо. Директор вернулся?

-- Нет, -- вполне себе спокойным голосом ответила Ольга. -- Но он обязательно вернется и со всем разберется. 

-- С чем <<всем>>? -- вкрадчиво осведомился Шурик. -- Что происходит, Ольга Дмитриевна?

-- Ничего не происходит, Александр. И вообще, некультурно вмешиваться в разговор. -- опять начала заводиться вожатая. -- Семен, где обходной?

-- Я его, кхм, уже употребил.

-- Что сделал? -- глаза на смазливом, но глупом личике, что называется, полезли на лоб.

-- Употребил. Подобающим образом.

-- Да что за смена мне досталась! -- завопила вожатая и выскочила за дверь.

-- Ну вот, коллеги. Говорила мама: ума нет -- иди в пед. А Ольга наша Дмитриевна -- яркое подтверждение правоты моей матушки. -- подытожил я, и подняв замотанный в футболку портативный компьютер, тоже направился к выходу. -- Рад был познакомиться, да и вообще верной дорогой идете, товарищи. 

-- Стой, -- Шурик резво вскочил со с стула, -- а что со всем этим будем делать? -- пионер кивнул в сторону, где, видимо, по его мнению должна была находиться граница тумана.

-- Да говорил уже: быть поосторожнее, чем-нибудь вооружиться на всякий случай. В остальном, как говорится, будем посмотреть.

-- Кстати, -- поднялся и Серега, -- скоро ужинать пора, пойдемте уж все вместе.

\paragraph{}<++>


К столовой понемножку стекались голодные пионеры, как всегда, разбиваясь на небольшие группки. Вон что-то оживленно обсуждают Алиса с Ульянкой, вокруг них собралось довольно много народу. А вон Витька, похоже, спешит в нашу сторону.

-- Привет, ботаники! Гы, ты шпиен, с ними что ли скентовался? Хотя да, советские технологии\ldots Все, молчу-молчу. -- Рыжий повернулся и поспешил в сторону столь же ярких пионерок.

-- Правильно, наверно, ты ему по носу заехал, -- задумчиво проговорил Электроник. -- Вроде ничего плохого не делает, а бесит.

Тем временем, двери столовой отворились, и пионеры пошли на штурм. Давка, прям как при погрузке в вагон метро часа пик. Остаться без ужина не хотелось, потому, не грех и поработать локтями. Кибернетики, кажется, пристроились в кильватер. 

Добраться до еды удалось в первых рядах. На ужин была котлета, пюре картофельное с зеленым горошком (видимо, чтобы спалось веселее) и <<чай>> с сахаром. Подозреваю, что весь подобный напиток готовится по одному рецепту: сахар  отжигается на сковородке, после чего вываливается в здоровенную кастрюлю с водой, все это доводится до кипения и получается, известное каждому, пойло.

Расположились за дальним столиком. Я же присел на место спиной к стене, чтобы просматривался весь зал. 

-- Вон вожатая суетится, поголовье что ли считает? -- ткнул я пальцем. 

Пионеры покосились в указанную сторону. 

-- Хоть бы не потерялся кто, -- проговорил Шурик. -- А так, может она и не безнадежна.

-- А по-моему, безнадежна, -- с набитым ртом ответил Сережка, -- Не расселись еще, а она мечется. О, Ленка к нам идет. -- Действительно, в нашу сторону направлялась та самая пионерка с неестественным, буквально -- фиолетовым, цветом волос, которую я видел утром на площади с книжкой. Ну и внешность, как здесь, видимо, заведено.

-- Можно? -- потупившись, произнесла девушка. 

-- Конечно, присаживайся, -- ответил я, -- меня, кстати, Семеном зовут.

-- Лена. -- Пионерка поставила поднос на стол и присела на свободный стул. 

Обсуждать при ней планы возможной обороны лагеря или что-нибудь в этом духе совершенно не хотелось, и ужин прошел, большей частью, в молчании.

-- Ладно, ребята, я там еще собирался свои вещи постирать, так что пойду я. До завтра. -- распрощался я с кибернетиками и Леной.

-- Пока. Бывай, -- попрощались пионеры.

\paragraph{}<++>

Дверь в домик заперта, значит никого. Хорошо. Бросив, завернутый в футболку, лаптоп на кровать, я начал выгребать из-под нее вещи. Штаны драные, да и великоваты стали, но можно приспособить. А вот с носками и ботинками -- беда. В той жизни я носил растоптанный сорок пятый. Здесь же, нога потеряла пару размеров. Стельку подложить, портянку намотать и как в сапог? А то перспектива остаться с одними сандаликами как-то не очень радует.

Так. Фляжку надо будет промыть и наполнить. Интересно, здесь воду из-под крана пить можно? Жаль, свой походный фильтр в дорогу не захватил.

Герметичный мешочек с документами, второй с сухим горючим и зажигалкой. Надеюсь, документы никто не видел. И не увидит. Всю жизнь мечтал сжечь паспорт. Вообще, не похоже, что здесь кто-то шарился. Я открыл обложку и взглянул на потрепанную, фактически, до состояния туалетной бумаги, книжечку. С фотографии на меня смотрел стриженный, практически налысо, паренек двадцати лет, с гладко выбритым и очень злым лицом.  <<Да, не самое радостное было время>> -- подумал я, убирая документ.


Пара чистых, но ставших большими, носков, да одна пара трусов. Те хоть хоть кое-как носить можно. Может, стоило бы вожатую спросить насчет нижнего белья? Вообще, не нужно было, наверно, над ней издеваться. Какого, собственно, черта я веду себя как школьник? Может быть гормоны юного, шестнадцатилетнего тела, виноваты? А кто его знает\ldots

Сгреб всю свою снарягу в охапку, подумав, захватил и комп. Во, самое главное -- мыло чуть не забыл. Ну, пора отправляться  приводить это все в порядок. На улице уже стемнело, а на небе появились яркие точки звезд. Граница тумана в темноте выглядела особенно зловеще. Кое-как, с полными руками, запер дверь и отправился в сторону умывальников.

Кто-то там уже вовсю принимал водные процедуры. 

-- Хе. Однажды отец Онуфрий обозревая окрестности Онежского озера обнаружил обнаженную Ольгу, -- пробормотал я себе под нос.

-- Что? -- девушка услышала шаги и бормотание, но слов, видимо, не разобрала. Да и не обнаженная она совсем.

-- Вечер добрый, говорю. -- ответил я и сложил вещи наземь. -- Всех в столовой пересчитали?

-- Да что ты все время язвишь? -- вскинулась Ольга. -- Я к тебе как к человеку, к пионеру, а ты гадости одни делаешь!

-- Нет, Ольга, я извиняться ни за что не собираюсь. Но все-таки, никто не пропал?

-- Мику на ужин не пришла. -- тихо ответила вожатая. -- Но завтра Петр Иваныч вернется и обязательно ее найдет. 

-- Вы сами-то в это верите? И кто такая Мику? Где живет, где обычно время проводит? Какого черта вы ее не ищете, а?

-- Девушка, пионерка, из музыкального клуба. Я туда ходила, и домик ее проверила. Лена ее не видела\ldots

-- Очень интересно. Давайте так: вы мне шмотки стираете, а я пойду поисковый отряд собирать. Фонарь у меня есть, -- щелчок и тьму рассек узкий луч яркого света.

-- С какой стати я твои грязные вещи стирать буду? -- поморщилась Ольга. -- Нашелся тут командир-спасатель. 

-- Я, по-крайней мере, голову в песок не прячу. Искать эту самую Мику будем, или нет? Как она выглядит хоть? -- пошел я в наступление, пододвигая ногой ворох шмоток в сторону вожатой.

-- Ты ее сразу по длинным хвостикам узнаешь, прям до пояса. И цвет такой голубой-голубой, прям как у Мальвины. -- Кажется, Ольга смирилась с предстоящей ролью прачки.

-- Что-что у нее голубое? -- решил я, будто что-то недопонял.

-- Волосы, -- как ни в чем не бывало, ответила девушка.

-- Круто. Мода у вас тут что ли такая? -- Вспомнилось, что в толпе пионеров на линейке и после, таки действительно мелькало несколько голов с волосами неестественных цветов. -- Ай, ладно! -- я развернулся и быстро пошел в сторону своего жилища (надо бы комп все-таки оставить). -- Носки не растеряйте!

Открыть дверь, быстренько затолкать сверток под одеяло, закрыть дверь. Так, теперь своих новых товарищей отыскать. Бегом к <<Клубу Кибернетиков>>. Свет не горит. На всякий случай подергал дверь -- заперто. 

Замечательно. Где живут Шурик с Сыроежкиным я не знал. Ладно, обойду лагерь. Загляну в темные уголки. 

Куда может пойти скучающая девушка? Например на речку. Топиться с тоски, ага. Впрочем, берег надо осмотреть.

На улице, разделяющей спальный район, примыкавший к <<артерии цивилизации>> показалась знакомая красная фигурка. То есть, девчонка в красной футболке и с красными же волосами. Ульянка.

-- Чего на обеде не был? -- сходу крикнула она, завидев меня. -- Я там такое блюдо приготовила!

-- Да уж наслышан. Французская кухня: лягушачьи лапки вместе с самой лягушкой. Хоть всем хватило?

-- Да уж от пуза наелись! -- Проказница залилась радостным смехом. -- Особенно вожатка! А ты на речку?

-- Тащемта, я Мику ищу. Не видела?

-- Ну, допустим, видела. -- Ульянка скорчила хитрую рожу. -- Расскажешь матерный стишок, скажу где.

-- Мда. Ну слушай:
\begin{verse}
	Из-за леса, из-за гор\\
	Показал мужик топор.\\
	Но не просто показал:\\
	Его к херу привязал!\\ 
\end{verse}

-- Тьфу, тоже мне, он же короткий совсем, -- надулась рыжая.

-- А ты поэму ожидала услышать? Колись давай, где Мику.

-- Так она с Алисой и Витькой в нашем домике в карты играет. Во-о-о-он в том! -- Ульяна указала пальцем на один из домиков. 

-- Что, они и сейчас играют?

-- Ну да. Я вот только оттуда\ldots

Тут уже смехом зашелся я.

-- Да, надо бы все-таки видимость поисков создать, -- произнес я отсмеявшись.

-- А чего, зачем? -- удивилась девочка.

-- Ну, пионер должен обещания выполнять, а я вот вожатой обещал\ldots -- состроил я серьезную гримасу.

-- Что-то ты врешь, по-моему. Хотя, пошли тоже с ними в карты играть!

О том, что в данном домике живут две рыжих непоседы, пожалуй, можно было и догадаться. Например, по висящему на двери <<Веселому Роджеру>>.

-- А я еще игрока привела! -- с порога воскликнула Ульянка.

На полу, при свете горящей посреди комнаты здоровенной оплывшей свечки, сидели две пионерки и пионер. Собственно, Витька, Алиса и, очевидно, Мику. Тоненькая, с несколько азиатской внешностью и, действительно, здоровенными голубыми хвостами. Впрочем, в темноте они казались синими.

-- Всем привет, кого не видел. Кто не знает, -- обратился я к азиатке, -- я Семен.

-- Да все тебя уж знают, -- пробормотала Алиса. -- Садись давай, сейчас карты буду сдавать\ldots В <<подкидного дурака>> играть, я надеюсь, умеешь.

Час, пожалуй, мы так просидели. Мику без устали болтала: о музыкальных инструментах, о неведомых мне группах, о Японии, о том, что никто не хочет записываться в музыкальный кружок, опять об инструментах, о погоде, о тумане, который она, оказывается, тоже заметила. В общем, как бы обо всем, но в то же время -- ни о чем. В какой-то момент я подумал о кляпе. А что, вставить кляп, связать по рукам и ногам, и отнести вожатой. Видимо, рыжие тоже устали от этой болтовни, и игра постепенно сворачивалась. 

-- Мику, давай-ка я тебя провожу. А то темно уже, -- предложил я, как только мы покинули домик. -- Кстати, надо бы к вожатой заглянуть, она о тебе беспокоилась\ldots 

-- Конечно, пошли. -- согласилась девушка. -- А ты на чем-нибудь играть умеешь, я ведь тебя так и не спросила? А то заходи ко мне в музыкальный кружок, у меня там столько разных инструментов: гитары акустические и электро, трубы, кларнеты и даже барабанная установка!

-- Только на нервах, -- скривился я. Подумалось, что идея с кляпом была не лишена смысла. 

-- Ой, только ты меня и ребят Ольге Дмитриевне не выдавай! -- вдруг переключилась девушка на более разумную тему. -- Давай, что-нибудь такое придумаем?

-- Гм, что бы такого придумать? Давай так: ты гуляла по берегу реки, играя на каком-нибудь инструменте, там я тебя и отыскал.

-- Ой, а зачем ты меня искал? -- удивилась девушка.

-- Ну так, кхм. Мне, вообще-то, Ольга Дмитриевна поручила. Тебя на ужине не было.

-- А нельзя сказать, что я в клубе была?

-- Не выйдет. Вожатая там побывала. И в твоем домике тоже. Так что прогулки под луной по берегу реки, хе-хе-хе.

Так мы и дошли до дома семнадцать. 

-- Тук-тук! Откройте, милиция! -- постучал я в дом вожатой.

Дверь распахнулась и нашему с Мику взгляду предстала Ольга в махровом халатике. 

-- Нашел! Все-таки нашел! Ой, как же все-таки хорошо! Мику, где ты была?! -- причитала нерадивая вожатая. Не давая Мику открыть рот, я быстренько рассказал заготовленную историю. 

-- Ладно, Ольга, кхм, Дмитриевна. Давайте я уже девушку домой отведу, и спать пойду. Кстати, как там мои вещи-то?

-- Охо-хо, -- вздохнула Ольга и поморщилась. -- Я их за своим домиком развесила сушиться, как высохнут -- заберешь. Ладно, Семен, Мику, спокойной ночи. -- Дверь закрылась, и мы с синеволосой девушкой отправились в противоположный конец улицы.

-- Семен, а какие вещи? -- покосилась на меня Мику.

-- Да так, это наш с Ольгой Дмитриевной, маленький секрет\ldots

\paragraph{}<++>

Когда я вернулся домой, Витек уже спал, укутавшись в одеяло с головой. Я же переложил сверток под подушку, разделся, и прихватив нож в кровать, тоже завалился спать.


\chapter{День третий}

Я опять иду по дороге. Вполне обычная трасса, идущая ниоткуда в никуда. А вокруг степь без единого холмика или балки. Желтая, высохшая трава до самого горизонта. Нещадно палит солнце. На высоком, пронзительно голубом, небе ни облачка. 

Пот градом катится по лбу и заливает глаза. Хочется пить, но с собою ни воды, ни еды. Надо идти. Шаг, еще шаг/ldots  

Все, что у меня есть -- хэбэшка не по размеру, сапожищи и портянки, да совершенно неуставный нож, что был при мне в день <<Ж>>. Его удалось сохранить, невзирая ни на что.

Что там -- далекий шум мотора? Поворачиваюсь и вглядываюсь вдаль. Погоня? Кто-то все-таки хватился и решил меня поискать? Возможно. Укрыться здесь негде -- многие километры плоской, как стол, степи. Сколько пройдено за эти несколько дней? Остается только ждать.

Точка на горизонте все увеличивается и увеличивается, постепенно принимая очертания белой <<волги>>. Просевшая под грузом овощей на крыше, с ржавым днищем, машина замедляет ход, скрипя, видимо, совсем уставшими тормозными колодками.

-- Эй, солдатик! Далеко идешь? Садись, поехали!

Открываю переднюю пассажирскую дверь и плюхаюсь на сиденье. За рулем мужик лет пятидесяти. Смуглое, выгоревшее на солнце, лицо. Из-под, когда-то белой, кепки видно темные с проседью волосы, над верхней губой седые же усы. Такая же, как кепка, грязная футболка и спортивные штаны известной фирмы <<Абибас>>.

Оглядываю его, он оглядывает меня. Молча трогаемся.

-- Смурной ты какой-то. И тощий совсем. Не кормят, что ли? Вон, на заднем сиденье в сумке пошарь, там бутерброды были. А в ящиках яблоки.

-- Спасибо.

Подвеска машины мягко глотает неровности дороги. Ленивое порыкивание мотора, шуршание покрышек по сухому асфальту, скрип пружин в сиденьях -- все сливается в эдакий монотонный, убаюкивающий звук. Поел --  разморило. Мужик с разговорами не лезет, и то хорошо. Глаза начинают слипаться, пространство вокруг смазывается и мутнеет\ldots
\\
\paragraph{}<++>

Сознание возвращается, и я понимаю, что по-прежнему нахожусь в пионерском лагере. На этот раз, сон не вызывает удивления и каких-то вопросов -- это мои воспоминания. Ту дорогу я помню хорошо, ведь именно она привела меня в неопрятную квартиру с заклеенными окнами.

Что день грядущий нам готовит? Надо вставать, тогда и узнаю. Неспешно одеваюсь, вешаю на привычное место ножны. Сосед все еще спит. Ну и ладно, не будем его будить. Захватив подмышку сверток с компьютером, беру пакет с умывальными принадлежностями и отправляюсь к колонкам. 

Время раннее, посему умыться и почистить зубы удается в одиночестве. Погодка, похоже, будет славная: солнце уже припекает, на небе ни единой тучки, и только серая мгла висит на неопределенном расстоянии. Ладно. Надо бы отнести пакет с мыльно-рыльными да зарядку сделать. Негоже запускать доставшееся тело.

Решено, отправляюсь на стадион. Генда на площади все также укоризненно глядел на каждого, представшего перед его взором, в столовой, похоже, начали готовить завтрак. Может быть, и что-то приличное будет в кои-то веки\ldots 

А вот и волейбольная площадка виднеется: зеленая трава, натянутая сетка, определенно неплохо. Здесь же какая-то постройка явно хозяйственного назначения. Надо бы потом, на всякий случай, проверить. 

Стадион -- традиционное футбольное поле с травяным покрытием опоясывает беговая дорожка.  О, неужто вон там раздевалка и душ? Может быть там и горячая вода есть\ldots Шкафчики на ключ не запираются, в стране побеждающего социализма, видимо, принято доверять ближнему. Ну ладно. Не долго думая, я сунул в приглянувшийся шкафчик свой сверток, снял рубашку и остался в одних шортах. Попрыгав и прикинув общий баланс, отправил вслед за рубашкой и плеер с наушниками, после чего и мультитул с фонарем. Мде, а вот в сандалиях бегать как-то не очень.  Даже босиком, пожалуй, удобнее.

Разомнусь для начала. Повращать головой, покрутить руками, наклоны, приседания, провести серию <<гири>>: май, уширо и маваши. Ну и турник не забудем: двадцать подтягиваний дались легко. Неплохо бы провести полноценный тест Купера: таймер в телефоне есть, а один круг по стадиону, наверняка двести пятьдесят метров. 

О, любительницы физкультуры и спорта подошли: на футбольном поле в облегающих спортивных костюмчиках разминались Лена и Славя. 

-- Привет, девчонки! -- крикнул я, направляясь в их сторону. 

-- Привет, привет, -- откликнулись девушки. -- Это хорошо, что ты тоже спорт любишь, Семен, -- одобрительно произнесла Славя.

-- Ага. Вообще, мне бы тоже какой-нибудь костюмчик да кеды не помешали. -- посетовал я на отсутствие нужного снаряжения и окинул пионерок взглядом. Да уж, хороши.

-- Сейчас, подожди, мы с Леной разминку закончим, и что-нибудь подберем тебе на складе.

Отошел чуть подальше и стал любоваться ладными фигурками. Видимо, девушки почувствовали, что я на них, откровенно говоря, пялюсь, и закончили упражнения довольно быстро.

-- Пошли, Семен, -- помахала рукой Славя, -- сейчас, я ключи в раздевалке возьму\ldots

Хранилищем спортивной формы, как и белья была, замеченная мной ранее, постройка. Здесь же лежали подушки, одеяла и тому подобное. 

Форму подобрали быстро: короткие шортики без карманов да футболка, высокие кеды, две пары носков. И двое плавок прихватил, как мне хотелось надеяться, нужного размера.

-- Спасибо, Славя! Пойдем, наверно, -- поблагодарил я девушку. -- А что, другие пионеры на стадион не ходят?

-- Да нет, -- Славя поглядела куда-то в сторону площади, -- просто подъема еще не было. А так, здесь с девочками Виола занимается, а с мальчиками Петр Иванович занятия проводил.

-- Удивительно, что не наоборот, -- сострил я, -- Значит, директор не вернулся? Дурной знак. 

Упоминание директора (а, может быть, не слишком удачная шутка) поставило на разговоре точку, и к стадиону мы шли погруженные каждый в свои мысли. 

Быстренько переодевшись я, зажав телефон в руке, направился к беговой дорожке. Поставить таймер на двенадцать минут, на старт, внимание, марш! Выкладывался, как положено, по максимуму, периодически замедляясь, когда уже становилось совершенно нечем дышать, а по горлу как будто прошлись наждачкой.

-- Бип-бип-бип! -- Запиликал телефон. Одиннадцатый круг. Больше двух с половиной километров -- весьма недурственный результат. Выдохся, правда, совершенно. Пройдя еще пару кругов шагом, выполняя упражнения для восстановления дыхания, я отправился в душ. 

Переодеваясь,  заметил, что на стадион понемногу подтягиваются пионерки и немногочисленные пионеры. Видимо, в отсутствии директора, большинство филонит. Остаться что ли, на девок поглядеть? А вон и Виола выходит из женской раздевалки. Какой же у нее размер? Всяко не меньше четвертого. Ладно, пойду-ка отсюда, пока она чего-нибудь не выкинула.

\paragraph{}<++>

Время перед завтраком я решил провести с пользой. Неплохо бы лодочную станцию обыскать, все-таки там жил-поживал сторож со столь говорящим отчеством. Сосед по-прежнему дрых, что называется, без задних ног, видимо, решив наплевать на линейку. Хотя, я сие замечательное мероприятие, пожалуй тоже пропущу. 

А вот и лодочная станция. Находилась она, надо сказать, буквально под боком.  Речка в этом месте изгибается, принимая подковообразную форму. Посреди подковы пара островов, а дальше виднеется железнодорожный мост. Интересно. Мне кажется, или там, практически на горизонте, та же самая стена тумана? Сколько не напрягал зрение, так и не понял. Может быть, это всего-лишь утренняя дымка? 

К причалу привязана пара небольших деревянных лодок, разумеется, без моторов. На берегу же обнаружилась будка. Ага, скромная обитель лодочника. Типичный вагончик-бытовка, в каких обычно живут строители. А вот на двери навесной замок. Неприятно. Небольшое оконце закрыто. В принципе, пролезть в него не проблема, но бить стекло пока не хочется.  Кто-нибудь обязательно услышит. 

Встав на цыпочки, заглядываю в окно: лежанка, какие-то ящики, стол или скорей верстак, на котором лежат незамысловатые столярные инструменты. Особенно впечатлил  знатный топор. В углу виднеется примус. 

-- Семен! -- ну вот, уже привлек к себе внимание. Славя решительно направлялась в мою сторону. -- Афанасия Кузьмича ищешь?

-- Да порыбачить хотел, -- надо же как-то выкручиваться. -- Думал, может удочки и снасти тут найду. 

-- Ну вообще-то, -- Девушка  улыбнулась, -- Афанасий Кузьмич с директором уехал, а удочки у него в будке. Ребята обычно не очень рыбалку любят.

-- А ты? -- Я отошел от окна и сделал пару шагов в сторону собеседницы. -- У тебя ж, вроде, ключи были. Может и компанию составишь?

-- Раньше любила. -- Легкая грусть скользнула по открытому лицу Слави. -- А от будки у меня ключа нет. Да и был бы, как-то нехорошо входить в жилье без приглашения. 

-- Ну да, ну да, -- покивал я головой. -- Конечно не хорошо, мой дом -- моя крепость. Кстати, никто больше за территорию не уходил?

-- М-м-м, ну ребята из нашего отряда вроде не выходили. А чего? 

-- Да я говорил уже, туман этот меня беспокоит. Заблудиться в нем легко. Если ты один, то плутать, ходя кругами можно, пока куда-нибудь не свалишься и шею не сломаешь. А там уж и друзья человека твой хладный труп подъедят. 

-- Бр-р-р, Семен, ужас какой. -- Пионерку передернуло. 

-- А тебя не беспокоит, что мужиков уже почти двое суток нет? Тем более не пешком пошли, а на <<запоре>> поехали.

-- Да вообще-то беспокоит. Меня это все беспокоит. У кибернетиков приемник почти ничего не ловит. Телефон не работает, а в лесу я видела странные следы. 

-- Никто не выходил, да? -- От упоминания следов на душе стало тревожно, -- как выглядели?

-- Ну, я не говорила, что никто не выходил. Так я каждый вечер в лесу гуляла. А вчера эти следы увидела. Неправильные и нехорошие какие-то следы. Похожие на человеческие, только узкие и пальцы слишком длинные. Мне страшно стало и я в лагерь быстренько вернулась. 

-- А, дерьмо! -- В сердцах, я плюнул на землю и попытался огладить несуществующую бороду. 

-- Что? -- В голубых глазах явственно читался испуг.

-- Знакомец старый. Пошли вожатую искать, иначе добром это все не кончится. 

Славя указала куда-то в сторону площади:

-- Так вон она, сейчас на линейку всех будет собирать.

Генда все так же укоризненно взирает на строящихся пионеров. Судя по всему, некоторые, особо прилежные, решили, что крепкий и здоровый сон организму нужнее, чем  сомнительной полезности организационные мероприятия.

-- Строимся, строимся, -- Зевая, пыталась командовать вожатая.

-- Ольга, кхм, Дмитриевна, на минутку. -- окликнул ее.

-- Доброе утро, Семен. Что-то случилось? -- удивилась Ольга, но все-таки пошла за нами. 

-- Вас вчера история с Мику ни на какие мысли не натолкнула? -- заговорил я со всей серьезность. -- Как и отсутствие директора. Я думаю, пора о мерах безопасности поговорить.

-- Что? -- Сказать, что вожатая удивилась -- ничего не сказать. -- Какой безопасности? Ты о чем, Семен?

-- Собственно, когда я сюда сквозь туман брел, то иногда натыкался на следы крупного зверя. Значения-то не придал особого, да и сильно уставший был. А вот сейчас со Славей разговорились и, оказывается, она эти следы  здесь в лесу видела. А следы, как говорит, медвежьи. Да, Славя?

-- М-м-м, да, Ольга Дмитриевна. -- девушка покраснела и отвела взгляд. Похоже, ложь дается ей нелегко, пусть даже -- эта ложь во благо.

-- Медведь, медведь, никогда не видела медведей. А они правда опасные? -- вожатая начала проникаться.

-- Ну как. Увидит человека -- может и не тронет, а может и порвет. Медведь -- зверь крупный, сильный и быстрый. Хищник.

-- От ведь! И Петр Иваныч с Кузьмичом уехали\ldots Что же делать-то? Ты чего предлагаешь?

-- За территорию никого не отпускать. Детвору собрать в надежных помещениях под присмотром старших.  Организовать вооруженные патрули, проверить целостность ограждения. Периодически делать обход. По одному никому не ходить. Даже в туалет: один гадит, другой страхует. По-хорошему, консервных банок в глухих местечках поразвесить.

-- Семен, -- округлила глаза Ольга, -- какое оружие? Ты что? Это пионерлагерь, а не военная часть!

-- Топоры, лопаты, арматура -- хоть что-то. 

-- Ну нет. Не хватало еще, чтобы тут дети с дубинами расхаживали. -- все-таки уперлась вожатая. 

-- Ага. Пусть лучше их сожрут.

-- Семен, -- продолжала упираться Ольга, -- ты, между прочим, ведешь себя странно. Может быть, и не медведь это. А, например, енот. А я тут всех на уши поставлю, дубины раздам, и кто-нибудь обязательно поранится\ldots

-- Ольга Дмитриевна, -- подала голос Славя, -- младшие отряды, хотя бы, давайте с улицы уберем и предупредим всех. Это правда медведь.

-- Ну хорошо. Сейчас все соберутся, и я объявление сделаю. -- вняла доводам разума Ольга и поспешила в центр площади.

Славя смотрела на меня со смесью неодобрения и беспокойства, но, в то же время, в ее взгляде читалась и надежда. 

-- Это ведь не собаки тебя там покусали? -- спросила девушка, отведя взгляд. 

-- Собаки тоже были. Но они, хотя бы, пиками не кидались.

-- И что ты собираешься делать?

-- Ну, по-хорошему, неплохо бы выследить гадину и убить. А так, все то, что Ольге говорил. Народ убедить, думаю, у тебя лучше получится. Я там еще с кибернетиками пообщаюсь и соседом. Пошли!

-- Куда? -- не поняла Славя.

-- Сторожку обыщем, -- решил я все-таки воплотить свой изначальный план. -- А ты мне в этом поможешь: на шухере постоишь.

-- Но нельзя ведь так, -- заупрямилась славя, -- чужое жилье все-таки.

-- В Советском Союзе все принадлежит трудящимся, -- нашел я отговорку, -- Пошли давай, пока тут все торчат.

И вот, Славя стоит эдак в десяти шагах и смотрит в сторону площади, в руках держит мой сверток с техникой. Видимо, совершать какое-либо правонарушение ей совершенно внове. Я же, прикинув все возможные способы проникновения, остановился на самом банальном: высадить стекло и влезть. Быстренько сняв с себя рубашку, намотал ее на руку.

Бум. Бум. Чуть посильнее. Стекло треснуло, и часть его провалилась внутрь рамы с приглушенным звуком. Оглянулся. Славя стоит, буквально, втянув голову в плечи. Но никто к нашим действиям интереса не проявил. Расшатав остатки стекла, я вынул его из рамы и удалил мелкие осколки. Ну вот, путь свободен. Ухватился за раму, подпрыгнул\ldots Вуаля, и я внутри!

-- Славя, Славя, сюда иди. -- Подумалось, что пусть она мне здесь компанию составит, чем с таким видом стоит у вагончика.  -- Залезай давай. Вот, за руку хватайся. 

Славя, однако, протянула мне компьютер  и сама резво влезла внутрь вагончика. Хорошая у нее все-таки физическая форма. Да и формы\ldots Кхм. 

-- Во! -- взяв топор с верстака, взвесил его в руке. -- Сидит, как влитой, достается за полсекунды, твари очень боятся. Интересно, а там у нас что? -- Я указал  на лежанку, которая, на самом деле, представляла из себя здоровенный ящик, с наброшенным поверх матрасом.

-- А там, Афанасий Кузьмич, наверно, ружье свое хранит, -- невинно произнесла Славя. -- Он тут охотится иногда в лесу\ldots

-- Что?! -- Возопил я, -- Ружье! И ты до сих пор молчала?! -- Резким движением  отшвырнул матрас, и обнаружил, что крышка ящика-лежанки заперта на такой же, как и входная дверь, навесной замок. Не беда! Удар обухом топора. Еще удар! И еще! Звякнул, упавший на пол, кусок металла. Откинуть крышку\ldots 

Так-так, что тут у нас? Брезентовый чехол, где в разобранном виде собственно, хранилось, ружье. 

-- Ага! -- радостно воскликнул я, извлекая из чехла поочередно приклад с колодкой, блок стволов и цевье. -- Ух ты! Пятьдесят четвертый!\footnote{ИЖ-54 -- двуствольное ружье двенадцатого калибра с горизонтальным расположением стволов.} Ну, Кузьмич, не посрамил гордое имя!

-- Ты с ним хоть обращаться-то умеешь? -- с сомнением покосилась Славя. -- Сейчас это редкость.

-- Что? -- удивился я, закрепляя стволы на колодке и прилаживая цевье на место, -- А уроки НВП?

-- Уроки чего? -- не поняла пионерка.

-- Ну как. Начальной военной подготовки. Калаш разбирать-собирать, противогазы натягивать по команде <<Газы>>, метать гранату и на сборы иногда выезжать\ldots Мой дед его вел, как в отставку вышел.

-- Ничего не понимаю, -- девушка задумчиво крутила в руках свою толстую русую косу. -- Ничего такого не было. А что такое <<Калаш>>?

-- Автомат Калашникова же. Основное стрелковое оружие советской армии. И самое массовое оружие в мире. -- Просвещая девушку, я продолжал рыться в ящике. Ага. Три пачки <<Рекорда>> в папковой гильзе.

-- Не знаю, Семен. У военных, вроде винтовка какая-то, деревянная. Не <<трешка>>, как у дедушки, а такая покороче, у нее еще штык откидывается.

-- Зашибись. СКС\footnote{Самозарядный карабин Симонова}, похоже. -- Пробурчал я, продолжая копаться в ящике. --  Вот дерьмо, тройка. А это получше -- четыре нуля, деревенский стандарт. 

-- Семен, а у тебя, значит, родители военные? -- Славя присела на верстак. 

-- Деды и прадеды -- военные, а родители -- инженеры. -- Я оглянулся на девушку, извлекая из ящика все новые предметы: пара десятков латунок\footnote{Имеются в виду латунные гильзы}, коробка капсюлей к ним, мешочек той же дроби-нулевки, пара банок <<Сокола>>\footnote{Бездымный порох <<Сокол>>, выпускается с начала 20го века} и, конечно же, слегка порезанный валенок. -- Ух, молодца, Кузьмич, с этим мы тут всех тварей уделаем. -- Окинул я взглядом, выросшую на полу, гору богатств. -- А вообще, деды, наверно, у всех воевали. Обана, смотри-ка, барклай!\footnote{Ручная закрутка для патронов.}

-- Где воевали? -- Славя встала с верстака и подошла ко мне вплотную, все также теребя свою косу. 

-- Ну как. Великая Отечественная Война, по-моему, всех затронула, -- не уловил  подвоха. 

-- С Наполеоном? -- Девушка отпустила, наконец, свои волосы. -- Семен, последняя война в восемнадцатом году кончилась, с тех пор у нас только мир и процветание. 

-- Да? -- похоже, на радостях, дал маху, -- Ну ладно, -- на дне ящика оказался замечательный кожаный патронташ с тремя отсеками по восемь патронов, -- Похоже, я из другого мира, у нас там постоянно воюют\ldots

-- Опять издеваешься? -- Славя глядела с осуждением. -- Надеюсь, ты тут по лагерю с ружьем бродить не собираешься?

-- Мда. Кстати, про ружье, -- я поднялся, и взяв двустволку, вложился. Длинное и неразворотистое. С таким бы на гуся ходить, а не неведомую зверюшку охотить. Поди еще и чоки\footnote{Дульные сужения.} в стволах. И приклад длинноват. Хорошо, хоть ложа полупистолетная, а не прямая. Что-то со всем этим нужно делать. -- Ну-ка, посторонись. 

Ножовка по металлу нашлась быстро. Как и пара дощечек, между которыми в тиски лег блок стволов. Пилите, Шура, пилите, они внутри золотые, как говорится. Отрежем половину, как раз получится <<ружье кучера>>, и пулей можно будет стрелять, и картечью. Надо бы, кстати, пулелейку сообразить, да тройку никчемную переплавить.

-- Семен. -- Кажется, Славя одобряла мои действия все меньше и меньше. -- Ты зачем ружье портишь? 

-- Не порчу, а модернизирую. Под тактическую задачу, хе-хе. -- Усмехнулся я, когда кусок стволов брякнулся об пол. -- Ну вот. Сейчас напильничком пройдусь, потом наждачкой\ldots 

Приклад тоже на опиловку, Ничего, что совсем короткий получается, такой проще упирать под ключицу. Лак, конечно, пострадает, но что поделать? Замотаем синей изолентой, вон ее здесь целых два мотка обнаружилось. Интересно, на кой Кузьмич антабку-то заднюю выкрутил?  

-- Ну вот, и новых коцек, вроде не появилось. -- Собрав ружье, я с удовлетворением  оглядел плод  своего труда.  -- Еще бы срез подворонить, да приклад лаком перепокрыть\ldots

Славя молча созерцала мои действия.

-- А вот под рубашку все равно не лезет, как ни примеряйся. Но лупару из него делать не буду. -- споро переломив двустволку, загнал в патронники по картонному цилиндру, и заперев ружье, убрал его в чехол. -- Пусть пока здесь полежит.

-- Все-таки нехорошо, -- Славя прошлась из одного конца сторожки в другой, -- вломились, все переворошили, ружье испортили. Вот вернется Афанасий Кузьмич, что мы ему скажем?

-- Славя, а Славя. А ты уверена, что он вернется? -- я мрачно посмотрел на девушку, вставляя патроны в гнезда патронташа.  

-- И что мы теперь будем делать? -- Девушка вновь принялась трепать свою косу.

-- Ну, для начала, неплохо бы этот карамультук отстрелять, да патроны проверить. Дабы не случилось конфуза.  

-- И где ты собрался стрелять? Да еще через весь лагерь со всем вот этим идти? Нет, Семен, так дело не пойдет.

-- Ладно. Пойдем, я свой рюкзачок заберу. Кстати, -- вспомнил я о завернутом в футболку лаптопе, -- это, пожалуй, здесь оставлю. 

-- А что это? -- Девушка подошла к ящику и попыталась развернуть футболку. 

-- Осколок другого мира. Пошли давай. Дамы вперед, прошу.

\paragraph{}<++>


Завтрак мы, безусловно, пропустили. Судя по положению солнца, время шло к обеду. Несколько ребят шло с полотенцами в сторону пляжа, но детворы из младших отрядов на улице видно не было. 

-- Неужто Ольга мозги включила? -- предположил я, бодро шагая в сторону домика вожатой. 

-- Семен, а почему ты на нее так сразу обозлился? -- догнала меня Славя. 

-- Да, понимаешь, не люблю людей, что зацикливаются на мелочах, игнорируя реальные проблемы. 

Домик вожатой был заперт. Видимо, Ольга действительно организовала какие-то массовые мероприятия для детворы. А, может быть, свалив это дело на кого-нибудь, ушла на пляж. Стоило обойти домик, и обнаружились мои вещи, развешанные на подоконнике. Рюкзак и штаны с носками уже высохли, а вот ботинки были влажными. 

-- Пойдем, я лишнее барахло дома сброшу. -- Предоставлять Славю самой себе мне очень уж не хотелось. Мало ли, что ей взбредет в голову: ружье там перепрячет\ldots

Витьки в домике не было, однако недавнее его присутствие выдавала табачная вонь.

-- Заходи, заходи не маячь, -- Славя неуверенно переступила порог. 

Бросив ботинки на пол, я стал переобуваться в кеды. Все лучше, чем в сандаликах ходить. Подумав, я вытащил из рюкзака носки. Ну вот, готов к труду и обороне. 

\paragraph{}<++>

У будки Кузьмича, как будто, ничего не изменилось. Та же безлюдная пристань, те же лодки, то же выбитое окно в вагончике. 

-- Стой на шухере. -- Произнес я и полез в окно. Разобрать ружье и затолкать в рюкзак, или замотать в штаны? Кажется, есть идея получше: на глаза попалась скатка с удочками. Размотать брезент, вложить ружье, замотать. Ну вот, более-менее. Хотя, кого я тут стесняюсь? Ладно, не будем пока шокировать местных эльфов. Так, а вот ремень от ружья уберем в рюкзак.

Подумав немного, надел патронташ на пояс и прикрыл его рубашкой. Вроде извлекать нож не мешает да и в глаза сильно бросаться не должен.

Накинув на плечи рюкзак и прихватив скатку, я лихо выпрыгнул в окно. 

-- Ага! -- из кустов выскочило красное пятно. -- А я все видела! 

Ульянка. Вот только ее сейчас не хватало.

-- А ты что здесь делаешь? -- я смерил девочку строгим взглядом.

-- Ну не с малышней же в библиотеке сидеть. -- егоза подбежала к выбитому окну и попыталась туда заглянуть. -- Вожатка что-то там про медведя на линейке рассказывала и нудила.

-- И ты, конечно, решила наплевать, взрослая?

-- А то! -- Кажется, Ульянка слегка хрюкнула. -- А я теперь с вами на рыбалку пойду!

-- Ну зашибись. Файтер, клирик и вор. Осталось найти мага. -- произнес я, поправляя лямку рюкзака. -- Вообще-то мы сейчас все питаться пойдем, время-то обеденное. 

Действительно, со всех сторон лагеря к столовой стекались пионеры. А вон и Ольга шествует во главе младшего отряда. 

-- Семен, Славя, вы где были? -- попытка продемонстрировать власть не заставила себя долго ждать. -- И зачем тебе удочки?

-- Обороноспособность лагеря укрепляли. На удочки мы наловим рыбы, а на рыбу выманим и изловим медведя.

-- Опять язвишь? -- обиделась Ольга.

-- Да вы что, и в мыслях не было.

Обед, в общем-то, оказался неплох: вполне наваристый борщ, на этот раз, без лягушек, картофельное пюре с курицей на второе, и стакан компота из сухофруктов. Хотя, летом-то можно было и из свежих фруктов сварить.

-- Всем привет, -- раздался знакомый голос, -- Семен, ты чего к нам сегодня не пришел? 

На свободный стул за нашим столиком приземлился Электроник. 

-- Привет, -- кивнула Славя. Ульянка почему-то промолчала.

-- Здорово. Да дела были, слышал же на линейке, что Ольга говорила.

-- А, то есть тут правда медведь околачивается? -- Округлил глаза парень.

-- Давай уж в клубе проведем совещание. -- Я многозначительно кивнул в сторону Ульянки. Та не замедлила подать голос:

-- А вот и нет! Сейчас же рассказывай! А то я Ольге Дмитриевне кое-что расскажу!

-- Тебе интересно послушать про организацию сигнализации по лагерю? Ну давай, принимай участие: предлагаю какой-нибудь звонок, срабатывающий на открытие ворот собрать. Собственно, тут мнение Шурика интересно, но его нет. Мысли какие-то есть, Ульяночка?

-- Тьфу, скукотища! Вы там все этом клубе зануды, -- надулась девчонка. -- Ты же на рыбалку со Славей собирался. 

-- Вообще-то, мы вечером собирались. Делу время, а потехе -- час. 

Обед подошел к концу. Ульянка, похоже, убежала строить козни и приставать к другим пионерам, а мы втроем отправились в <<Клуб Кибернетиков>>. Сыроежкин нес в руках очередной бумажный пакет с обедом для Шурика.

\paragraph{}<++>


Собрание в клубе шло полным ходом. Шурик задумчиво макал в стакан с кефиром булку, Электроник крутил в руке молоток, а Славя дергала свою косу.

-- Значит, это не медведь? -- Серега подбросил молоток и перехватил его в воздухе.  

-- И зачем ты вообще панику поднял, а Семен? -- Присоединился Шурик.

Чтож, придется держать ответ:

-- Вообще, я не знаю, что это за тварь. Но она однозначно враждебна: напала на меня в тумане. Как минимум, зачаточно разумна: пользовалась костяной, что ли, пикой. Живучая. Я ее хорошенько тогда отделал.

-- Семен, а может это другое существо? -- посмотрела на меня Славя.

-- Может быть, -- согласился я, -- но от того не легче. А если тварей несколько, то вообще швах. Так что по поводу сигнализации, Шурик? Неплохо бы и периметр как-то укрепить.

-- Я подумаю, -- откликнулся парень.

-- По-моему, Семен, тебе не терпится куда-нибудь выплеснуть агрессию, -- Славя отняла у Электроника молоток и положила на стол. 

-- А ты хочешь, чтобы люди начали пропадать?  Или, может, ты из тех болезных, что вопят на каждом углу о правах животных?

Внезапно на улице раздается истошный девичий визг. Пионеры вскочили, и я, подхватив скатку, также ринулся на выход. 

Визг повторился. Кажется, откуда-то со стороны домиков. 

-- За музклубом, -- крикнул Шурик.

Я судорожно, на бегу, пытался развязать шнурки, скрепляющие скатку. 

-- А, к черту! -- нож рассек непослушные узлы и удочки посыпались на землю. Лакированное дерево и вороненая сталь внушали мысль, что все будет хорошо. Только бы успеть.

-- Всем назад! -- снять с предохранителя, палец вдоль скобы, ускориться\ldots Кажется, ребята заметили ружье и поотстали. Впереди, за деревьями, виднеются белые пионерские рубашки, и яркие голубые волосы. 

Лена с ножом в руке смотрит куда-то в лес, а за ее спиной прячется Мику.

-- Какого черта? Что случилось? -- взять под прицел подозрительный участок. 

-- Что-то за окном возилось. -- Заговорила Лена. -- Мику вышла посмотреть. Закричала. Я вышла. -- В девушке не осталось и капли застенчивости. -- Какой-то уродский человечек в лес убежал.

-- Оно кого-то жрало, -- рыдала Мику, -- кого-то пушистого!

Тем временем подбежали ребята. 

-- Семен! -- окликнула меня Славя, -- те же самые следы. 

Действительно, на песчаной почве явственно отпечатался знакомый след. Узкая ножка с длинными пальцами. 

-- Оно тебя не тронуло? -- Я поглядел на Мику.

-- Нет, -- сквозь слезы ответила девушка, -- страшное очень.

-- Угу, мерзость. -- Подтвердила Лена. -- Это ваш со Славей медведь?

-- Какая догадливая. -- Щелкнул предохранителем и положил ружье на сгиб левой руки.  -- Кто поохотиться желает?

-- Вообще, я часто с дедушкой в лес ходила, -- вызвалась Славя, -- похоже, это правда гадкое создание, Семен. 

-- Так. Коллеги, ищете вожатую, скажите ей, что медведь зашел в лагерь. Пусть, что ли, все в столовой собираются и запрутся. Лена, забирай Мику и иди с ними. 

\paragraph{}<++>


Лес, по большей части, хвойный. Впрочем, что еще будет расти на песчаной почве? Разве что акация. След читался хорошо. Даже такой немудреный следопыт как я прочтет его легко. 

-- Смотри, кажется кровь. -- Обратила мое внимание на важную деталь Славя. Действительно, на земле виднелись бурые капли. 

-- Угу. -- А девушка молодец. Под стволом не крутится, директрису не перекрывает, идет тихо. Гораздо тише, чем я.

Лес становился все гуще. Расстояние между следами увеличивалось. Похоже, что тварь перешла на бег. Кажется, что весь лес затаился. Что-то здесь было неправильное. Чувствовалась тревога. Перед глазами мелькнуло мрачное болото с болезненными карликовыми деревьями. В нос ударило болотным смрадом. Остановиться, сморгнуть. И снова вокруг хвойный лес с хилым подлеском.

-- Ты видела? 

-- Что?

-- Черт. Глюки, похоже.

Идем дальше. Под ногами похрустывают ветки. Удивительно тихо. Ни птиц, ни насекомых: лес, как будто, вымер или затаился. Эта тишина заставляет нервничать и тыкать стволами  в плохо просматривающиеся затененные участки. 

-- Слышишь? -- прошептала Славя.

-- Что? -- ответил не оборачиваясь.

-- Тишину. Странно\ldots

Идем вперед. След я потерял и надеюсь только на чутье лесной девушки. Вся эта охота мне нравится с каждой минутой все меньше и меньше. Что я знаю об этом странном создании? То, что оно обитало в брошенном коровнике и не побоялось напасть на противника крупнее себя. Что пользовалось примитивным оружием. То, что страшное и мерзкое. И то, что раньше я таких не видел. Или все-таки видел? 

Память подсовывает какие-то совершенно безумные картины: провал подземелья, откуда смотрят десятки голодных глаз, потом детская площадка, где, среди качелей и скамеек, сидит подобная же тварь и с аппетитом жрет чью-то кисть\ldots Меня передернуло.

-- Стой! -- воскликнула Славя. -- Смотри.

Впереди, шагах в десяти перед нами, виднелся какой-то чужеродный предмет. Что-то маленькое и рыжее. На воткнутую в землю костяную пику была насажена окровавленная голова белки.

\paragraph{}<++>

В памяти всплывали неясные образы: мрачный и сырой коридор, похожий на какие-то катакомбы, тьму рассекает яркий луч фонаря, под ногами хрустят маленькие косточки; жилой дом -- классическая пятиэтажка, каким-то чудом оказавшаяся посреди чистого поля. Картинка резко меняется: теперь это брошенная гостиница или общежитие, повсюду слой пыли, на котором явственно видны человеческие следы, перемежающиеся  отпечатками узких ступней с длинными пальцами. 

-- Дерьмо, -- выдохнул я. -- Вот дерьмо!

Указательный палец лег на нижний крючок и потянул. Тишину разорвал грохот выстрела и в грудь ощутимо лягнуло. Сноп крупной дроби буквально смел мерзкий памятник, или что там это существо хотело изобразить.

-- Ты зачем стрелял? -- поразилась Славя.

-- Мерзость, гнусь поганая, -- пробормотал я, переломив ружье и выковыривая стрелянную гильзу. -- Ненавижу! 

Мне показалось, или что-то пискнуло, сорвавшись с места? Свежий патрон занял свое место в стволе, и ружье хищно лязгнуло. 

Боковое зрение уловило какое-то неясное движение: вскинулся, стволы ищут цель, палец  готов выжать спуск. Но цели не видно, а правило <<не вижу -- не стреляю>> написано кровью.

-- За мной. -- Я сорвался с места. Трудно сказать, видел ли я, слышал ли то, что должно стать добычей. Скорее чувствовал. Ноздри раздувались, ловя запахи хвойного леса, в ушах стучало, а пальцы сжимались на шейке приклада. Пришел охотничий азарт, а с ним и осознание, что уйти этому несуразному человечку не удастся. Догоню и разорву на части. А все их норы выжгу дотла. Норы?

Кажется, я перешел на бег. Из горла вырывается то ли хрип, то ли рык. 

-- Семен! Стой! Семен! -- слышится девичий крик откуда-то сзади. 

Замедляю бег, потом перехожу на шаг. Ярость и азарт куда-то уходят, и я снова вижу  вполне себе обычный лес. Такой же тихий, но угрозы не чувствуется. Точнее, вообще ничего странного не чувствуется. Вдруг, где-то далеко послышалось птичье пение.

За спиной слышны шаги. Оборачиваюсь и вижу Славю.

-- Ты что? С ума сошел? Куда сорвался? -- видно, что девушка ощутимо запыхалась. -- След мы давно потеряли, здесь земля тверже.

-- Не знаю, -- почесал я подбородок. -- Как будто чувствовал его. Потом тебя услышал и все ушло.

-- У тебя такое лицо было, когда ты эту голову увидел\ldots

-- Что-то знакомое, -- произнес я, стирая заливающий глаза пот, -- не мог же видеть, но вспоминаю. 

-- Семен, пошли лучше назад. -- Славя была явно напугана. То ли возможной встречей с неведомой зверушкой, то ли моим странным поведением. Впрочем, последнее мне и самому не очень-то нравилось.

-- Угу, пойдем.

Трудно сказать, по каким ориентирам девушка находила путь, но шла она уверенно. Я же брел, поотстав шагов на пять. Прислушивался и принюхивался, пытаясь уловить какие-либо признаки опасности. Но тщетно. Лес словно ожил: невдалеке застрекотал кузнечик, а среди крон деревьев иногда пролетали птицы, потревоженные нашим присутствием. Смеркалось.

-- Славя! -- позвал я. -- Лес изменился.

-- Ага. Тоже заметил? -- отозвалась спутница.

\paragraph{}<++>


Нас со Славей встретили возле входа в лагерь. Трое бодрых пионеров. Шурик, Электроник и Витек. В руках первых двух молотки, а последний раздобыл, видимо -- в столовой, огромный кухонный нож. 

-- Мы тут обход делали. -- сообщил главный кибернетик, -- Подстрелили?

-- Увы, -- поморщился я, -- ушла паскуда.

-- Так и что делать-то будем? -- Серега рубанул воздух молотком и перехватил его поудобнее, -- тут некоторые думают, что это все игрушки. 

-- А вожатая наша вообще истерику закатила, -- вступил в разговор рыжий. -- Виола ее валерьянкой отпаивает или чем-то покрепче\ldots

-- А где она сейчас? Надеюсь, хоть по лагерю народ не шляется? -- неприспособленность Ольги к нештатным ситуациям мне категорически не нравилась. На улице, вроде, кроме нас пятерых никого не было, да и тихо довольно-таки. 

-- В столовой все сейчас, -- Шурик поглядел на меня и поправил очки. -- Пойдемте, надо решать что-то\ldots

В столовой собрался весь лагерь. Ребята из старшего отряда сгрудились вокруг сдвинутых столов и что-то обсуждали, детвора из младшего, видимо, не могла усидеть на месте и носилась между рядами. В общем, шум стоял дикий.

Вожатая выглядела, мягко говоря, не очень. Лицо бледное, глаза красные. Плакала, что ли? Рядом, с каменным лицом, сидит наша медсестра. Хм, за руку что ли Ольгу держит? Ну да ладно.

Славя решительно направилась к столу нашего <<начальства>>. 

-- Ольга Дмитриевна, вы как? -- пионерка была явно обеспокоена видом вожатой. -- Ольга Дмитриевна?!

Наблюдать очередную истерику или еще какие-нибудь чудеса женского внутреннего мира мне совершенно не хотелось, и я направился в сторону кухни, где должна была находиться еда.

-- Эй, Семен, -- догнал меня Электроник, -- а ты так и будешь теперь с ружьем таскаться? Как-то страшновато народу.

-- Кхм, -- я поправил разомкнутую двустволку, покоящуюся стволами на плече, -- а чего? Патронники пустые вон. Да, так и буду теперь ходить. Может, считаешь, что если какая-нибудь пакость вылезет, стоит попросить ее подождать, пока я за оружием сбегаю?

-- Ну не, -- стушевался парень, -- но все равно, не привычно как-то. 

-- Мне б пожрать тут. Пока по лесу шарились, проголодался как волк. 

-- А, ну так это, сейчас. Теть Клава, -- позвал пионер, и из недр кухни к раздаточной стойке подошла весьма упитанная повариха в характерном белом фартуке и колпаке.

-- Ну, охотничек. Поднос бери, давай.

Желудок я набивал в спокойной обстановке. Или, может быть, был слишком увлечен едой, чтобы реагировать на беготню детворы или разговоры и крики пионеров. Впрочем, с расспросами никто не приставал, и это радовало. Щека подпиралась ладонью, локоть уперт в столешницу. Столовая начала смазываться\ldots 

\paragraph{}<++>

-- Эй, не спи, замерзнешь! -- кто-то ощутимо хлопнул меня по плечу. Да так, что лежащая на коленях двустволка, чуть было не свалилась на пол. 

Отвратительный скрип двигаемого по полу стула, еще один тычок. Алиса, ну кто же еще.

-- Ну просыпайся давай, -- не перестает тормошить меня девушка. -- Что там было-то? Ну!

-- Блин, да отстань, уже -- язык еле ворочается, желание одно: завалиться лицом на стол и отключиться. Окружающая обстановка вновь начинает смазываться, 

Похоже, за стол подсел еще кто-то. Слышны девичьи голоса. Пионерки что-то явно обсуждают. Я, как будто, нахожусь в двух местах: в этой столовой и еще где-то. 

Пятиэтажка в чистом поле. Та самая. Где-то я ее уже видел. Тротуарчик, ведущий к подъезду, как будто, отрезан: асфальт обрывается и начинается голая земля. Следов запустения, однако, не заметно: обычный такой дом-хрущевка, каких полно в любом городе. Окна целые и, в большинстве своем, забраны занавесками.  Эдакий обыденный, нормальный дом, только вот в ненормальном месте. 

Слышны голоса, кто-то опять трясет за плечо. 

-- Да оставь ты его в покое, пусть спит. -- Славя, похоже.

-- Да бензорезом сноси, мы пока эту проверяем. -- уже другой, мужской голос.

Я в квартире. Она обычна: перед входной дверью подставка для обуви, вешалка для одежды. Куртки, пальто, рабочие штаны. Какие-то картины на стене коридора. Пейзажи.

Комната. Советская стенка с фарфором, книгами и тому подобным хламом, резко контрастирует с немаленьким современным ЖК телевизором. Обшарпанные обои, идиотского вида оранжевая люстра. Старый диван, на котором лежит кто-то, отвернувшийся лицом к спинке и укрывшийся по самую голову клетчатым одеялом. Рядом с диваном -- журнальный столик, на котором стоит лаптоп, судя по всему, давно отключившийся.

-- Проверь. 

Подхожу к дивану, откидываю одеяло, рывком поворачиваю к себе лежащего. Мужчина лет сорока. Кажется, что он спит. Но только кажется: нет ни дыхания, ни пульса.

-- Мертв. Идем дальше.


\paragraph{}<++>

Будит меня чувство тревоги. Что-то словно ползет по моим ногам. Ружье! Левая рука успевает схватить ускользающее оружие. Скатываюсь со стула, срываю  дистанцию. Не видя окружающей обстановки, во что-то врезаюсь. Звук бьющейся посуды, женский вскрик. Нож вылетает из-за пояса и застывает в правой руке, отгораживая от опасности\ldots

-- Семен! Ты что! -- пелена перед глазами рассеивается, исчезают остатки сна, и перед собой я вижу, сидящую на полу, Ольгу с широко распахнутыми глазами, стоящую, чуть поодаль, Виолу с выражением ужаса на красивом лице и повариху, взирающую на все происходящее безобразие, уперев руки в бока.

Похоже, что все <<взрослое>> население лагеря собралось в полном составе. А вот пионеры\ldots Столовая опустела, и только передвинутые столы и стоящая кое-где грязная посуда, напоминают о былом столпотворении. 

-- Какого черта?! -- Поразмыслив пару секунд и оценив ситуацию, убираю нож за пояс. -- Где все?

Вожатая подымается, потирая, видимо, ушибленную при падении, руку. 

-- Ружье отдай, -- Ольга подымается и пытается напустить на лицо властное и суровое выражение. Хорошая мина при плохой игре? Ну-ну. 

-- Да с хрена ли? -- Сохранять спокойствие удается все труднее. Так и хочется залепить зарвавшейся девице хорошую оплеуху. -- Где все, я спрашиваю? Отвечать!

От последнего выкрика Ольга подпрыгивает и прижимает к себе руки. 

-- Семен, кажется, ты заигрался, -- более-менее спокойно произносит медсестра. -- Мы всех распустили по домикам. А ты ведешь себя все более странно.

-- Да вы что, совсем тут долбанулись? -- от такого поворота событий, удивлению моему нет предела. -- А медведь, а туман этот, а связь, а директор со сторожем? 

-- А с этим предоставь взрослым разбираться, -- подает голос повариха, -- неча тут с ружьем бегать!

-- Какие же вы курицы\ldots -- Желания препираться с этими достойными дамами нет ни малейшего. Спиной вперед я продвигаюсь к выходу из столовой. Ольга делает несколько шагов в мою сторону. -- Назад, дура!

Через мгновение я оказываюсь на улице. Однако. Вот чего-чего, а подобной неразумности я никак не ожидал. Надо, пожалуй, переговорить с ребятами\ldots 

\paragraph{}<++>

В кружке <<кибернетиков>>, судя по всему, было многолюдно. Горел свет и доносились голоса пионеров. Шло оживленное обсуждение на повышенных тонах. 

-- А я говорю, нужно передатчик собирать и связаться с кем-нибудь -- голос Шурика. 

-- Да погоди ты, надо сигнализацию доделать, -- это уже явно Электроник, -- или пусть всякая гадость в лагерь лезет?

-- Ты сейчас, на ночь глядя этим заниматься хочешь? -- уже девичий голосок. 

-- Нет, утра подождем, пусть ночью кого-нибудь утащат!

Ну, пора бы и принять участие в совещании. 

-- Верно Серега говорит, -- заявляю с порога. В клубе собрался почти весь старший отряд. Даже Ульянка здесь. -- А где Славя?

-- Вместе с Женей младший отряд в библиотеке делом занимают, -- вступает сидящая на столе Алиса. -- Выспался? 

-- Да уж, так выспался, что чуть ружье не уперли. -- поправляю висящую на сгибе локтя разомкнутую двустволку. -- Так что там с сигнализацией?

-- Релюху на ворота установили, кое-где провод натянули и звонки расставили, -- Шурик поднялся и прошел к стоящему в углу приемнику. -- Но я думаю, что нужно все-таки с кем-нибудь связь наладить. Или хотя бы то сообщение разобрать попытаться. 

-- А нечисть та? -- Электронику явно не дают покоя произошедшие днем события. -- Лен, ну хоть ты скажи ему!

Скромно сидящая в уголке девушка отрывается от созерцания столешницы и пожимает плечами. Кажется, ей явно не комфортно при таком скоплении народа.

-- Ну вот смотри: мы здесь, а детворой Славя с Женей занимаются. -- Шурику явно не хотелось покидать кружок на ночь глядя. -- Ну, что тут может случиться? 

-- Например, вожатую нашу дурную сожрут. -- принимаю сторону Электроника, -- Медичка, кстати, тоже умом не блещет. О поварихе уж не говорю.

-- Ну что, погнали тогда? -- Витька соскакивает со стола и начинает ходить из угла в угол. Ему явно не терпится чем-нибудь заняться. -- Алис, идешь?

-- Вот. Нас уже трое. Шурик, ты с нами или тут будешь копаться? -- подымаюсь, и загоняю пару патронов в стволы, защелкивая ружье. -- Ставите с Серегой и Витькой сигналку, я прикрываю. 

-- Уговорили. -- Главный <<кибернетик>> вытаскивает из-под стола сумку, видимо, с инструментом. Сыроежкин же тащит из подсобки бухту кабеля. 

-- Кстати, а смысл тогда трещетки ставить? -- белобрысый пионер явно вошел во вкус. -- Может просто кабели натянем да ток пустим? 

-- А ты где ты изоляторы нормальные найдешь, а? -- не соглашается Сашка.

-- Ладно, -- подытоживаю я, -- остальные, тогда сидите здесь, эфир слушайте.

-- Я с вами пойду, -- решает присоединиться к инженерной бригаде Алиса.

\paragraph{}<++>

Как это ни странно, но работа идет спокойно и буднично. Безоблачное небо, в котором ярко светят звезды, стрекот ночных насекомых, свет фонарей в лагере, вечерняя прохлада\ldots Одно слово -- идиллия. 

Но поверни голову в сторону высоковольтки, и взгляду предстает стена мрака, за которой теряются даже звезды. И идиллия рушится: лес кажется жутковатым, тени принимают устрашающие формы и появляется ощущение недоброго взгляда. 

Работали, в основном, молча. Когда не хватало света, Алиса подсвечивала моим фонарем Шурику с Электроником, Витька рубил колья для протяжки проволоки и забивал их в землю. 

Проще всего было мне: бесконечно можно смотреть на огонь, воду и на то, как другие работают.

Размышления мои прервал шорох: со стороны лагеря кто-то явно двигался в нашу сторону. 

Тихо, без щелчка сдвинуть предохранитель вперед, сместиться чуть в сторону от вероятного места выхода бредущих в темноте. Слышен женский голос: приглушенная ругань, шарканье ног, хруст веток. Еще секунда и из тьмы показывается белая блузка. Ну, и все остальное.

-- Опа! Ольга, кхм, Дмитриевна. Зря вы по ночам в лесу шляетесь. 

-- Сергей, Александр?! Вы-то, что тут делаете? -- вожатая явно удивлена открывшейся ее взору картине: бухта кабеля, набитые в землю вдоль забора колышки, пионеры за работой. -- Семен в <<зарницу>> заигрался и вы туда же?

о- Семен дело говорит, -- Электроник оторвался от работы и строго посмотрел на вожатую. -- Утащат вас так, Ольга Дмитриевна, в лес и съедят.

-- Да кто? Вы ведь это все выдумали про медведя! -- сорвалась на крик Ольга. Похоже, что ей банально не хотелось верить, что все происходящее -- реальность.

-- Ага, -- буркнул Шурик. -- И туман вон тот мы выдумали. 

-- Тьфу. Вот приедет милиция, всех на учет поставит. -- Вожатая повернулась и собралась было двинуться назад в лагерь. 

-- Стоять! -- рявкнул я. Девицу передернуло. -- Сожрут, что неясно? 

-- Семен, -- произнесла та, обернувшись, -- ты мне ружьем что ли угрожаешь? -- Вид у нее был, надо сказать, весьма бледный.

-- Ни в коем разе. О вас же пекусь. Как закончим, провожу до домика -- запретесь.

Запас истерик и возражений, как будто, сошел на нет. Мы неспешно продвигались вдоль ограды, вбивая колья и протягивая по ним кабели. Вожатая послушно плелась рядом. 

-- Глянь-ка, -- подошла ко мне Алиса, -- она там плачет что ли?

-- Да черт ее знает? Пойди, успокой.

-- А что я? Да ну ее\ldots

Не сказать, что работа прямо спорилась. Затянули мы всего-навсего эдак треть периметра лагеря,  но вымотались жутко. Даже меня давила усталость, хотя вся работа моя -- всматриваться в лес, да прислушиваться. 

Ольга присела, прислонившись спиной к стволу дерева и, как будто, задремала. 

-- Давайте, поднимайтесь. -- я аккуратно потрепал ее по плечу. -- Мы тут закончили.

-- Не трогай меня! -- Вожатая дернулась и подскочила. -- Ты-ты-ты\ldots

-- Что я? Пойдемте.

Забрезжил рассвет. Вся наша инженерная команда просто валилась с ног. Ольгу же совсем развезло. Видимо, действительно не валерьянкой ее Виола накануне отпаивала, однозначно не валерьянкой.

-- Заканчиваем, в общем. -- произнес, зевая, Шурик. -- С ног валюсь. 

-- Угу, ага. -- соглашались остальные. Да уж, идея с сигнализацией только казалась простой и изящной.

-- Давай, Ольга, опирайся на меня. -- я пытался придать вожатой вертикальное положение одной рукой, другой удерживая ружье. --  Вот уж угораздило\ldots 

Шурик с Электроником отправились провожать Алису, а мы с Витькой тащили наше неразумное <<начальство>>.

\chapter{День четвертый}
-- Очнись! Давай, братишка! -- Сквозь кровавую пелену я чувствую, как меня куда-то тащат. Рывок со стороны спины, больно стукаюсь обо что-то затылком. Вяло перебираю ногами, пытаюсь двигаться. Падаю. Кто-то продолжает меня волочить. Во рту острый металлический привкус.  Жутко воняет кровью и разлившейся горючкой. Что-то гадко шипит. 

-- Сейчас накроет! Давай, шевелись! -- Шипение  переходит в свист, потом в визг, просто истошный визг. Из ушей идет кровь или это только кажется? 

Что-то сотрясает землю. Подбрасывает и начинает крутить во все стороны. Опять визг и вой. Удивительно, почему я до сих пор не оглох? Кто-то кричит. Яркая вспышка, а потом все поглощает тьма.  

Сознание неспешно возвращается. Я лежу. Явно не на кровати: на чем-то жестком, в очень неудобной позе.  По-моему, болит каждая клеточка. Пытаюсь приподняться, но без сил падаю. 

-- Живой! Живой! -- раздается радостный голос. -- Погоди, сейчас. Давай, вот так\ldots

Кто-то помогает мне принять полусидячее положение. От этого сразу начинает мутить, подкатывает комок к горлу и меня рвет. 

Глаза открывать больно, да и застилает их какая-то белесая пелена. Или это туман? Кто-то опять помогает присесть. Передо мной мужик в грязном и окровавленном камуфляже незнакомой расцветки. 

-- Где я? -- язык еле ворочается, и вместо вопроса звучит какой-то хрип. Мужик, однако, меня понимает. 

-- Да хер знает. -- незнакомец плюхается рядом, вытягивает из-под лямки снаряжения шланг и поливает из него жидкостью платок. -- На, блевотину хоть утри.

-- А лагерь, пионеры? -- вытеревшись, бросаю платок на землю. Поверхность покрыта белой субстанцией. То ли снег, то ли пепел -- непонятно.

-- Сема, ты чо? Какие пионеры, какой лагерь?! -- человек кашляет, прикладывается губами к шлангу,  полощет рот и сплевывает. -- Возврат не сработал. Бардак наш в клочья, Петьку на куски порезало, Сашка пропал.

-- Какой возврат, какой бардак? А ты кто, вообще? -- вся последняя вереница событий никак не укладывается в голове.

-- Да зашибись. -- Мужик в сердцах бьет кулаком по земле. -- Память отшибло? И ведь хрен знает, где нас выкинуло. Даже если еще группу пришлют, где они нас искать теперь будут\ldots -- Последняя фраза звучит обречено и адресована явно не мне.  

\paragraph{}<++>

Дурнота никак не проходит: голова кружится и все вокруг плывет. Переворачиваюсь на живот, затем встаю на четвереньки, пытаясь подняться на ноги. Опять тошнота, рвет желчью.

-- Да лежи ты! -- товарищ по несчастью переворачивает меня набок. -- Не дергайся. Ух ты! 

Мужик хмыкает, что-то дергает у меня в районе груди, раздается щелчок и взгляду предстает черный пластиковый магазин. Точнее, его половинка: он словно рассечен наискось бритвой. 

-- А вот Петька-то чуть ближе к силовому сидел\ldots -- опять возится и что-то отстегивает. -- А вот автомат, вроде, целый. 

Кажется, на какое-то время отключаюсь. Мне тепло и хорошо. Как будто, слышится пение птиц. Покой и умиротворение. 

-- Не вырубайся! Эй! -- меня яростно тормошат. -- Сейчас, погоди. Да давай, ты, отстегивайся! 

Укол в грудь. Как же больно! Вскакиваю. Сначала на четвереньки, потом поднимаюсь на ноги. Делаю пару шагов, ноги подгибаются. Падаю на колени. В висках стучит, хочется бежать, но измученное тело не позволяет.

-- Что?! -- все, что удается из себя выдавить.

-- Стимулятор вколол. Чуть не отошел ты. 

Шок начинает проходить и, наконец, удается осмотреться. Видимость отвратная: очень густой туман. Он неоднороден и, как будто, движется. Сквозь разреженные участки можно видеть чахлые низкорослые деревья. 

-- Вот срань. -- Тру лицо рукой и сразу же натыкаюсь на бороду. А поначалу-то и не заметил. Немудрено: не до того было. Борода в какой-то гадости. Ну да, понятно. Начинаю судорожно ее очищать. 

-- О, полегчало, гляжу, -- произносит спутник. -- Идти сможешь? 

Прервав чистку, пытаюсь подняться и делаю пару шагов, после чего опять опускаюсь на колени.

-- Башка кружится, мутит адски. Куда идти-то?

-- Я тут сканером кое-что поймал. Бубнит  про выживших и сбор -- больше не разобрать. А вот это, -- Из динамика рации раздается, знакомый каждому, сигнал: три коротких, три длинных, три коротких, -- Это где-то рядом. Пеленгатора нет. Придется по интенсивности сигнала ориентироваться. 

Окидываю себя взглядом и, наконец-то, понимаю, что на мне точно такой же камуфляж, как на незнакомце. Только без навесного.

-- Броник я с тебя стянул. -- По-своему понимает спутник. -- Надо бы надеть, мало ли\ldots

-- Я так-то идти не могу, какая нахрен броня? -- Вновь пытаюсь встать. -- Мы военные, что ли?

Мужик делает пару шагов в сторону, после чего подтаскивает ко мне ворох какого-то барахла.

-- Вот твое добро. Точняк, память отшибло. Ты меня-то хоть помнишь?

-- Не. -- Остается только пожать плечами. -- Обрывки какие-то: пятиэтажка эта с трупами, болото, город заброшенный\ldots

-- Дом был, да, -- кивает боец. -- А вот город? Какой, к чертям, город? Ладно. Игорь я. Потом вспомнишь. 

\paragraph{}<++>

Идти тяжко. Голова не перестает кружится, да и тошнота никак не проходит. Вес снаряжения так и тянет к земле: хочется сбросить проклятущий броник вместе со всем навесным, выбросить автомат и шагать налегке. Или, скорее, ползти. 

Игорь матерится сквозь зубы, когда я в очередной раз оступаюсь и висну на его плече. Периодически останавливаемся и проверяем рацию. Вроде бы, затухания сигнала нет. Если не приближаемся, то, хотя бы, не удаляемся.

Туман то редеет, то сгущается. Ничего нового: вся та же равнина с торчащими тут и там корявыми деревьями. 

Тишину нарушает лишь звук шагов, побрякивание оружия да редкая приглушенная ругань. 

-- Жрать-то как хочется, -- бормочет спутник. -- Ладно, еще час идем, потом на привале пороемся в сухарке. 

Вяло переставляю ноги. Еще и еще. Сколько можно?

-- Слышь, Игорь, -- окликаю товарища, -- а может мы того уже? 

-- В смысле? -- не понимает\ldots

-- Ну, сдохли мы, -- останавливаюсь, и делаю свободной рукой неопределенный жест. -- А это все\ldots Ну ты понял.

-- Да с хера ли? -- Игорю такая мысль явно не нравится. -- Пошли, что встал?

Идем. И нет этому пути ни конца, ни края. Хорошо, что все мое снаряжение, за вычетом пострадавшего магазина, в целости. Тот же гидратор с содержимым. А я так хотел все это бросить. 

Возникает неприятное чувство. Еще одно, в дополнение ко всем остальным: боли, тошноте и усталости. Ощущение пристального взгляда. Тревожно.

-- Стой. -- Перестаю опираться на спутника. Пытаюсь снять с предохранителя автомат и не нахожу привычного переводчика огня. И рукоятка затвора явно не на месте. Оружие только казалось привычным АК. -- Вот дерьмо!

-- Где? -- Слышен тихий щелчок, напарник берет оружие на изготовку. 

-- Недалеко. -- Вглядываюсь в туман, но ничего необычного не заметно: вся та же белая мгла, местами завихряющаяся и собирающаяся в плотные облака. Ощупываю оружие и нахожу переводчик огня на левой стороне ствольной коробки. Прямо над большим пальцем. Рукоятка взвода оказывается на трубке газоотвода. Удивленно хмыкаю. -- Хрена себе. Прям ХеКа \footnote{Heckler \& Koch}. 

-- Видишь кого? 

-- Нет. -- Ощущение враждебного присутствия уходит, вновь накатывает тошнота. -- Ушел. 

-- Ну ты экстрасенс, блин. -- Вновь тихо щелкает предохранитель. -- Человек, зверье? 

-- Нечисть, кажись. -- Ощущения были очень похожи на те, что я испытывал в лесу возле лагеря. Или не испытывал? Был ли он вообще, этот лагерь? Или, может быть, я сплю на кровати в домике после бессонной ночи, а все это: туман, гибель боевой машины, спутник мой -- сон? 

-- Пошли давай, что задумался? -- прерывает мои размышления хриплый голос. -- Калаш на пред поставь. 

Все-таки калаш. Нащупываю переключатель и возвращаю его в среднее положение.

-- Слышь, а ты мне точно не снишься? -- осматриваю свое оружие, потом перевожу взгляд на спутника.

-- Завязывай. -- Игорь подставляет плечо. -- Опирайся, двинули.

-- Это ж не калаш нихрена. -- не могу успокоиться я. -- Предохранитель, затвор, приклад -- все другое.  

-- Мля, обычный девяносто четвертый\footnote{Вымышленная модель. Вольная интерпретация хотелок и доработок, о которых много лет твердят Ижмашу: разобщенная рукоятка взвода на жестко закрепленной газоотводной трубке с вивером, ствольная коробка без выреза, предохранитель с переводчиком огня под большой палец, откидная влево спусковая скоба и складывающийся вправо приклад с регулируемой щекой.}. Все, вперед. 

\paragraph{}<++>

Идти тяжело даже опираясь на плечо спутника. Перед глазами плывут клубы тумана, мерещится движение, иной раз взгляд фиксирует все те же кособокие деревья, тянущие свои ветки в белую мглу. Что-то привлекает внимание моего спутника и мы останавливаемся. Сползаю на землю, очень хочется лечь и уснуть\ldots

-- Смотри. -- Игорь что-то ковыряет на земле, после чего демонстрирует довольно крупный камень. -- Бетон. Глянь, тут и кирпич битый. Пошли. 

Идти все тяжелее: дорога портится. Камни, куски каких-то балок, кажется, попадались даже битые стеклянные бутылки. Несколько раз ноги путались в мотках проволоки и сохранить равновесие стоило немалых усилий. Свалка? Очень на то похоже. 

-- Опа, оградка! -- Прямо перед нами полуразвалившийся бетонный забор, часть секций которого попросту повалилась наземь. -- Похоже, недалеко осталось. 

Действительно, впереди угадываются очертания какого-то крупного строения. Даже отсюда понятно, что оно необитаемо. По-крайней мере, людей здесь нет точно. 

Недостроенное панельное здание высотой в семь этажей. Бетон без штукатурки, не остекленные оконные проемы\ldots Почему-то этот заброшенный недострой вызывает стойкую ассоциацию с больницей. 

-- Ну, хоть крыша над головой будет, -- Игорь кивает в сторону, где, видимо, должен быть вход.  -- Пошли посмотрим. 

-- Стой! -- опять возникает стойкое чувство тревоги. Очень похожее на то, что я испытывал возле заброшенного коровника. -- Похоже, там твари.

-- Хрена себе, чутье. -- Щелчок предохранителя. -- А жрать их можно?

-- Не пробовал. -- Стоять самостоятельно тяжело: жуткое головокружение, к горлу сразу же подкатывает комок. 

-- Так, -- Боец перекидывает автомат к левому плечу, -- Пистолет возьми и обопрись. Спину держи. Да на бедре кобура, елки-палки. 

Действительно. Легкий пистолет на пластиковой рамке, практически без выступающих частей. Из органов управления только кнопка сброса магазина и утопленный в пластик рычаг затворной задержки. Слегка сдвигаю затвор: да, патрон уже находился в патроннике. 

Здание однозначно заброшено. Хотя, люди здесь явно бывали. Об этом свидетельствует огромная надпись на стене: "Вася ЛОХ". В углах валяются запыленные пустые бутылки и тому подобный хлам.

-- Мда. -- Игорь сплевывает, -- Детишки тут любили погулять. 

-- Значит, жилье близко. 

-- А не факт. -- Делаем еще несколько шагов по коридору, осматриваем провалы комнат. Дальше идти совершенно не хочется. Напарник, судя по всему, тоже не горит желанием. -- Ладно, давай вот в этой комнатке отдохнем, пожрем, а потом дорогу искать будем. 

-- Слышь, тебе это дерьмовый фильм ужасов не напоминает? -- говорю, садясь на холодный бетонный пол, прямо напротив дверного проема, ведущего в коридор.

-- А?  Погодь, <<ревун>> настрою, -- боец что-то достает из рюкзака и выходит из комнаты. --  Напротив входа не сиди. Какой еще фильм?

-- Ну, смотри: туман этот, заброшенная больница\ldots

-- А с чего ты взял, что это больница? -- Доносится из коридора. 

-- Ну, показалось так. А сейчас ты решишь пойти посмотреть, что там дальше, и тебя утащат в подземелье. 

-- Да никуда меня не утащат. -- В проеме появляется человеческая фигура. -- Все, <<ревун>> поставил. И разделяться мы тоже не будем. Сейчас пожрать что-нибудь сообразим\ldots Во, лови. Протеиновый батончик. 

-- Да, с голоду пока не помрем.

-- А потом тварь подстрелим и шашлыков нажарим!

Честно говоря, есть мне совсем не хочется, но на всякий случай распечатываю батончик, откусываю, машинально пережевываю и с трудом глотаю. Тут же к горлу подкатывает комок и съеденное просится наружу. Однако, порывы удается сдержать. Через пару минут отпускает. Главное -- не двигаться. Спать, как же хочется спать\ldots

Покой и умиротворение. Тепло и опять кажется, что где-то рядом щебечут птицы. Вот сейчас я открою глаза и вновь окажусь в пионерском лагере. В юном и здоровом теле, в окружении дружелюбных ребят и красивых девушек. Закончим с кибернетиками сигнализацию, Ульянка отмочит какую-нибудь хохму, после чего за ней с воплями будет гоняться Ольга\ldots

Тишину разрывает рев сирены, за которой следует короткая автоматная очередь. Слух мгновенно притупляется. Падаю на левый бок, одновременно выхватывая из кобуры пистолет. Точнее, пытаюсь это сделать: что-то не так с координацией. Оружие за цепляется за снаряжение. Еще одна очередь. Наконец, беру на прицел дверной проем.

-- Чисто! -- раздается знакомый голос. -- Иду назад!

Спиной вперед в комнату входит Игорь. Что-то волочит. 

-- Гляди, кто к нам на огонек пожаловал. -- Тащил он знакомого вида зверушку. В грудине ее виднелись три входных отверстия, а с развороченной спины обильно текла кровища. Те самые тонкие ручки и ножки, здоровенная зубастая пасть и уже остекленевшие глаза в половину морды. 

-- Да уж, знакомы, -- Принимаю сидячее положение и со второй попытки убираю пистолет в кобуру. Вновь начинает мутить, дико кружится голова. 

-- Что это вообще за дрянь? -- напарник с интересом осматривает добычу. -- До чего же мерзкая!

-- Не помню\ldots -- Говорить трудно, язык заплетается, -- В руинах живут\ldots 

-- А какого хрена в отчетах про них ничего не\ldots Эй! Вот черт! Не вырубайся! 

Все плыло перед глазами. Сил не оставалось даже на то, чтобы пошевелить рукой. Звуки растягивались и постепенно превращались в гул. Все поглощал мрак.

\paragraph{}<++>

Жарко. Меня будит солнечный луч, падающий на лицо. Вскакиваю. Не осталось ни малейшего следа головокружения или тошноты. Знакомый домик в пионерском лагере. На соседней кровати дрыхнет Витька. 

Ничего не понимаю. Реально ли это место или все-таки является порождением травмированного мозга? Быть может, я умираю в том заброшенном здании, а лагерь и его население -- попытка уйти от жуткой реальности? 

Провожу руками по лицу: оно принадлежит все тому же семнадцатилетнему мне. 

Это молодое тело, в общем-то, вписывается в концепцию вымышленного мира\ldots 

Ладно. Быстренько одевшись, сунув за пояс нож и подхватив с кровати двустволку, я направляюсь на выход. Надо подышать свежим воздухом да и ополоснуться не помешает. 

Утро выдалось жарким: на небе опять ни облачка. Вдохнув полной грудью чистый воздух, направляюсь в сторону умывальников. 

-- Привет! -- Слышится знакомый девичий голосок. -- Почему один ходишь, сам ведь говорил\ldots

-- Привет, Славя, Лена. -- Киваю я девушкам. -- Мне-то можно: вооружен и очень опасен.

-- Идешь на спортплощадку поразмяться, а Семен? -- Обе пионерки были уже в спортивных костюмах.

-- Да, пожалуй. Подождите, сейчас умоюсь.

Быстро сполоснув лицо и почистив зубы до сих пор непривычным зубным порошком, я отправился с девушками на спортплощадку. 

\paragraph{}<++>

Не сказать, что я сильно выкладывался: размялся по-быстренькому, да отмотал четыре круга в среднем темпе. 

Оставив Славю с Леной, я решил обойти лагерь. Вдруг что-то подтолкнет меня к разгадке. 

Идти было одно удовольствие: свежий, слегка влажный от близости реки, воздух, запахи нескошенной травы и близкого леса, пение птиц и стрекот насекомых. И даже стена туманной мглы не казалась такой угрожающей. Наверное, я к ней попросту привык. 

Вот здесь мы ставили сигнализацию: бухта проволоки и нарубленные колья так и лежат. Интересно, сколько я спал? Часа два, не больше. Хотя, пребывание в той, другой реальности, казалось куда более длительным. Ай, ладно. Все равно пока ничего не понять.

Следов неведомых тварей вроде бы не попадалось. Хотя, следопыт из меня не ахти. Ладно, пойдем дальше.

Похоже, что вчерашние события  оставили некоторый отпечаток на пионерах: бродящих в одиночку не встречалось. 

-- Привет охотникам! -- Из-за угла выскочило красное пятно с двумя характерными хвостиками. Да уж, всегда найдется тот, кто все сделает по-своему. Хотя, уж кто бы говорил\ldots

-- Привет, Ульянка. -- Бросил на девчушку осуждающий взгляд. -- Чего одна бегаешь?

-- А я не одна, я с тобой вот. Кормить нас сегодня будут?

Мы как раз следовали мимо столовой, и судя по запаху, ответить можно было утвердительно.

-- Ну должны, а чего?

-- Да вожатка, похоже, дрыхнет. И ни линейки, ни завтрака\ldots

На площади, надо сказать, народ понемногу собирался. Очевидно, сказывалась привычка. Мда, распорядок летит в тартарары. Ну и ладно. Не особо-то и печалюсь на этот счет. 

-- Завтрак мы сообразим, -- хмыкнул я, -- пошли.

Стоило только поставить ногу на ступеньку крыльца столовой, как раздался звук горна. 

-- Ну вот, сейчас опять нудить начнет\ldots -- Протянула шкодница. 

-- Ладно, давай послушаем. -- Бросил я, развернувшись к площади. -- Все равно недалеко же.

\paragraph{}<++>

Ольги Дмитриевны, однако, видно не было. Вместо нее, в центре площади, стояла уже переодевшаяся в пионерскую форму, Славя.

Вроде бы, собрались все: даже Витька потирал глаза, да Алиса теребила пионерский галстук, намотанный вокруг запястья. Тут же галдела детвора, окружившая черноволосую, коротко стриженную пионерку в очках. Та напустила на себя строгий вид и что-то втолковывала ребятам.

А Славя, похоже, нервничала: подергивала кончик своей роскошной косы, переброшенной через плечо. 

-- Все собрались? -- Помощница вожатой окинула глазами площадь и удовлетворенно хмыкнула. -- Ольга Дмитриевна приболела, и сегодня линейку проводить буду я. 

Стоящая рядом со мной Ульянка гнусненько хихикнула:

-- Ага, нажралась вчера в зюзю. 

-- Тихо ты, -- шикнул я, -- не позорь честное имя почтенной дамы.

-- По порядку рассчитайсь! -- Командовала, тем временем, Славя. В принципе, пересчет логичен: так гораздо проще, чем пытаться пересчитать всех по головам. А учитывая, что знаешь весь лагерь в лицо\ldots
-
-- \ldotsДцатый, -- послышался чей-то голос. -- Расчет окончен!

-- Итак, все на месте, никто не пропал, -- констатировала исполняющая обязанности вожатой. -- В свете последних событий (если кто еще не знает, в окрестностях лагеря объявился медведь), требуется предпринять некоторые меры безопасности: младший отряд переселяется в административный корпус. Соответственно, после завтрака нужно будет перенести туда кровати. Вожатой младшего отряда будет Женя. Жень, на тебе мероприятия в библиотеке, ну ты знаешь.

Серьезная пионерка кивнула. Видимо, ей не привыкать.

-- По лагерю перемещаться поодиночке запрещается. -- Продолжала Славя. -- После обустройства младшего отряда, продолжим установку сигнализации. Ответственным по безопасности будет Семен. Добавишь что-нибудь?

Подумав секунду, я вышел из строя и двинулся в центр площади. 

-- За территорию не выходить, если что-то странное увидели, орите погромче. Желательно, что-нибудь для самозащиты найти: ножки там от мебели, молотки, топоры, ножи кухонные, наконец. Все. 

Со стороны младшего отряда послышалась какая-то возня и вперед протиснулась белоголовая девчушка лет десяти. Наверняка хочет что-то спросить. Вон и руку тянет.

-- Ну говори, ага.

-- А когда Петр Иванович вернется? -- грустно спросила девочка.

-- Не знаю, дите, не знаю. -- Оставалось лишь покачать головой. Не скажешь ведь ребенку, что, скорее всего, директор уже на том свете.

-- Ладно, -- вновь взяла инициативу в свои руки Славя. -- А сейчас мы идем завтракать. 

\paragraph{}<++>
-- Эй, не спи, замерзнешь! -- ткнул меня в бок локтем Электроник. -- Ешь быстрей, надо сигналку закончить. 

Вожатая все-таки объявилась: вон она, за одним столиком с Виолой. Удивительно, но Ольга, как будто, даже не пыталась вклиниться в перешедший на самоуправление лагерь. Или, может быть, тяжело командовать на больную голову?

-- Электричество нам подается, -- заговорил, решивший все-таки посетить заведение общепита, Шурик, -- значит на электростанции есть люди. Может, они больше нашего знают на счет всего этого?

-- Да, пожалуй, -- Серега пожал плечами и хмыкнул. -- Только вот до нее километров семьдесят.

-- Машины у нас нет, -- резюмировал Шурик. -- Значит, надо собирать передатчик. 

-- А как же <<Волга>>? -- удивился я.

-- Это не машина, -- Шурик поморщился, -- а ведро ржавое. Петр Иваныч ее все чинил, а чтоб ездил -- я не видел.

Оба пионера предельно серьезны. Да и вообще, чувствуется, что в лагере что-то изменилось: в воздухе повисла тревога. Не слышно веселого смеха, никто не носится в проходах, нет эдакого гула беззаботности. Все что-то обсуждают вполголоса. 

-- Слышь, Шурик, а ведь передачи-то с той стороны идут, -- я отложил вилку и потер виски ладонями, -- Может и еще кто-нибудь выйдет. Только вот здесь у меня дурное предчувствие.

Пионеры переглянулись и как-то помрачнели. 

-- Опять что-то рассказать забыл? -- В голосе Сыроежкина слышалось явное осуждение. -- Эти твои <<медведи>> приемниками научились пользоваться?

-- Не. Крутится что-то такое в голове, не могу вспомнить. -- Есть больше не хотелось, и я отодвинул тарелку. -- Честно, я тут вообще запутался.

-- В чем ты запутался, Семен? -- донесся из-за спины голос Слави. Девушка, тем временем, обошла столик и присела на свободный стул. -- Так в чем дело?

-- Да вот не знаю, -- я посмотрел в больше голубые глаза пионерки, -- где реальность, а где бред\ldots

-- Что-то странный ты сегодня, -- Электроник вытянул руку и попытался ощупать мой лоб. -- Ну вот, еще и дерганый. Температуры-то нет? Может, не выспался?

-- Я был там раньше, -- взяв со стола вилку, ткнул ей в сторону, где, по моим представлениям, находилась стена тумана. -- Может быть, до аварии. А, может быть, я до сих пор там.

-- Что? -- Брови Слави поползли вверх. -- Как понять, <<там>>?

-- Подыхаю в заброшенной больнице, а это все -- глюки.

-- Ну, я вот точно не глюк, -- поднялся из-за стола Серега. -- Если все наелись, пойдемте сигналку доделывать.

\paragraph{}<++>

По пути в кружок <<кибернетиков>>, к нам присоединилась Алиса с Витькой. Славя также решила составить компанию. Выглядела она обеспокоенно: то и дело поглядывала в мою сторону.

Шурик с Электроником, натягивали проволоку и ставили какие-то детали, Витька, как и ночью, рубил, заострял и загонял в землю колья, Алиса тоже делала что-то, наверняка, очень нужное\ldots

От меня же пользы, в общем-то, было немного. Вглядываясь в лес, то и дело прокручивал в голове недавний сон. Или не сон. Чем бы оно ни было, но душевного спокойствия не добавляло. Перекатывая в ладони пару картонных цилиндриков, периодически подбрасывая в воздух то один, то другой, кажется, я совсем погрузился в свои мысли.

-- Эй, слышишь меня? -- приблизилась Славя. -- Значит, память возвращается?

Девушка замерла в каких-то трех шагах от меня. Смотреть на нее, как всегда, приятно. Не только потому, что хороша собой. Бывает такое в некоторых людях: словно светятся изнутри, распространяя вокруг себя тепло. 

-- Понимаешь, -- в последний раз перекатив патроны в руке, убираю их в патронташ, -- она, память эта, как лоскутное одеяло.  

-- Например? -- Девушка перекинула свою косу через плечо и дергала кончик. Нервничает.

-- Разное там. -- Мешала то ли врожденная подозрительность, то ли живое воображение, рисовавшее картину: туман рассеивается, все налаживается, а меня тащат в застенки КГБ как представляющего интерес для государства и науки. Нет, так дело не пойдет. 

-- Ну, что ты все время виляешь? -- Пионерка покачала головой, -- Сказал <<а>>, говори уж <<б>>.

-- Посмотрим, -- буркнул я в ответ и направился в сторону трудолюбивых <<кибернетиков>>. Почему-то всегда жду от людей подвоха.

\paragraph{}<++>

Делать было, в общем-то, нечего: работающий над установкой сигнализации уже сложился. Боевое прикрытие, вроде бы, не требовалось. И оставалось мне, посему, слоняться без дела. Славя, после нашего разговора куда-то убежала. Видимо, организует быт лагеря.

-- Ребят, справитесь тут? -- обернулся я к пионерам. -- Пойду над снарягой поколдую.

-- Да иди, уж, -- откликнулся Электроник, -- Если что, позовем. 

В лагере было немноголюдно: похоже, что население лагеря сидело по домикам. Хорошо, однако, вчерашние события зацепили. 

С площади доносился какой-то скрежет и бряцанье, перемежаемые тонкими девичьими голосами. Интересно, что там происходит? Я прибавил шаг.

Точно! Организация жилья для младшего отряда в самом разгаре: Лена и Мику несут матрасы, а Славя и, как это ни удивительно, Ольга, волокут кровать в сторону административного корпуса. Кровать периодически задевала асфальт и раздавался тот самый скрежет. Ну вот, достигли газона. 

Помочь им, что ли? Пока наблюдал, девушки практически донесли кровать до дверей. Не буду мешать, надо все-таки доделать то, что собирался.  С этими мыслями, я и свернул в сторону кружков.

А вот и домик под номером восемнадцать. Мое скромное жилище. Задерживаться здесь, в общем-то, незачем, рюкзак только захватить. Именно за ним и пришел.

Припомнив, через что этому рюкзаку пришлось пройти, принюхиваюсь. Да вроде не пахнет. Ладно, пора идти. 

На причале безлюдно, хотя со стороны площади слышна возня. Постояв пару секунд и оглядевшись для проформы, направляюсь к сторожке. Выбитое окно, осколки стекла на земле. Ну, что поделать. 

Аккуратно перебросив двустволку в окно, забираюсь в сторожку и сам. 

-- Эй! -- Ульянкин голосок ни с чем не перепутаешь, --  А чего ты там делаешь?

Поглядев в окно, вижу снаружи и источник голоса. Ну вот когда успела объявиться?

-- Опять одна? -- Тон мой приветливым не назовешь. Девчушка, тем временем, кряхтя начала карабкаться в окно. -- Ну куда лезешь?

-- Уф! -- Приложив еще немножко усилий, непоседа, наконец, вваливается в оконный проем. -- Помог хоть бы!

Говорить с этой маленькой егозой не охота. Вообще не хочется говорить. Надо ремень хоть как-то к ружью приладить. Зачем, спрашивается, бывший владелец вывернул антабку из приклада?

Повернувшись, укладываю ружье на верстак и начинаю искать подходящий саморез. Ввинтить его в приклад, согнуть -- уже хоть что-то. Или, может быть, все-таки родная найдется? Надо еще в том ящике-топчане порыться. 

-- Ты чего делаешь? -- Опять лезет Ульянка. -- О, а дай ружье подержать!

Разомкнув замок, проверяю патронники. Пустые. 

-- Ладно, -- разомкнув замок, проверяю патронники. Пустые. -- Держи, только вхолостую не щелкай. 

-- Почему? -- Брови девчушки смешно взлетают вверх.

-- Портится потому что. -- Начинаю опять перебирать содержимое коробки. Вот ведь, хотел еще дробь переплавить. Ладно, сначала надо с ремнем разобраться. -- А чего тебе с остальными не сидится? 

-- Скучно там. -- За спиной слышна возня и кряхтение. Обернувшись, вижу, как пионерка, закусив губу, пытается разомкнуть ружье. -- А как оно открывается?

-- Рычаг вон тот вправо сдвинь. -- Даю совет по обращению с оружием. -- И палец со спуска убери. Вообще, из скобы убери его!

-- Уф, тугой какой! -- Наконец, переломив ружье, рыжая заглядывает в стволы. -- А почему палец  туда нельзя?

-- Потому, что, -- откинув крышку с матрасом, выкладываю содержимое. Патроны, валенок, коробки с закруткой и прочим добром для снаряжения. Может быть, где-то на дне завалялась?

-- Так почему? -- не унимается Ульяна.

-- Вот идти будешь с пальцем на спуске, споткнешься и отстрелишь себе ногу. Хотя, из ружья обычно тех, кто рядом идет, стреляют. -- В ящике, тем временем, не осталось ничего. И вожделенной детальки там, судя по всему, нет. 

-- Семен, а ты где с ним обращаться научился? -- выпускать ружье из рук девушка, видно не хочет. Понравилось открывать-закрывать.

-- Да у деда такое же было. -- Машинально отвечаю, обходя сторожку. -- О, а вот это может пригодиться. 

Металлическая дверца от хлебницы или чего-то в этом роде. Вырезать пластинку под размер, согнуть, просверлить отверстия и посадить на саморезы. А петлю из гвоздя можно сделать. Решено!

-- Ульян, дай ружье, замерить надо. -- Прикидываю, каких размеров нужна пластина и, отложив ружье, зажимаю кусок металла в тиски и берусь за ножовку. -- Так чего ты все по лагерю одна бегаешь?

-- Так не сидеть же с Женей и мелкими в библиотеке? -- пожимает плечами девушка. -- Скучно!

Работать с ножовкой и болтать -- верный путь к травматизму, потому беседа плавно сошла на нет. Изделие получилось довольно аккуратным. Хотя, не велика его сложность, трудно испортить.

Примерив, и подогнав, завинчиваю шурупы. Отлично! Достав из рюкзака брезентовый ремень цвета хаки, цепляю его на антабки. Вскидываюсь, прикладываюсь и, просунув в ремень левую руку, закидываю ижак за спину.

Времени на то, чтобы покидать содержимое ящика обратно, потребовалось немного. Смахнуть металлические опилки с верстака на пол, да можно и собираться. Надо бы еще патронов в рюкзак подбросить. На всякий случай. Да ружьшко зарядить, нечего теперь, при наличии ремня, с пустым оружием разгуливать.

\paragraph{}<++>
В столовой, похоже, собрался весь лагерь. Однако, на будничный ужин мероприятие походило мало. 

За сдвинутыми четырьмя столами расположился старший отряд в полном составе. Тут же и весь немногочисленный персонал лагеря: вожатая, медсестра и повариха. 

Даже ужинавшая, как обычно, детвора, галдела гораздо меньше. 

-- Сигнализацию мы подключили, -- отрапортовал Шурик. -- Подключили к динамикам системы оповещения лагеря. Если что-то полезет, мы это услышим.

-- И долго нам так сидеть, на осадном положении, -- Женя, скривившись, окинула взглядом всех собравшихся. -- Подумаешь, медведь. 

"Кибернетики" переглянулись. 

-- Думаю, темнить больше нет смысла, -- повернулся ко мне Электроник и, откашлявшись, продолжил. -- Официально заявляю, что это был не медведь, а неизвестная форма жизни. Агрессивная. Семен?

-- Угу, агрессивные, -- отхлебнув компот, я принялся ковырять вилкой котлету. -- Пользуются примитивным оружием.

-- Так! -- перебивает Ольга, -- Сначала вы мне рассказываете сказки про медведя, теперь вот это? Вы меня за дуру держите?

-- Оль, погоди, -- Виола, кажется, не особенно удивлена, скорей заинтересована. -- Ты, Семен, их там, в тумане видел?

-- Да. -- Резко отложив столовый прибор, я потер глаза. Опять нахлынули воспоминания: нападение в тумане, заброшенная больница, какие-то коридоры или даже туннели, хрустящие под ногами кости\ldots Лучи фонарей выхватывают из мрака, валяющиеся на полу, останки некрупных млекопитающих. Кажется, кошек и собак. Однако, тут и там попадаются и части явно человеческого происхождения. Смена декораций: мечущиеся в ревущем пламени бледные фигуры, истошные визги, хлопки одиночных выстрелов. Добить тех, что не сгорели. Выжечь все, дотла!

-- Морлоки это. Как у Уэлса. -- на лицах собравшихся удивление, у той же Ольги глаза, буквально, лезут на лоб. -- Жрут все, но мясо любят особенно. Когда нечего, молодняком своим не брезгуют. Плодятся быстро. Могут на спящего или раненого напасть, даже в одиночку. А когда много их -- жутко это.

-- Вообще, ребята, это все бредом сумасшедшего попахивает, -- Виола морщится и пристально посмотрев на меня, переводит взгляд на Ольгу.

-- Вот! -- кивает та и, вскочив со стула, встает, уперев руки в боки. -- Ты, Семен, с самого начала себя странно ведешь!

-- С другой стороны, -- продолжает медсестра, -- Туман этот -- что-то явно ненормальное. Кстати, этого вашего морлока, кроме Семена, кто-нибудь видел?

-- Ну, вообще-то, да, -- поднимается Шурик. -- Это создание Лену и Мику напугало. После этого мы и решили сигнализацию организовать.

-- Нет, ну что за бред? -- не унималась вожатая, -- морлоки какие-то!

В столовой повисла тишина, кажется, даже детвора перестала галдеть. Каждый, видимо, обдумывал новую информацию.

-- Семен, -- нарушила тишину Славя, -- оно оттуда же, откуда и ты?

Однако. Похоже, раскрыли. С другой стороны, сложить два и два не так сложно. Достаточно, всего-навсего, не закрывать глаза на факты. 

-- Я уже сам не знаю, откуда я. -- пробормотал я себе под нос, но девушка, кажется, меня услышала.

Внезапно стало, как будто, темнее. Для заката еще рано, может быть, наконец, тучки на небе появились? 

-- Эй, смотрите! -- закричала Ульянка, -- туман!

Действительно, за окном виднелась пока еще прозрачная дымка. Вскочив, я подхватил ружье и бросился к окну. По земле ползли серые щупальца тумана, где-то возникали завихрения, словно маленькие смерчи. 

-- Что это такое? -- Засмотревшись на поглощающую лагерь мглу, я и не заметил, как подошла Ольга. Обернувшись, поразился как же она бледна. И, кажется, ее колотит.

Туман становился все гуще. Главная улица терялась во мгле, и даже административный корпус практически скрылся из виду. И тут, мигнув, разом погасли все лампы. 

\paragraph{}<++>

В воздухе повисла, буквально, звенящая тишина. Стихли разговоры. Пропали все, характерные для столовой звуки. Даже гудение холодильников и электроламп. Слышно лишь, как в раковину из крана капает вода. 

-- Бля, страшно-то как! -- разорвал тишину пронзительный детский голосок.

-- Наружу никто не выходил? -- в горле внезапно пересохло, и мой голос стал хриплым. -- Все здесь?

Воображение живо рисовало тянущиеся из мглы щупальца, хватающие людей, опрометчиво рискнувших покинуть здание. Тела, опутанные паутиной, летящие на свет твари, напоминающие огромных мух\ldots Может быть, я видел это в кино?

Обернувшись, окинул столовую взглядом. Кажется, все на месте. По крайней мере, старший отряд. И персонал. Да и младшие, вроде, все здесь. Хотя, я их и по именам-то не знаю\ldots

-- Семен. -- Виола, скрестив на груди руки, сверлила меня взглядом. -- Почему отключился свет?

-- Очевидно, -- бросил я, подходя к входной двери, -- от станции нас отрезало. Надо бы запереться пока.

-- Как отрезало? -- округлила глаза медсестра. -- Что это вообще такое?

Откровенничать, как и раньше, не хотелось. На самом-то деле, я и сам мало чего понимал в происходящем, но соображения все-таки были.

-- Что-то творится с реальностью. -- вновь повернувшись к окну, я вглядывался в туман. Прозрачная дымка превратилась в непроглядное серое марево. Жуткое зрелище. -- А туман этот никакой нахрен не туман.

-- И что нам теперь делать? -- Подключилась к разговору вожатая. В голосе ее явственно слышались панические нотки. -- Так ведь не бывает! Не бывает!

В тумане, как будто, угадывалось какое-то движение. Или мне всего-лишь кажется?

-- Эй! -- Чуть отступив, я взял ружье на изготовку. -- Живо все от окон. Ну!

Сняв с кармана фонарь, повернул его голову в крайнее положение. Максимальная яркость. Яркий луч света пронзил стекло, но тут же растаял в серой мгле, так ничего толком и не осветив.

-- Дерьмо. Ничего не видно.

-- Ты только проверять не пойди, -- дал совет Электроник. -- А то, с тебя станется. 

-- Да как-то нет желания. -- Выключив фонарь, сунул его обратно в карман. -- Но выйти все равно, рано или поздно, придется. 

-- А сейчас-то что делать? -- Витька приник к окну. -- Эй, а оно движется!

Схватив рыжего за шкирку, я буквально оттащил его от стекла. 

-- Совсем дурной? -- возмутился парень.

-- Не соваться к окнам! -- рявкнул я. -- Не понятно? 

Палец застыл на предохранителе. Вообще, не похоже, что снаружи сновали какие-то твари. Движение -- это, скорее всего, завихрения тумана. Но нервы пошаливали.  

\paragraph{}<++>

За несколько часов туман не рассеялся. В столовой стало темнее. Страх, буквально, витал в воздухе. Слышался детский плач. Всхлипывали и шмыгали носом пиоренерки. Вожатую, буквально, трясло. Виола, похоже, пытается ее успокоить, держа за руку и шепча что-то на ухо. 

Несколько раз я обошел столовую. Запер главный и служебный входы, проверил окна. Жаль, что не на всех есть решетки. Непонятно, зачем их вообще нужно было ставить лишь на пару окон?

С другой стороны, не сидеть же здесь бесконечно? Хотя, еда и вода пока есть. Так что с выходом наружу можно и потерпеть. Другое дело, что при отсутствии электричества то, что в холодильниках довольно быстро испортится. 

Толстая повариха, как ни в чем не бывало, колдовала над газовой плитой. Питание от баллонов -- это хорошо, практично. Особенно, когда нет электричества. 

Вообще, тетка молодец. Происходящее ее, как будто, совершенно не трогало. Помешав что-то в огромной кастрюле, тетя Клава принялась что-то месить на огромной разделочной доске. Что она там, опять котлеты лепит? 

Восковая свеча, припасенная, судя по всему, на случай проблем с электричеством, освещала лишь стол, за котором занималась готовкой повариха. Вся остальная столовая погрузилась во тьму.


-- Семен, -- ухватив меня за рукав, прошептала Ульянка, -- там по крыше что-то ходит\ldots

Замерев, прислушался. Тихо. Никто не решался говорить в полный голос. Всхлипы, шепот, поскрипывание стульев и столов. 

-- Да вроде не слышно. -- Для верности, я приложился ухом к стене. -- Ты бы поспала, что ли. 

-- Там точно кто-то был, - Нахмурившись, девчонка уселась прямо на стол. 

-- А ты хочешь, чтобы я сейчас полез на крышу и проверил? -- Смерив ее взглядом, я плюхнулся на тот же столик. -- Как в идиотском ужастике. Тебе бы сценарии писать\ldots

Однако, на крыше, похоже, действительно кто-то был. Или что-то. Шорох, будто кто-то, не слишком тяжелый, перепрыгивает с места на место. 

-- Вот, вот, слышишь?! -- пискнула девчонка. 

-- Успокойся. -- Нарочито безразлично буркнул я и поднялся. -- Сюда не лезет, и ладно. К окнам не подходи.

Подсвечивая фонарем под ноги, я в который раз обходил помещение. Женя и Славя хлопотали вокруг детворы: успокаивали, хотя и сами напуганы.

Заметив мое внимание, Славя сделала пару шагов навстречу.

-- Может с нами посидишь? Детям в темноте страшно. 

-- Посижу, конечно. -- Фонарик на самом минимуме дает не много света, но это уже что-то. -- Может у тети Клавы еще свечки есть?

-- Может, -- пробормотала Славя. Чуть придвинувшись, она оперлась о мое плечо и, кажется, задремала. 

\paragraph{}<++>
Кажется, задремал и я. Точнее, сознание находилось в некоем пограничном состоянии. Сон вполглаза? Скорее, в четверть. 

В который раз я прокручивал в голове все произошедшее и не находил сколько-нибудь правдоподобного объяснения. Параллельная реальность? Эта версия напрашивается сама-собой. Однако, чем объяснить внезапное омоложение?

А странные сны? Другой я, пострадавший при взрыве боевой машины, заброшенный город, населенный странными тварями\ldots

Лагерь и люди его населяющие -- все это совсем не походило на бред. Пионеры и персонал кого-то я узнал получше, с кем-то перекинулся лишь парой слов. Но все они производили впечатление живых людей.

Вдруг, как на автобусной остановке несколько дней тому назад, накатило отчаянное чувство тревоги. 

За окнами что-то захлопало, раздался звук бьющегося стекла. Какое-то отвратительное кряхтение и скрежет, а следом отчаянный, полный ужаса, женский визг. Вот кричат уже несколько человек. Все это укладывается в какую-то пару мгновений.

Скатившись со стула и приложившись коленом о столешницу, подхватываю ружье. Славя, не удержав равновесие, кажется, падает на пол. Кричат уже, как будто, все. 

Наконец, вижу причину. В разбитое окно лезет нечто, напоминающее летучую мышь-переростка. Кожистые крылья, которыми гадина молотит, пытаясь протиснуться внутрь, увенчаны то ли когтями, то ли шипами. Отвратительного вида башка на тонкой шее: всю морду, как будто, занимает огромная зубастая пасть. 

Обрезанный приклад лягает в ключицу, вспышка! Два выстрела сливаются в один. Я оглох? Двенадцатый калибр в закрытом помещении, короткие стволы -- да запросто. Большой палец ощутимо обжигает. Привычный полный хват совершенно не годится для двустволки. Разомкнуть замок, резко тряхнуть казной вниз, но гильзы, вопреки ожиданию, не выпадают. Их слегка подуло. Выковырять, бросить, вогнать в стволы два патрона, оружие на изготовку\ldots

Головы у существа больше не было. Верхнюю часть размолотило первым выстрелом, а все, что осталось, превратил в кашу второй. Дистанция метров семь, четыре нуля -- деревенский стандарт. А грамм там сколько? Да уж не меньше тридцати пяти, судя по отдаче, а то и все сорок.

Ткнул срезом стволов в застрявшее в окне тело, и оно с каким-то мерзким хлюпаньем вывалилось наружу. 

-- Вот срань! -- Обнаруживаю, что стою в огромной луже крови, натекшей из, провисевшей здесь лишь несколько секунд, туши. -- Измазался, тьфу!

Хлопанье крыльев, за окном мелькает темная тень. Стреляю, скорее, на звук, по наитию. Выстрел уже не кажется таким громким. Что-то мерзко пищит, под окном какая-то возня. Еще выстрел в источник звука. 

Нужно перезарядиться. Снаружи, вроде бы, тихо.

-- Эй, -- кричу я, -- фонарь сюда!

-- Держи. -- Витька бледен как мел. Это видно даже в тусклом свете энергосберегающего режима. 

Переключив яркость на максимум, выглядываю в разбитое окно. Фонарь зажат между пальцами левой руки, обхватившей цевье дробовика. Туши под окном нет. Темное пятно, а скорее, лужа -- кровь. 

-- Ушло? -- Тихо спрашивает Витька. 

-- Скорей утащили. -- Пытаюсь осветить окрестности, но все равно ничего не видно. -- Не шибко-то без башки побегаешь. 

-- Надо окна завалить. -- Это явно голос Шурика. И Электроник здесь же. -- Давайте, что ли, столы подтащим.

\chapter{День пятый}

До утра вздремнуть так и не удалось. После нападения летающих тварей, пришлось всерьез задуматься об укреплении своего убежища. Как и советовал Шурик, окна завалили столами. Хлипко, но хотя бы что-то. Заодно и свет наружу не проникает. Хотя, много ли света от свечи и фонаря на минимуме? Да и сомнительно, что твари так лихо ориентируются в тумане, полагаясь лишь на зрение. 

Светало. Сквозь щели в между баррикадой и окнами начал проникать тусклый свет и по столовой стало возможно передвигаться без фонаря, не опасаясь запнуться обо что-нибудь и расшибить себе голову.

Отодвинув загораживающую оконный проем столешницу, я выглянул на улицу. Вместо густого тумана осталась лишь прозрачная дымка. Хотя, погожего солнечного денька, похоже, не будет. Природа была, как будто, серая: осевшая на земле и на растениях белесая взвесь и свинцового цвета, словно затянутое пеленой, небо. Лагерь разом утратил все былое очарование и приобрел нервирующие и пугающие черты. 

-- Что делать будем? -- Подошел Электроник и тоже выглянул в окно. -- Надо ж посмотреть, что там снаружи происходит, лагерь обойти.

-- Угу. -- Трудно не согласиться. -- Пойдем. Надо только Витьку и Шурика предупредить. Где они, кстати?

Окинув взглядом помещение, я так и не обнаружил интересующих меня личностей. Вон там, вповалку спит детвора. Там же прикорнула и Женя. Лена и Мику сидят за одним из немногих, не пошедших на баррикаду, столиков. Чай пьют. Слави не видно, на кухне, что ли, тете Клаве помогает? Ольга спит на полу, положив голову на бедра сидящей, опираясь спиной на стену, Виолы\ldots Та, в свою очередь, поглаживает волосы вожатой -- забавное зрелище.

-- Витька вон, с Лиской шепчутся. А Шурик, -- пионер кивнул куда-то в сторону стойки, -- дрыхнет в уголке -- умаялся. 

-- Пусть спит пока. -- Бросив осматривать однообразный серый пейзаж, я направил стопы в сторону шушукающихся рыжих. -- Мы тут с Серегой прошвырнемся по лагерю. Если чего, орите, стучите в посуду, ну и оружие какое-никакое вроде есть тут\ldots

-- Так может я с вами? -- Парень поднялся с пола и отряхнулся.

-- А девчонок от монстров кто охранять будет? -- Осадил Витьку Электроник.

-- А чего нас охранять? -- Возмутилась было Алиса, но вспомнив, видно, ночные события, быстро сникла. 

-- Не, Витек. Если чего, Шурика буди, а мы пойдем. -- Скинув ружье из-за спины, я направился к выходу. 

\paragraph{}<++>

Снаружи лагерь выглядел ничуть не приветливее, чем изнутри. Все, как будто, перекрасили в серый свет. Даже Солнца не видно, свет какой-то рассеянный. 

Тишина и полное безветрие. Ни пения птиц, ни стрекота насекомых. Лишь звук шагов, слегка поскрипывающая белесая дрянь под ногами. Все это создавало чувство ирреальности происходящего. И навевало ужас. Какой-то задавленный, прячущийся в глубине сознания - липкий и холодный. 

Мне было проще. Все это я уже видел, а вот Сереге, похоже, стало дурно. Застыв на крыльце, он затравленно озирался по сторонам, губы его шевелились, нашептывая что-то мне неслышное.

-- Эй, ты чего? -- Хлопнул я пионера по плечу.

-- А, да-да, нормально. -- Электроника передернуло, он потер глаза, зажмурил их и вновь распахнул во всю ширь. -- Это мне не снится?

Сплюнув, я покосился на пятна крови под окном, куда свалился подстреленный летун. А ведь действительно, тварь утащили -- очень уж подтеки характерные. Сначала по земле, а потом взлетели. Метрах в десяти валялась оторванная окровавленная лапа. Кожистая, будто птичья. Три пальца с узкими, но внушительными когтями, противостоящий четвертый палец.

-- Глянь, забавный сувенир. -- Подхватив лапу, я подбросил ее в воздухе и протянул Сыроежкину. -- На тебе, материальное свидетельство. 

-- О как. -- Покрутив частичку новой реальности в руке, парень протянул ее обратно. -- И чего с ней делать?

-- Суп не сварить, да. -- Забрав окорочок, я зашвырнул его на середину площади. -- Пошли уже. 

На памятнике ночью кто-то явно сиживал. Голова Генды была заляпана кровью ночного гостя, а рядом с постаментом валялись объедки: куски шкуры, какие-то комки шерсти, что-то напоминающее кости\ldots Ну и помимо крови, дерьма, в прямо смысле слова, хватало.

-- Да. Кишки короткие у наших птичек. -- Разглядывать подробнее остатки трапезы не хотелось. -- Где пожрали, там и насрали.

-- Мерзость, -- согласился Сережка.

\paragraph{}<++>
Кружок "кибернетиков" был заперт. Вроде, ночью в него никто не ломился: нет следов на крыльце или под окнами, стекла целы. В целом, выглядит все довольно безопасно. 

-- Погоди, не суйся, -- остановил я Электроника, уже повернувшего ключ в замке и приоткрывшего дверь.

-- Так заперто же было, -- возразил тот, и попытался протиснуться в проем. Пришлось схватить его за шиворот и ощутимо дернуть.

-- Да куда лезешь. Под ствол хоть не суйся. -- Отпустив парня, приоткрыл дверь левой рукой, правой удерживая дробовик у плеча. -- Сначала захожу я, ты спину держишь со своим молотком.

Внутри было пусто и тихо. Не гудит ни компьютер, ни лампы дневного света -- щелчок выключателя, естественно, не произвел никакого эффекта. Вроде бы и все знакомое, но\ldots  Все предметы были на тех самых местах, где их и оставили, но отсутствовало то ощущение, когда приходишь, например, раньше всех на работу, включаешь свет, неспешно раздеваешься -- чувство привычности и безопасности. 

-- Тут батарейки где-то были. -- Электроник порылся в одном ящике, потом в другом. -- К приемнику-то. Может чего и поймаем.

Подойдя к столу, я заметил здоровенную, почти квадратную батарею с контактами-пластинками. 

-- Это не оно?

-- Не, -- мотнул головой пионер, -- крона нужна.

Еще несколько минут поисков, и батареи найдены. Шипение и свист при нынешних обстоятельствах звучали зловеще. Электроник все прощупывал диапазон, крутя ручку. Внезапно тональность изменилась. Еще немножко подстроить и можно будет что-то разобрать.

-- Ого, -- пионер направился в угол комнаты, где должен был быть штекер наружной антенны, -- сигнал-то гораздо четче стал, без внешки поймал. 

Подключив приемник, Электроник подкрутил ручку настройки и из динамика донеслось: <<Внимание! В городе опасно! Много тварей! Разлом вызывает безумие! Убежище на радиозаводе, есть еда и медицинская помощь! Частоты для связи\ldots >>

-- Охренеть, -- почесав затылок, я присел на краешек стола. -- Та самая передача. 

-- Только понятнее не становится, -- обернулся Электроник. -- Какой еще разлом?

-- Думаю, что это все. -- Я кивнул в сторону, где когда-то была стена тумана. -- Откуда гады всякие лезут, да туман этот долбаный.

Сообщение, тем временем, повторялось. Запись, понятно. 

-- Надо остальным сказать, -- Электроник выключил приемник и выдернул штекер из гнезда. -- Там на заводе этом, помощь должна быть.

\paragraph{}<++>

Народ в столовой, как будто, успокоился. Даже разобрали часть баррикады, чтобы было на чем завтракать. Хотя, я думаю, зря -- можно было и потесниться. Сразу же, на пороге, нас встречала делегация в составе Ольги, Виолы и Слави.

-- Ну как, разведчики, -- с каким-то придыханием, произнесла медсестра, -- есть жизнь снаружи?

-- Опасных форм жизни не обнаружено! -- Выпалил Электроник.

-- Если не считать, остатков их жизнедеятельности, -- добавил я. -- В любом случае, разбредаться нельзя. Мы тут передачу интересную поймали.

Переместившись за один из столов, рассказал о зацикленном сообщении.

-- Наверное, нам нужно туда -- Ольга, кажется, начала приходить в себя, -- на радиозавод этот.

-- А как мы туда доберемся? -- взглянула на вожатую Славя, -- вы вообще знаете, сколько до этого завода идти?

Честно говоря, отправляться непонятно куда эдакой неорганизованной толпой мне совершенно не хотелось. Особенно, если задуматься, что можно встретить по дороге.

К нашему столу, тем временем, подходили люди. Витька с Алисой, проснувшийся, очевидно, Шурик и даже Ульянка.

-- Может это, <<Волгу>> заведем? -- Предложил рыжий.

-- А водить-то умеешь? -- Покосилась на него Алиса. 

-- Ну\ldots -- Витька как-то, поначалу, стушевался, после чего расправил плечи и заявил: - А чего там уметь? Жми педали, да баранку крути -- разберемся! Подумаешь\ldots

-- Вообще, -- прервала пионера Виола, -- у меня есть права. Если куда-то поедем, конечно. 

-- А это ведро с гайками на ходу? -- невинно полюбопытствовал Шурик.

-- Пока не проверим, не узнаем, -- подытожила Славя.

\paragraph{}<++>

Ключей от машины, разумеется, ни у кого не было. По общим предположениям, находиться они могли либо в кабинете директора, либо где-то в его домике. Две поисковых бригады решили не организовывать, исходя из соображений безопасности. На этот раз к нам с Электроником присоединились Славя, Витек и Алиса.

Желания копаться в вещах пропавшего директора у меня не было ни малейшего, потому предпочел остаться снаружи. Покараулю.

Судя по звукам, доносящимся из кабинета, поиски шли ожесточенно. Что они там, мебель опрокидывают и ломают?

-- Да что вы творите? -- донесся из помещения голос Слави.

-- Искусство требует жертв! -- это явно Алиса. За репликой сразу же последовал грохот.

Похоже, это не обыск, а какой-то погром. С этими мыслями я и направился внутрь здания. 

Картина представала устрашающая: кучи какого-то канцелярского барахла на полу, вывороченные из столов ящики, разбросанные повсюду книги\ldots Когда только успели?

-- А теперь давайте это все подожжем! -- произнес я с порога.

-- Зачем? -- не оценил предложения Электроник.

-- Ну, раз уж вы тут все разгромить решили, чего на полпути останавливаться? -- Пнув ногой какую-то книжку, я подошел к столу. -- Ключи, конечно, не нашли?

Славя обернулась и посмотрела на меня с какой-то грустью.

-- Семен, а не получится так, как с Петром Ивановичем и Афанасием Кузьмичом? -- девушка указала куда-то вдаль. -- Даже если заведем машину, там ведь это, всякое\ldots

-- Угу, -- согласилась Алиса. -- Ты вот уедешь со своим ружьем, а нас тут гадины летучие сожрут. 

-- Тебя вон, Витька защитит, -- как-то зло бросил Сыроежкин. -- А вообще, если мы их передачу слышим, значит и нас услышат. Связь нужна.


\paragraph{}<++>
Дело, должно быть, шло к полудню. В любом случае, от тумана мало, что осталось. Можно было подумать, что сегодня просто пасмурный летний денек, и ничего особенного не случилось. 

О произошедшем напоминала только белесая субстанция, покрывающая тонким слоем практически все. Как будто, маленькие гранулы. Явно не пыль и не пепел. Может быть, какие-то споры? Соскреб ножом на ладонь, я присмотрелся. Нет, никаких ассоциаций и воспоминаний.

Я сидел на лавочке и чистил ружье, задумчиво прогоняя, слегка промасленный, кусок ткани через стволы шомполом. Смысла, в общем-то, особого в этом не было, ничего ружью не будет. Но чистка оружия ли, заточка ножа -- эти процессы успокаивают. 

Электроник с Шуриком колдовали у себя в клубе над передатчиком. Зачем-то к ним присоединилась и Ульянка. Неужто, что-нибудь понимает в радиотехнике? Ажиотаж вокруг "Волги" несколько поутих -- ехать куда-то наобум, все-таки, никому не хотелось. Да и не зря же, в пойманной передаче, дали частоту для связи. 

Любопытно, что никто из населения лагеря не знал, о каком, собственно, заводе идет речь. Только Виола припомнила, что собирались что-то такое строить на окраине города. А, может быть, уже и начали\ldots

А наталкивало это на мысль, что уже не тот это мир победившего социализма. И не из портала какого-то, похоже, всякие твари лезут, а весь лагерь ухнул в тар-тарары. Пропавшее электричество в картину укладывается.

Все заняты каким-то делом. Вон, пионеры, а точнее, пионерки заселяют административный корпус. Кровати туда перенесли ранее, теперь вот пришел черед и остальных вещей. Да, в домиках жить теперь небезопасно, нужно держаться вместе.

Почувствовав чье-то присутствие за спиной, я дал провалиться шомполу в один из стволов, а правой рукой потащил из патронташа патрон. 

Однако, угрозы не было: позади, облокотившись на спинку лавочки, тихо, как тень, стояла пионерка. Лена. 

-- Кхм! -- прочистил я горло, -- а ты чего здесь?

-- Извини, -- девушка, что ли, покраснела или это так свет падает? -- мешаю?

-- Да ну, не, -- отпустив патрон, расслабился я. -- Ходишь тихо.  

Девушка обошла скамейку и присела на краешек. Одета она была, почему-то, не в пионерскую форму, а в спортивный костюм. Облегающие трико и футболка, на плечи накинута легкая олимпийка, а на ногах высокие кеды.

-- Ты зарядку, что ли, делала? - покосился я на нее. 

Вроде и обстановка не располагает, но до чего же тяжело оторвать глаз\ldots Поймав себя на том, что попросту пялюсь на затянутые в черные трико ножки, я крякнул и подобрал валявшуюся на земле тряпку. 

-- Так удобней будет, -- Лена глядела куда-то вдаль, -- если убегать придется. -- И, сделав паузу, добавила, -- или драться.

Какое-то время мы молчали. Я, в который раз, протирал, начищенные уже, должно быть, до зеркального блеска, каналы стволов. Лена все также задумчиво осматривала то ли лагерь, то ли, теряющийся в тумане, горизонт. А пионеры продолжали переносить вещи. 

-- Семен, -- позвала соседка по скамейке, обводя окрестности рукой, -- откуда все это?

-- Не знаю, -- пробормотал я. Почему-то хотелось поделиться, рассказать что-нибудь этой замкнутой девушке, но толку? -- По-моему, что-то нехорошее с миром приключилось, то есть с мирами.

-- Эй! -- Со стороны клубов бежали трое: кибернетики и Ульянка. -- Мы кое-что придумали, питание нужно!

\paragraph{}<++>
Для того, чтоб запитать передатчик, было решено использовать аккумулятор от <<Волги>>. Только вот, задачу осложняло то, что ключи от машины так и не были найдены. Вполне возможно, что директор их, попросту, забрал с собой, перед тем, как канул в Лету. Да и в автоделе никто из населения лагеря не разбирался.

-- Ну, и что делать будем? -- Электроник, аж приплясывал. -- Может это, поддеть чем-нибудь и сковырнуть? 

-- А вон, -- шустрая Ульянка уже во всю изучала капот, едва не протискиваясь сквозь радиаторную решетку. -- Семен, дай фонарь, а!

-- Чего нашла? -- Протянув девчушке желаемое, я подошел ближе и тоже пригляделся. 

-- Да вон, там под значком что-то, -- Ульянка потыкала кнопку фонаря, включив его поярче, --  может ткнуть туда прутом твоим?

Похоже, что шомпол пригодится и для вскрытия машины.

-- Подсвети, -- просунув в щель над радиатором кончик латунного прута, поводил им туда-сюда,  -- во, нащупал что-то. 

Тычок, еще, немного. Может по-другому нажать? Характерный металлический звук и капот приоткрылся. 

-- Ураааа! -- Захлопала в ладоши Ульянка, -- какая я молодец! 

-- Молодец, молодец, -- похвалил и я, -- фонарь-то верни. 

Шурик, тем временем, извлек из, принесенной с собой сумки, характерного вида прибор. Явно не школьный вольтметр, а что-то куда более серьезное: тройная шкала, черная регулировочная ручка, какие-то кнопки (видимо, для переключения режимов) -- прямо советский мультиметр.

-- Эх, вот незадача, -- разочарованно потянул Шурик, -- аккумулятор-то почти дохлый. Что делать будем?

-- А чего делать, -- Электроник наклонился к открытому капоту, -- Снимать надо. -- Парень что-то там поковырялся, и ругнулся сквозь зубы. -- Да он прикручен тут. Есть чем отвернуть?

-- На, держи, -- Сняв с кармана мультитул и раскрыв пассатижи, я протянул его парню. 

\paragraph{}<++>

-- А вы точно на нужной частоте передаете? -- Крутя в руках очки, произнесла Виола.

В помещении кружка <<Кибернетиков>> было не протолкнуться. Помимо, собственно, членов клуба, здесь присутствовала добрая половина старшего отряда, и две трети персонала, в лице медсестры и вожатой.

-- Да, да, -- пробурчал Шурик, -- я все на бумажке записал и перепроверил\ldots Давайте еще попробуем. 

-- Радиозавод, ответьте лагерю, -- забубнил в изъятый из музыкального кружка микрофон Электроник. -- Радиозавод, ответьте лагерю! 

-- Почему они не отвечают? -- В очередной раз, проявила редкостную "смышленость" Ольга. Словно ответ на вопрос мог знать хоть кто-нибудь в этой комнате. 

-- Может, нет их давно, -- поморщился я, -- сожрали или перебили всех, а запись на автомате крутится, что и так ясно.

-- Ужасно, -- побледнела вожатая. -- А электричество?

<<Кибернетики>> переглянулись и как-то синхронно пожали плечами.

-- Да генератор какой-нибудь, -- Шурик оторвался от аппаратуры и обернулся. -- Кстати, поздравляю, товарищи: у нас сдох аккумулятор.

Разговоры и шушуканье затихли. Все молча переглядывались, только Лена безучастно глядела в окно. Казалось, происходящее в комнате ее совершенно не интересовало. 

-- Ну это, батарея еще найдется, -- мне наконец удалось оторваться от созерцания Леночкиной задницы. -- Только вряд ли ее надолго хватит.

-- Можно СОС пустить, -- предложил Электроник. -- Может, хоть кто-нибудь, да услышит. А что за батарея?

Предложение вызвало неприятные ассоциации. В памяти шевельнулось что-то нехорошее -- вспомнился пойманный раньше, но куда-то пропавший сигнал. 

-- Лена, подойди сюда, пожалуйста, -- позвала Виола, -- не отвлекай пионеров\ldots

Девушка, вздрогнув, резко обернулась. На щеках ее был заметен румянец. 

-- Так что, -- Виола обращалась уже ко мне, -- там у тебя за батарея, Семен? 

Рассказывать про лаптоп, и тем более демонстрировать его широкой общественности, по-прежнему не хотелось. 

-- Да есть тут у меня, двенадцативольтовка, кажется, -- ответил я, мысленно ругая себя за излишнюю болтливость. Не знаю почему, но идея Электроника нравилась все меньше. -- А может ну его, а Серега? 

Кажется, теперь на меня с недоумением глядели все собравшиеся.

-- Тебе что, батарею жалко? -- Удивился Сыроежкин. 

-- Было б жалко, -- почесал я затылок, -- не сказал бы. Предчувствие дурное.

-- Знаешь, Семен, -- с истеричными нотками в голосе, вновь вступила Ольга, -- твои предчувствия\ldots Надо с людьми связаться! Электричества нет и еда скоро кончится!

Здравое зерно в этой тираде, пожалуй, все-таки было. Еда, действительно, конечна. Что здесь делать, когда кладовка при столовой, шкафы, или что там еще, опустеют - не понятно. Суп из летающих страшилищ варить? В лес податься, грибы и яблоки собирать? 

-- Ладно, -- бросил я, направившись к выходу, -- будет вам батарея.

\paragraph{}<++>

Лес, окружавший лагерь, производил весьма гнетущее впечатление. Вроде бы, нормальная, здоровая растительность: какие-то хвойные, акация\ldots Да вот, все это, как будто, посерело. В который раз, я задавался вопросом: что же, все-таки, это за гадость выпала?  Да и жиденький туман очарования не добавлял. 

Аккумулятор, в итоге, достался <<кибернетикам>>. Дорога до будки Кузьмича  много времени не заняла, как и процесс извлечения лаптопа из недр ящика. 

Вопросы любопытствующих пионеров и медсестры, о происхождении источника питания с иностранными маркировками, остались без ответа. 

Сейчас я занимался обходом территории. Неспешно шел вдоль забора, снаружи лагеря, в поисках чего-нибудь необычного или опасного. Жаль, что электричества нет. Получается, зря сигнализацию ставили, потратив уйму сил впустую\ldots

Тишина оглушала. Не так должно быть в летнем лесу, ох не так. Ни дуновения ветра, ни пения птиц, затаились даже насекомые: им бы, в это время года, стрекотать наперебой. А если прислушаться, то можно различить голоса пионеров, переговаривающихся в глубине лагеря.

Неспешно, внимательно глядя по сторонам и под ноги, я продолжал свой путь. 

Вспомнились слова вожатой. А ведь, действительно, что же мы будем есть? На то, что откуда ни возьмись, примчится отряд спасателей, я не надеялся. Может быть, силки попробовать в лесу поставить? Знать бы еще как, не лесной я человек. Надо бы у Слави поинтересоваться, вдруг она умеет.

А вот и необычное: возле забора было натоптано. Хоть какая-то польза от этой странной субстанции -- следы на ней видно, что надо. Чьим конечностям принадлежали отпечатки, гадать не приходилось. Морлоки. И, явно, не в единственном числе. Как будто, вышли из леса, потоптались у забора, да и ушли обратно. Двое или трое. 

Если они за подмогой отправились, может нам прийтись худо. Растяжку бы здесь поставить, а не силок. Тяжело в пионерлагере без гранаты. Пороха в трубу натолкать, сдобрить все это гвоздями и гайками\ldots А воспламенять чем? Да капсюль, гвоздь и какую-нибудь пружину, для накола, можно приспособить. Хотя пороху, если честно, не так много, чтобы им разбрасываться. 

А интересно, подумалось, насколько эти морлоки съедобны? Людей-то они, вроде как, жрать не стесняются. Довольно мерзкие, конечно, создания. Хотя, вот раки, например, питаются, в том числе, падалью, при этом являясь деликатесом. Хамон из морлока к трапписткому элю, мда.

С такими гастрономическими мыслями, я и вглядывался в лес, поглаживая двустволку. Из леса, однако, никто не спешил выбегать и набрасываться. Идти по следу, тем более в одиночку, желания не возникало. 

Насколько разумны эти твари? Настолько, что пользуются простеньким холодным оружием. Хочется надеяться, что не устроят они нам здесь партизанской войны. В идеале, перебить бы гадин, да вот где гнездо их находится -- неизвестно. Не похоже, что далеко.

\paragraph{}<++>

Здравый смысл возобладал: прежде чем начинать обыскивать лес, я все-таки решил предупредить общественность. Да прихватить, хотя бы, компас. Впрочем, его не мешало бы, для начала, найти. Именно за этим я и вернулся на территорию клуба <<кибернетиков>>. 

Шурик с Электроником что-то собирали, пустив в ход батарею, снятую с моего лаптопа.  В общем-то, моего появления они, как будто, и не заметили. С тем же успехом в клуб могло ввалиться какое-нибудь исчадие иного мира, также оставшись незамеченным.

-- Ау, вызывает Земля! -- Воскликнул я и хлопнул в ладоши. Лишь тогда пионеры обратили на меня внимание. -- Компас есть у вас? 

-- Тут хранить, только портить, -- буркнул Шурик и вновь погрузился в работу.

-- А зачем тебе? -- Проявил чуть больше интереса Электроник, -- куда-то собрался?

Ну что за мода отвечать вопросом на вопрос?

-- Надо, -- ответ получился резковатым. -- Так есть или нет?

-- Да в домике, -- закивал Электроник. -- Тебе он срочно нужен?

Вкратце обрисовал ситуацию со следами около забора. Поделился и идеей пройтись по следу, а если повезет, отыскать логово. 

-- Ты совсем сдурел, что ли! -- Пионера, буквально, перекосило. -- В лес собрался? А еще кого-нибудь встретить, кроме морлоков этих, не боишься? Ладно сам, а если на нас тут полезут, пока ты в лесу бродишь?

Да, логика в его словах, действительно, есть. Появившиеся твари, в самом деле, могут создать проблем. Учитывая, что народ до сих занят переноской вещей и укреплением административного корпуса. 

-- Кстати, гении, -- мысль, крутившаяся в голове с того момента, как я переступил порог клуба и застал здесь увлеченно работающих <<кибернетиков>>, наконец, попала на язык, -- вы бы хоть запирались тут. 

Шурик, оторвавшись от горки деталей, как-то зло поглядел на нас обоих, и вновь погрузился в процесс\ldots

-- Пошли, короче, -- указал я на дверь, -- снаружи обсудим.

\paragraph{}<++>

На этот раз дверь все-таки заперли, оставив внутри Шурика с радиодеталями. 

-- Кстати, а как вы паять-то умудрились? -- Вспомнился характерный запах при отсутствии электричества. 

-- Да на спиртовке нагрели, -- ответил пионер с ноткой гордости.

По дороге решили заглянуть и в наше общее убежище. То есть, в бывший административный корпус. Работа там кипела во всю: Витька заколачивал окна откуда-то  извлеченными досками, девушки под руководством Слави носили из столовой какие-то припасы да, похоже, емкости с водой. В принципе, логично. На случай нашествия монстров, если наружу выходить будет опасно -- мера не лишняя.

-- О, -- помахал нам рукой Витька, -- помогайте крепость строить!

-- Потом, -- отмахнулся Электроник, -- Семену компас нужен, и вообще тут твари бродят рядом.

-- Какие твари?! -- Раздался звонкий голосок Ульянки, -- летучие?

Тут, следом за рыжей непоседой, из здания показалась и вожатая. Похоже, и для нее наше появление не осталось незамеченным. Я вновь поведал о следах и возможности атаки. 

Ольга скрылась в помещении, а через несколько секунд появилась с пионерским горном в руке. 

-- Вот, -- вожатая покрутила духовой инструмент в руках, -- подойдет тревогу поднять. 

Здравых мыслей я от нее не ждал, потому несказанно удивился. Инструмент, тем временем, перекочевал в руки Ульянки. Видать, роль часового ей пришлась по душе. Главное, чтоб не забросила, если вдруг надоест. 

-- А еще есть? -- неплохо бы оснастить такими дудками несколько человек, а там можно о наблюдательных постах подумать, и даже о патрулировании лагеря. Какая-никакая, а система оповещения получается.

-- Должны быть на складе, -- кивнула Ольга.

\paragraph{}<++>

Время обеденное, однако обедать никто не спешил. Электроник, после того, как мы забрали компас, вернулся к возне с передатчиком. Вожатая, в кои-то веки, исполняла свои обязанности организуя быт в переоборудованном под убежище, административном корпусе. Ей помогали девушки, за вычетом умаявшейся Слави. Витька тоже решил сделать перерыв в укреплении окон. Так, втроем, мы и присели попить чайку в развороченной столовой. Чуть погодя, присоединилась и Виола.

-- И что ты будешь делать, если найдешь их логово? -- Славя опять нервно теребила косу и как-то осуждающе глядела.

-- Да видно будет, -- произнес я, поправив стоящее у бедра ружье. --  Перестреляю, сколько смогу.

Сидящая напротив, медсестра приподнялась и, перегнувшись через стол, пощупала мой лоб. В который уже раз. Ну сколько можно?

-- Пионер, а ты не заболел? -- Отняв ладонь, скрестила руки на груди Виола. -- Откуда такая кровожадность?

-- А если их там пара десятков? -- Продолжала убеждать Славя, --  патронов у тебя сколько? 

-- Если много, отступлю, и будем думать. -- Не скажу, что перспектива бродить по туманным чащобам меня радовала, -- не ждать же, пока эти твари сюда толпой полезут?

-- Ну так защиту надо строить, -- повысила голос Виола, -- а не по лесам бегать! 

С обороной лагеря, честно говоря, все было плохо. Четверо парней, а все остальные -- женщины и дети. Потому и хотелось устроить налет на вражье логово, посеять там панику и разрушение, чтобы морлоки и думать о нас боялись. С другой стороны, я и сам прекрасно понимал безумие идеи. Двустволка -- не великой огневой мощи оружие, и патронов маловато. Вот если взять с собой Витьку и <<кибернетиков>>, вооружить их ударно-дробящим и колюще-режущим\ldots Можно еще коктейлей из бензина с маслом намешать, да устроить развеселый поджог. 

А с другой стороны, есть опасность заплутать в тумане. Даже с компасом. Или опять какая-нибудь аномальная активность проявится, и останется лагерь вовсе без защитников. 

Пойти с Витькой, да попросить Славю в качестве проводника выступить? Хотя, пожалуй, она в прошлый раз со мной по лесу набегалась. 

Рыжий сосредоточенно хлебал чай. С самого начала обсуждения он не проронил ни слова. Похоже, догадывается, что я собирался предложить.

-- Вот что, -- видя мои размышления, перешла в наступление Виола, -- занимайся защитой, а <<кибернетики>> наши пусть связь налаживают. 

-- Хе, -- уперев локоть в столешницу, я обхватил ладонью лицо, -- покомандовать захотелось?

-- Ну что ты опять ершишься, -- поддержала медсестру Славя, -- никто тобой не командует! 

\paragraph{}<++>

А обороной заняться все-таки пришлось. Честно говоря, о том, чтобы более-менее серьезно укрепить весь лагерь, речи даже не шло. Все упиралось в малочисленность ``гарнизона'' и скудное вооружение. А любые пассивные защитные меры -- ничто, при отсутствии активных. 

Первой мыслью было построить дополнительную ограду вокруг административного корпуса.  Да и столовую с медпунктом неплохо бы огородить. С другой стороны, кто всем этим будет заниматься и сколько времени займет строительство? С материалами тоже вопрос. 

Поглядев в сторону музыкального кружка, я почесал затылок. Лесок-то внутри периметра основательно затрудняет видимость. Вырубить его надо будет. Заодно и ресурсы на укрепления появятся. Да и дровишки не помешают. Но работы тут явно не на один день. 

Но пока я сделал немного. Импровизированная сигнализация из кастрюль, кружек и прочей столовой утвари. Растянул поперек дороги проволоку, прикрепив к одному концу погремушку из того, что нашлось на кухне и удалось отбить у тети Клавы. 

Где-то просто развесил кастрюльки так, что если кто-нибудь полезет через кусты, то наделает шума. С другой стороны, полезность такой сигнализации, когда поднимется ветер будет, пожалуй, отрицательной.

Кое-где просто натянул проволоку и воткнул в землю несколько заточенных колышков. Глядишь, запнется зубастая зверушка, да напорется. 

В голове крутились идеи о страшных, виденных в кино, ловушках вроде самозатягивающихся петель, падающих бревен с кольями, волчьих ямах\ldots Да уж, реальность -- не кино. Попробуй тут такое бревешко взгромозди на высоту, закрепи, да еще механизм придумай, чтоб оно действительно упало куда нужно.

-- Ты есть будешь? -- окликнула меня показавшаяся из-за угла нашего убежища Славя.

-- О, обед уже? 

-- Так ужинать пора, пошли, -- взмахнула рукой девушка и скрылась из виду. 

Время, действительно, было уже вечернее. 

\paragraph{}<++>

Вечер сменился ночью. Удивительно, но никаких ужасов это не принесло. Просто стало темнее, и население лагеря, прекратив все работы, собралось в административном корпусе, худо-бедно укрепленном за день. Тесновато -- здание не такое уж большое. 

Разместились, как говорится, мальчики -- налево, девочки -- направо. Немногочисленной мужской части населения достался растерзанный не так давно кабинет директора. Помимо меня, <<кибернетиков>> и Витьки, место нашлось и троим мелким, чьих имен я не знал. 

Пионеры и немногочисленный персонал, видно, за день умаялись -- разговоров не слышно. Мне, однако, спать не хотелось: усевшись у входной двери, прислушивался, не зазвенит ли развешенная по лагерю посуда. Как-то слишком уж тихо вокруг. 

Внутри здания темно: свечки приходилось экономить. На улице темень, хоть глаз выколи: ни звезд, ни луны. А как только начало темнеть, дымка вновь превратилась в туман. Даже от одной мысли о выходе наружу становилось не комфортно.

Подумалось, а не пойти ли в комнату, да попытаться уснуть? Странные, тревожившие меня в этом месте сны прошлой ночью не приходили. Поднявшись с пола, я подхватил ружье и двинулся было в кабинет-спальню, как услышал странный звук. Вдалеке что-то постукивало. 

-- Эй, вы слышите?! -- воскликнул я, ворвавшись в комнату.

Кто-то замычал и заворочался, послышалось невнятное бормотание, кажется, ругательства. Стук, тем временем, приближался и все больше напоминал нечто знакомое.

С кровати вскочил Электроник. В соседней комнате, очевидно, звук также не остался незамеченным. 

-- Да это же поезд! -- пришло, наконец, понимание. -- Я наружу, запритесь тут!

Сыроежкин начал тормошить пионеров, а я рванул к выходу. И, буквально, лоб в лоб столкнулся с кем-то в коридоре. 

-- Стой, куда?! -- Славя, ну кто же еще\ldots 

-- Поезд там, -- протиснулся я мимо девушки. 

Стук колес становился все громче. Повернув ключ в замке, наконец, распахнул дверь и выскочил наружу. 

Ни черта не видно, проклятый туман. Или домики заслоняют весь вид? Следом за мной выскочила и Славя и кто-то еще.

-- Да куда лезете? -- бросил я, обернувшись. -- Запритесь!

Вопреки опасениям, все население лагеря за мной не увязалось. Только неугомонная Славя и, как это ни странно, Лена.

-- Блин, за руки держитесь, не хватало тут еще потеряться\ldots -- сняв с кармана фонарь, я щелкнул кнопкой. -- Ежики в тумане.

Несколько шагов, и мы на площади. Противоестественная тишина контрастировала со стуком колес вдали. Где-то далеко мелькали огни -- должно быть, прожектор локомотива.

-- С пристани должно быть лучше видно, -- заметила Славя и потянула меня за руку. 

Идти совсем недалеко и места знакомые, но мгла вокруг приятных мыслей не навевала. Сделав пару шагов, я остановился. 

-- Славя, за пояс мой держись, -- скинув из-за спины ружье, взял его наизготовку, уложив цевьем на кулак с зажатым фонарем.

Так, гуськом, мы и двинулись к пристани. Дошли аккурат к тому моменту, когда состав проходил мост -- свет прожектора локомотива был хорошо виден в тумане, эдакий световой конус. И тусклая световая полоса на самом составе, должно быть, свет окон. Поезд-то, похоже, пассажирский. 

Какие-то несколько секунд, и свет растворился в ночи. Остался лишь удаляющийся звук и тусклый лучик фонаря. Втроем мы стояли, опираясь на поручни пристани и вглядывались во мглу. 

-- Уехал, -- почти шепотом произнесла Лена.

-- Угу, -- согласился я, собираясь сказать что-то еще, но тут вдалеке  грохнуло, заскрежетало и стук колес прервался. 

Мы поняли: случилось нечто ужасное.

\chapter{День шестой}

Утра дожидались с нетерпением. Почти всю оставшуюся ночь звучали разговоры вполголоса и перешептывания. Кто-то предлагал даже идти на место крушения немедленно. Всерьез, к счастью, подобные инициативы не воспринимались.

Вернувшись из ночной вылазки, я все-таки решил хоть немного вздремнуть, ведь утром нужно проводить спасательную операцию. Обязательно нужно. 

Плюхнулся на кровать не раздеваясь, положив сбоку двустволку. Кое-как пристроил содержимое карманов так, чтобы оно не мешало лежать и не впивалось в тело. Но сон  не шел. Не то чтобы мешали разговоры пионеров, но сна не было ни в одном глазу. Так и валялся, ворочаясь с боку на бок.

Незаметно подкралось утро. Кромешная тьма постепенно рассеялась. Нет, солнечные лучи не пробивались сквозь заколоченные окна, но разглядеть что-нибудь на расстоянии вытянутой руки уже было можно.

Удивительно: ночь прошла, но на лагерь так никто и не напал. Или, может быть, все твари отбыли на место аварии, пожирать несчастных пассажиров поезда?

В соседней комнате кто-то уже встал -- слышались шаги и копошение. Быстренько обувшись, я выбрался в коридор. Посветил фонарем в женскую комнату, и увидел там уже одевшуюся в спортивную форму Славю.

-- Привет, собрался уже? -- обернулась ко мне девушка.

-- Угу, -- хмыкнул я. -- То есть куда собрался?

Брови пионерки удивленно взлетели, а рука потянулась к переброшенной через плечо косе. 

-- Так к поезду. Сейчас, вот, Виола проснется\ldots

-- Стой, а ты-то куда собралась?

В обеих комнатах послышалось сонное ворчание и шарканье. Похоже, что всех разбудили. Да и рассвело уж окончательно. Щелкнув кнопкой, я убрал фонарь в карман.

\paragraph{}<++>

В который раз меня посетила мысль, что обживать административный корпус было ошибкой. А все потому, что еду готовить все равно приходилось в столовой, предварительно проверив ее по всем правилам, не заползла ли внутрь ночью какая-нибудь дрянь и не притаилась ли в темном уголке. 

Проверили столовую? Теперь подежурить возле умывальников на время водных процедур, покуда пионеры изводят остатки воды из баков-накопителей.

-- Да щас, -- от перспективы стоять и приглядывать за детским садом, буквально, перекашивало. -- Ульянка, сюда иди! Горн твой где? Дуди, если что, а я лодку пошел готовить.

-- Семен, а есть ты не будешь? -- обнаружилась  Ольга, также одевшаяся по-спортивному.

-- Некогда, -- оглядев собравшийся возле умывальников народ, заприметил там рыжего. -- Витька, быстрей давай!

Кибернетиков кое-как после подъема проинструктировал, чтоб наготове были, да ударно-дробящее под рукой держали. Точней -- Электроника, Шурик-то весь в мыслях о доработке передатчика и желания куда-либо идти не проявляет. 

У вожатой, вроде бы, стали проявляться какие-то зачатки разума. Да и Славя останется в лагере, а у нее с организаторскими способностями все в порядке.

Спасательная команда образовалась небольшая: я, Витька и Виола. Больше в лодке не поместится, а меньше -- бессмысленно. 

От добровольцев-то отбою не было, каждый хотел принять участие. Кроме, пожалуй, Шурика и поварихи.

\paragraph{}<++>

-- Глянь, даже тут эта дрянь\ldots -- Витька зачерпнул пригоршню воды за бортом и начал ее разглядывать. -- А мы эту воду набирали, еду готовили\ldots 

Действительно, вода приобрела сероватый оттенок. Как, впрочем, и все вокруг. Серо, пресно и тихо.

-- И дышим, -- добавила Виола. -- Весь этот туман, -- это что, по-твоему? 

Промолчав, я плюнул за борт и продолжил грести. Миновали уже небольшие, но поросшие лесом островки. 

-- Так что, -- не унималась медсестра, -- Семен, может быть, ты еще что-нибудь вспомнил? 

-- Например? -- пропыхтел я, налегая на весла.

-- Когда все это случилось, ты о морлоках рассказывал. Я-то их не видела, но если ребята говорят\ldots Славяна тогда заикнулась, что они оттуда же, откуда и ты. Не подскажешь, что она имела в виду?

-- Да шпион он американский, -- Витька оторвался от созерцания пригоршни воды, и перевел взгляд с меня на Виолу. -- Нож у него импортный. И фонарь тоже. 

-- Ну все, раскусил, -- перестав грести, я поерзал на лавке. -- Сейчас вот весло из уключины вытащу, и как дам по наглой рыжей морде! А вообще, давай, греби-ка теперь ты, коли такой бдительный.

-- Так  что, Семен? -- похлопала меня по колену медсестра, когда я уселся рядом с нею на лавку. 

А правда, что? Откуда все эти обрывочные знания, то и дело всплывающие из недр подсознания, неясные образы и воспоминания\ldots

-- Да никакой я не шпион! -- лучше покамест изображать амнезию. По крайней мере, до тех пор, пока сам хоть как-то не разберусь в происходящем. -- А так, урывками что-то вспоминается. Не знаю, откуда. -- В общем-то, против истины не сильно и погрешил.

\paragraph{}<++>

Лодку на берег вытаскивать не стали. Тем более, что тащить ее на насыпь ни у кого желания не возникло. Вбили в землю колышек, да пришвартовались веревкой, предусмотрительно захваченной из лагеря.

-- Ни черта не видно, -- следов аварии я пока не углядел. Рельсы уходили вдаль и терялись в тумане. -- Поезд вон в ту сторону шел, идемте.

Похоже, что Витька обиделся за обещание огреть веслом, и шел молча. Виола тоже молчала и с подозрением косилась по сторонам. Нервничает, что не мудрено.  

-- Мне кажется, или туман густеет? -- медсестра повела плечами и поежилась. -- Как-то прохладно становится.

-- Виола, за мной держись, -- скинув ружье из-за спины, лязгнул замком и проверил патронники. -- Витька -- замыкающим.

-- Угу, -- буркнул пионер. -- Пошли уже.

Идти по шпалам было неудобно. Сквозь мелкую гальку насыпи пробивалась трава, а колючки репейника цеплялись за штаны. 

-- Страшновато как-то, -- забеспокоилась медичка. -- А если туман усилится?

-- Не заблудимся, -- отмахнулся я, -- рельсы прямые, мимо моста не пройдем. А там и лодка. 

Впереди показалось что-то темное.  

-- Да это ж поезд! -- Витька дернулся было вперед, но был схвачен мной за рубашку. 

-- Куда?! Под ствол не суйся.


Картина открылась  нам, прямо скажем, жутковатая. Поезд сошел с рельс и завалился набок. Несколько вагонов в середине состава буквально сплющились в гармошку. Кругом осколки стекла и какие-то отвалившиеся детали. Кое-где на земле видны какие-то маслянистые пятна. Сильно воняет соляркой. Удивительно, как это все не загорелось, а то и не рвануло.

-- Эй, есть кто-живой? -- позвала Виола. В ответ послышался какой-то скрип. Легкий шорох и все стихло,

Дошли мы уже до середины состава, но признаков жизни так и не встретили: ни стонов раненых, ни мольбы о помощи. Вся та же мертвая тишина, иногда перемежаемая тихим шорохом.

-- Там кто-то есть, -- Витька нервно озирался, перехватив поудобнее топор. -- Почему они не выходят?

Внутрь заглянуть не получалось. Подойдя к вагону, я прикидывал, как бы на него вскарабкаться. Как назло, зашли мы со стороны крыши. Подумав, достал нож и постучал по обшивке рукояткой, прислушавшись. Внутри, как-будто, что-то пошевелилось. Стукнул еще пару раз, но так больше ничего не услышал.

-- Не нравится это мне, -- поморщился я, убирая нож. -- Пошли дальше.

-- Угу, -- кивнул Витька. -- Слышь, Семен, поезд-то какой здоровый!

Подходя к началу состава, мы увидели странное: такое впечатление, что рельсы внезапно кончились. Вместе с насыпью. А поезд ухнул вниз, перевернувшись в процессе. 

-- Это как вообще? -- подойдя к краю насыпи, медсестра удивленно глядела вниз. -- Почему здесь шел поезд, если пути не достроены? 

Краем глаза заметил какое-то копошение в кабине лежащего на боку тепловоза. По спине, как говорится, побежали мурашки. 

-- Твою мать, -- выдохнул я, отталкивая медсестру. В воздухе что-то коротко свистнуло, а следом грохнул дробовик. Выстрелил я, буквально на рефлексах, не думая. 

Что-то истошно завизжало, а я рванул вперед. Визг перешел в хрип и какое-то бульканье. 

-- Витька, охраняй врачиху! -- крикнул я, спускаясь по насыпи. 

Надо перезарядиться, пока есть возможность. Разомкнуть замок, вытащить гильзу\ldots 

-- Вот срань,-- ругнулся я сквозь зубы, -- опять подуло! -- гильза упорно не поддавалась. Вытащив нож, поддел за закраину, и патронник, наконец, освободился.

Тепловоз, изрядно смявшись, где-то на треть зарылся носом в землю, стекла вылетели. И не только стекла: впереди, метрах в десяти валялся труп. Мужчина в характерной голубой рубашке и темных, непонятного цвета, штанах. Машинист, должно быть. 

В кабине что-то хрипело, хлюпало и, как будто, скребло по металлу. Внутри темно, не видать ни черта. Сняв с кармана фонарь и переключив его в турбо-режим, я осветил разрушенную кабину. 

Так и есть: там, скорчившись, подыхал морлок. Видимо, дробь попала ему в грудь во время броска пики. То ли жизненно важные органы обнесло, то ли нулевка с короткого ствола слабовата\ldots Кровищи, однако, натекло очень много, и тварь явно не жилец.

-- Подыхай, подыхай, мразь, --  Существо заскулило и попыталось уползти вглубь кабины, но вместо этого, лишь поскребло конечностями металл. 

Поставив ногу на раму и ухватившись рукой за боковушку, я собирался влезть на борт тепловоза, но вовремя заметил еще один бледный силуэт. 

Вместо того, чтобы метнуть костяную пику, вторая тварь, видно, решила покараулить сверху и, как только представится такая возможность, пойти в рукопашную. Или, быть может, просто трусливо пряталась.  

-- Ага, попалась, -- прошептал я, вскидывая ружье. Однако, существо, словно почувствовав опасность, с диким верещанием прыгнуло с тепловоза на землю и рвануло на четвереньках прочь. Упускать врага я не собирался, и резво втащив себя на борт тепловоза, вновь вскинул ружье. 

Дробь ударила морлока в спину, видимо, задев позвоночник. Неуклюже кувыркнувшись, уродец рухнул на припорошенную беловато-серым землю. Пытаясь ползти на передних конечностях, он жалобно скулил.

Окинув взглядом окрестности, я увидел как выше на насыпи вжалась спиной в крышу лежащего на боку вагона Виола. Витька же, агрессивно выставив топор вперед, ожидал нападения.

-- Эй, товарищи! -- позвал я, -- идите сюда, кое с кем познакомлю. 

\paragraph{}<++>

-- Вот мрази-то, а! -- Рыжий глядел на труп машиниста со смесью жалости и омерзения. -- Они ж ему пальцы на обеих руках отгрызли!

-- А ты еще добивать не хотел и рожу кривил, -- я зло глянул на пионера. -- Зоозащитник хренов. 

Бледная как мел Виола склонилась над телом и что-то рассматривала. 

-- Похоже, что он мертвый уже был, -- пробормотала медсестра, -- когда они жрать его начали. -- Голова проломлена и шея сломана, видно, при падении. 

-- Похоронить  надо, -- подал светлую идею Витек.

-- А если там кто-то живой еще остался? -- Виола выпрямилась и кивнула в сторону состава. -- Может, еще что-то можно сделать\ldots

-- Кстати, надо карманы его проверить, -- спохватился я. -- Документы какие-то должны быть. 

Документов не нашли, видно, остались они где-то в кабине. Добычей стала смятая пачка сигарет неизвестной мне импортной марки, зажигалка, которую я, недолго думая, присвоил, и связка ключей. 

Было решено влезть на борт поезда, и двигаться от тепловоза к хвосту состава, осматривая вагоны и прислушиваясь, не подает ли кто признаков жизни.

-- К окнам не нагибайтесь, -- предостерег я, -- внутри еще твари могут быть. И осторожней, внутрь не провалитесь.

\paragraph{}<++>

-- Слышишь? -- полушепотом позвал меня Витька, -- там кто-то шевелится.

Действительно, в глубине вагона что-то перемещалось. В то, что это кто-нибудь из выживших, верилось с трудом. Дошли мы уже до середины состава. По мере продвижения заглядывали в окна, но видели там лишь недвижные тела. То, что никто не выжил при крушении, поражало. В конце концов, не самолет же.

-- Будьте тут, -- я посветил фонарем внутрь в надежде обнаружить источник шума. -- Сейчас попробую спуститься. --  Проверить все равно надо.

Прыгать сломя голову внутрь выбитого окна не хотелось. Из недр вагона вновь донесся шорох. Велик соблазн пальнуть на звук, но вдруг это кто-нибудь из жертв аварии? Может быть, серьезно раненый.

Как не приглядывался, но отчетливо так ничего и не разглядел. Лишь краем глаза заметил, как что-то прошмыгнуло в противоположном конце вагона.

-- Во, через тамбур попробую, -- решил я, -- там посвободнее. 

Действительно, задний тамбур пострадал гораздо меньше остального вагона. Может быть за счет того, что там особо нечему было сминаться и отваливаться кроме, собственно, стен. Дверь изрядно перекосило, но от пары сильных ударов ногой, она с грохотом провалилась внутрь.

Сюрприз ждал сразу же, как только я спустился и попытался заглянуть в коридор. Труп мужика, который, судя по всему, решил покурить или вышел в туалет. В момент аварии его, буквально размазало о двери. 

-- Вот дерьмо, -- ругнулся я сквозь зубы, пытаясь пролезть через кровавое месиво, не перемазавшись с головы до ног. 

Дверь туалета приоткрыта, и подсветив фонарем, я увидел внутри еще один труп. Пострадал он гораздо меньше того, кто ждал своей очереди.

-- Семен, что там? -- Виола явно беспокоилась. -- Нашел кого?

Со стороны купе проводника донеслось тихое постукивание. 

-- Тут кто-то есть, -- крикнул я в ответ. -- Сейчас купе проверю!

Дверь, однако, не поддавалась. Похоже, что была заперта изнутри. И хотя ее несколько и покорежило, но не настолько, чтобы повредить запорный механизм. 

-- Эй там, внутри! Отодвинься от проема, я сейчас замок отстрелю! -- в ответ донеслось лишь какое-то вялое постукивание. -- Слышишь меня? 

Заткнув мизинцем левой руки левое ухо, а правое плотно приложив к плечу, дабы избежать контузии, выстрелил в замок с полусогнутой руки. 

Кашляя от повисшего в тесном помещении порохового дыма, я кое-как сдвинул покореженную дверь. Луч фонаря выхватывал из тьмы приборную панель, отражался от битого стекла и, видимо, высыпавшихся из шкафа подстаканников. На полке пусто.

-- Эй! -- позвал я, обшаривая лучом тесную каморку. В ответ вновь раздался тихий стук. Удивительно, как мы услышали его снаружи вагона. Хотя, не только его, до моего спуска, в вагоне определенно шнырял кто-то еще.

Источник звука я заметил не сразу. Скорчившееся  тело в темной форме проводника под столом, как будто сливалось с темнотой. Да и, в таком сжавшемся виде, мало напоминало человека. 

-- Ты живой там? -- сдвинув дверь настолько, чтобы можно было протиснуться внутрь, я начал нелегкий спуск. Как бы не приземлиться на пострадавшего. Или на стол не напороться.

Оказавшись на полу, а точнее стене вагона, я присмотрелся к жертве аварии. Проводница, коротко стриженая женщина, буквально, сжавшаяся в комок, сидела на стене под столом, плотно охватив руками колени. А звук она производила, видимо, стукаясь головой о столешницу при покачивании.

-- Мля. -- Меня передернуло. -- Неужели и ты труп\ldots

-- Тук-тук-тук.

Протянув руку, тронул шею проводницы. Вроде, теплая. Так, нащупать сонную артерию\ldots Есть пульс!

\paragraph{}<++>

Однако, радоваться было рано. Все мышцы пострадавшей были жутко напряжены или, как будто, сведены судорогой. С трудом мне удалось развести ее руки, сжимающие лодыжки. 

Осторожно ощупал голову. Да вроде целая. 

-- Может мне спуститься? -- крикнула снаружи Виола.

-- Оба спускайтесь, -- я окинул взглядом темную, тесную и покореженную каморку. -- Вытаскивать будем! 

В общем-то, сказать проще, чем сделать. Покуда большая часть спасательной команды проникала в вагон, я пытался погрузить проводницу на найденное здесь же одеяло. 

Корячились долго, хотя женщина была довольно легкой и небольшого роста. Вытащить скрюченное тело в коридор, сражаясь с подклинивающей дверью, стараясь не стукать спасаемую о выступающие поверхности, пробраться по узкому и низкому коридорчику, хлюпая по растекшейся из баков воде, преодолеть проем с месивом из человеческой плоти\ldots Ружье, висевшее за спиной, мешало преизрядно. 

Добрались до тамбура. Теперь нужно втащить недвижное тело наверх. Сказать -- это проще, чем сделать. 

-- Слушайте, пионеры, -- проявила, вдруг, чудеса наблюдательности Виола, -- а если нижнюю дверь растворить и землю слегка копнуть\ldots Тут, вроде, места больше.

-- Витька, вылезай, -- скомандовал я, оценив по достоинству новый план, -- бери топор и начинай подкапывать снаружи\ldots

-- Может лопату поискать? -- предложил рыжий.

-- Где? В таком-то бардаке! Давай, пошел, я сверху прикрою.

\paragraph{}<++>

-- Вот это у нее в бедре сидело, -- Виола продемонстрировала небольшой, сантиметров десяти в длину, сборный костяной дротик с, как будто, бумажным, оперением. Покуда Витька занимался подкопом, медсестра зря времени не теряла. -- Похоже, отравленный, оттого ее так и скрутило.

Пострадавшая лежала на взятом из ее купе одеяле. Кое-как оказали первую помощь: растерли мышцы и придали нормальное горизонтальное положение. Медичка что-то вколола. Но в себя жертва катастрофы так и не приходила. Удивительно, как дротик не разломали, когда вытаскивали проводницу из вагона. 

-- Ой-ей, -- тревожно воскликнул Витька, -- там, кажется, опять эти\ldots

Действительно, внутри вагона слышался шорох и какое-то попискивание. Что-то стукнуло и разбилось, за чем последовало недовольное верещание. 

-- Ну я им бошки посношу, -- щелкнув предохранителем, я направился к подкопу. -- Эй, уроды, готовьтесь!

Верещание сменило тональность, что-то опять упало, звук стал удаляться и вскоре затих. Притаились?

-- Эй, да они сами нас боятся, -- пришло, вдруг, понимание. -- Трусливые твари!

-- Семен, стой! -- окликнула меня Виола, -- нужно раненную в лагерь доставить. 

%%%%%%%%%%%%%%%%%%%%

-- А другие выжившие как? -- уходить мне категорически не хотелось. -- Не может быть, чтобы на весь состав уцелела одна проводница! 

Изнутри вагона вновь послышалось копошение и повизгивание. Удивительно, почему морлоки не напали, когда мы вытаскивали пострадавшую.

-- В общем, стойте тут, -- бросил я и полез в подкоп, -- осмотрюсь еще, а тварей перестреляю, сколько смогу. 

\paragraph{}<++>

Дело не заладилось сразу: пробираться внутри лежащих на боку вагонов не так просто. Проще всего было в купейных, можно ползти по боковушке, а сами купе осматривать через дверной проем. 

Тел без явной расчлененки хватало. Меньше всего должны были пострадать те, кто лежал на нижних полках. Но создавалось впечатление, что из тех, кто спал на своих местах, не выжил никто. Признаки, как говорится, на лицо: отвисшие челюсти, образующие жуткие гримасы, окоченение и  натеки. Где-то это все я уже видел.

Вспоминалась та злосчастная пятиэтажка. Осматривали мы ее долго: срезали двери бензорезом, осматривали квартиры, но живых так и не нашли. Все обнаруженные тела лежали на своих кроватях. Люди уснули и не проснулись. 

Осматривая очередное купе, пытаюсь ухватить еще хоть клочок тех воспоминаний. Несколько квартир были пусты, в смысле, без трупов. Может, хозяева ушли куда, перед тем, как дом черте-куда провалился? 

Внимание привлек шорох со стороны коридора и я вскинул двустволку. Однако, нападения не последовало. Прислушался, после чего опираясь кое-как на полки, выглянул в коридор. Никого. Похоже, просто скрип покореженного металла. 

В вещах специально не рылся, но борсетка валялась прямо под ногами и грех ее было не осмотреть. Паспорт гражданина Российской Федерации, деньги, вроде бы, отличающиеся от тех, что я привык видеть -- очень уж купюры неприличного достоинства, с кучей нулей, записная книжка с именами и телефонами\ldots И билет "Н-ск -- Москва" на шестнадцатое июля тысяча девятьсот девяносто шестого года. Билет я решил прихватить с собой.

-- Эй, Семен, -- донесся с улицы голос Витьки, -- ты как там? 

-- Нормально! -- откликнулся я и полез в коридор.  -- Жутко вообще, никого жи\ldots

Что-то ужалило в шею, руки подогнулись и я рухнул обратно в купе. Все мышцы скрутило судорогой, из глаз брызнули слезы и сил не осталось даже, чтобы сделать вдох. Шея горела огнем, словно ее облили кислотой. Где-то рядом, как будто, раздалось мерзкое хихиканье. 

Жечь начало все тело, а особенно глаза. Вдруг понял, что ничегошеньки не вижу, а в следующее мгновение потерял сознание. 

\paragraph{}<++>

Оглушительно ревела сирена. Пульт пылал красным от мигающих сигнальных ламп - похоже, сдохло все, что могло сдохнуть.

-- Не успеем уйти, -- донесся из-за спины девичий голос. -- Не могу стабилизировать канал!

Мерзко воняло горелой проводкой, кровью и паленым мясом. Где-то недалеко стучали автоматные очереди. Полыхнула зеленоватая вспышка, и кто-то истошно завопил совсем рядом. 

-- А-а-а-а-у-у! -- завыл я, поднимаясь на четвереньки. Глаза болят, словно их вскипятили, мышцы горят, но двигаться могу. Я все там же, внутри перевернутого вагона, в купе. Черт, да где ружье? Кто-то крадется в коридоре. Определенно, я слышал мерзкое хихиканье. 

Совершив, буквально, акробатический прыжок, в купе с визгом сваливается морлок, с ходу пытаясь достать своим утлым копьишком то место, где по его, уродца, прикидкам должно было лежать мое обездвиженное тело.

-- Семен, Семен?! --  голоса Витьки и Виолы. -- Что там?!

Хрипя от боли и злости, впечатываю локоть левой руки в бледное тельце мутанта. В удар вложена вся масса тела -- буквально подминаю врага под себя. Правая рука сама собой ложится на рукоять заткнутого за пояс ножа, выхватываю и бью одним движением.

Гадина пищит и пытается вырваться: ухватить зубам мою руку, сжимающую тонкую шею. Бью  ножом, еще, еще! Чем больше нанесешь ударов, тем быстрее сдохнет враг. Рукоятка теплая и мокрая, писк переходит в хрип. Еще удар в грудину. Выше и выше. Последний колющий удар приходится в горло. Убрать левую руку и резануть вверх, распахав трахею до подбородка.

-- Уф, -- выдыхаю и загоняю нож морлоку в глазницу. -- Контроль!

-- Семен! -- в голосе Виолы истерические нотки. -- Семен, отзовись! Вить, его, похоже сожрали\ldots

-- Да живой я! -- ору в ответ.  -- Сейчас вылезу, только ружье найду!

\paragraph{}

Сказать, что Виола с Витькой были поражены -- ничего не сказать. Перемазанный кровью с головы до ног пионер -- зрелище, думается, весьма занятное. 

-- Это\ldots -- Оторопело произнес Витек, указывая пальцем.  -- У тебя из шеи иголка торчит. 

-- Вот срань! --  Выругался я, вытащив и себя, очевидно, останки дротика. -- Думал, сдохну. -- Говорить было тяжеловато, начало колотить. А еще дико хотелось жрать. 

Назад шли быстро. Так быстро, насколько это возможно, когда приходится тащить на плечах парализованное тело. Хорошо, хоть весила проводница не много. 

Доверять медсестре ружье я не то, что не хотел, а попросту боялся: не умеет же пользоваться. И не то, что врага, а как бы себя не подстрелила\ldots Хотя, как показывает практика, из длинноствола по неосторожности стреляют, обычно не в себя, а в окружающих.

Вот с такими веселыми мыслями я и оглядывался назад. А еще радовался, что ружье опилил, в свое время: теперь хоть одной рукой им манипулировать более-менее удобно.о

Из головы никак не шел чертов бункер. Метались смутные образы и обрывки воспоминаний: палаточный городок, выросший у ограждения из колючей проволоки -- отчаяние и безысходность, красивый город на берегу моря -- отвращение и ненависть\ldots 

А вот и мост. Где-то здесь мы оставили лодку. 

\paragraph{}<++>

%%%%

-- То есть, на тебя не подействовал яд? -- осмотрев место прокола, медсестра протерла его проспиртованной ватой. -- Интересно. Кстати, как твоя покусанная нога поживает? 

Действительно, про раны на ноге я забыл, буквально, на второй день. Интересно получается. 

-- Да нормально. -- На всякий случай я закатал штанину и осмотрел ногу. Не осталось даже шрама. -- Яд подействовал, очень хреново было: парализовала, потом начало все тело жечь, думал, что глаза вскипят.

Спасенная проводница несколько раз приходила в себя, пыталась что-то сказать, но очень быстро теряла силы и отключалась. 

На лодке до лагеря мы добрались без приключений. На пристани нас встретило, похоже, все население лагеря. Дальше последовала суета, расспросы, размещение пострадавшей в медпункте, куда в скором времени отправили и меня. Поначалу, собрался там почти десяток человек, если не считать пациентки: собственно Виола, Ольга, которой было очень интересно происходящее у состава, Шурик с Электроником, Славя и Лена, даже непоседливая Ульянка прибежала, ожидая захватывающего дух рассказа\ldots 

В общем, тесновато стало. Немного поразмыслив, медсестра решительно выставила всех лишних вон. Славя вызвалась помочь с осмотром и получила разрешение присутствовать. 

-- Слушайте. -- пришла мне в голову идея. -- А если у меня к этой дряни морлоков иммунитет, может из моей крови попробовать противоядие синтезировать?

Виола в сердцах хлопнула рукой себя по лбу и прикрыла ладонью глаза. 

-- Ты что, думаешь, что у меня тут биохимическая лаборатория? Очнись, Семен, -- это пионерлагерь!

-- И что, -- вмешалась Славя, -- так и оставим ее умирать?

В голове словно щелкнуло. Аудитория, только сидят в ней не молоденькие, разодетые, кто во что, студенты, а люди в военной форме. Седой человек с кибернетическим протезом правой руки водит лазерной указкой по проецируемому на экран изображению\ldots

-- По идее, -- заговорил я, -- часов через пять, должно отпустить. Оно так действует: сильные болевые ощущение при попадании в кровь, дальше растворение за десятки секунд в зависимости от места попадания, потом судороги и несколько часов практически полного паралича. В зависимости от реакции организма, возможна потеря сознания, аллергическая реакция, влекущая отек легких\ldots

Девушки замерли и с оторопью переглянулись.

-- А еще что-нибудь полезное можешь вспомнить? -- Славя обернулась от кушетки, стремительно приблизилась и уставилась мне в глаза. -- Откуда это все, Семен? Пожалуйста, вспомни!

-- Не знаю. -- Чем больше я пытался окунуться в свое прошлое, тем больше путались мысли. -- Когда очнулся в автобусе, то думал, что ехал к родителям. Вообще, хорошо помню то, что было где-то до семнадцати лет: юность и школа там\ldots Дальше начинается мешанина всякого разного.

-- Так ведь тебе сейчас явно не больше семнадцати. -- Поднялась, сидевшая на краешке стола, Виола. -- Как ты можешь помнить, что было после? Бред какой-то. Я, вроде бы, не дура и понимаю, что творится тут нечто из ряда вон: туман этот, морлоки эти жутки, поезд этот\ldots

-- Кстати. -- Встрепенулся я, и вытащив из кармана билет, протянул его медичке. -- Если уж здесь все люди взрослые и неглупые\ldots

Истерик и воплей в стиле ''я не верю, этого не может быть'', действительно, не последовало. Девушки осмотрели доказательство темпорального разлада и молча уставились на меня. Прошло несколько минут и, видимо, наглядевшись, заговорила Славя. 

-- В общем-то, я что-то такое подозревала. Когда ты ружье пилил и рассказывал разное\ldots

-- Ладно. -- Поморщилась Виола. -- Поезд из другого времени. А ты ко всему этому, Семен, каким боком? Допустим, что ты тоже иновременец, и что дальше?

-- А то, -- вскочил я со стула, -- что в автобус я садился, будучи несколько постарше. А теперь не помню, куда и зачем вообще ехал. Помню этот долбаный город, населенный морлоками и, одуревшими от голода и влияния разлома, людьми. Когда засыпаю, то вижу своего напарника Игоря, что тащит меня раненного на себе через этот сраный туман, помню бричера нашего, Сашку. А чьего ``нашего'' - нет. Лица, образы, иногда имена -- вот и все\ldots

Пионерка с медсестрой переглянулись и вновь вперили в меня взгляды. 

-- А еще, -- продолжал я, -- когда я поезд осматривал, вспомнил: если спишь в момент перехода, то ты труп. А я спал, когда сюда провалился.
\end{document}
